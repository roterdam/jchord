\xname{dynamic_analysis}
\chapter{Dynamic Analysis}
\label{chap:dynamic-analysis}

This chapter describes all aspect of dynamic analysis in Chord.
Section \ref{sec:writing-dynamic-analysis} describes how to write a dynamic analysis,
Section \ref{sec:running-dynamic-analysis} describes how to compile and run it,
and Section \ref{sec:instr-events} describes common dynamic analysis events
supported in Chord.

\section{Writing a Dynamic Analysis}
\label{sec:writing-dynamic-analysis}

Follow the following steps to write your own dynamic analysis.

Create a class extending \code{chord.project.analyses.DynamicAnalysis} and override
the appropriate methods in it.
The only methods that must be compulsorily overridden are method \code{getInstrScheme()},
which must return an instance of the ``instrumentation scheme" required by
your dynamic analysis (i.e., the kind and format of events to be generated during an
instrumented program's execution)
plus each \code{process<event>(<args>)} method that corresponds to event {\tt <event>}
with format {\tt <args>} enabled by the chosen instrumentation scheme.
See Section \ref{sec:instr-events} for the kinds of supported events and their formats.

A sample such class \code{MyDynamicAnalysis} is shown below:

\texonly{\newpage}

\begin{center}
\begin{framed}
{\small
\begin{verbatim}
    import chord.project.analyses.DynamicAnalysis;
    import chord.instr.InstrScheme;

    // ***TODO***: analysis won't be recognized by Chord without this annotation
    @Chord(name=<name-of-analysis>)    
    public class MyDynamicAnalysis extends DynamicAnalysis {
        InstrScheme scheme;
        @Override
        public InstrScheme getInstrScheme() {
            if (scheme != null)
                return scheme;
            scheme = new InstrScheme();
            // ***TODO***: Choose (<event1>, <args1>), ... (<eventN>, <argsN>)
            // depending upon the kind and format of events required by this
            // dynamic analysis to be generated for this during an instrumented
            // program's execution.
            scheme.set<event1>(<args1>);
            ...
            scheme.set<eventN>(<argsN>);
            return scheme;
        }
        @Override
        public void initAllPasses() {
            // ***TODO***: User code to be executed once and for all
            // before all instrumented program runs start.
        }
        @Override
        public void doneAllPasses() {
            // ***TODO***: User code to be executed once and for all
            // after all instrumented program runs finish.
        }
        @Override
        public void initPass() {
            // ***TODO***: User code to be executed once before each instrumented program run starts.
        }
        @Override
        public void donePass() {
            // ***TODO***: User code to be executed once after each instrumented program run finishes.
        }
        @Override
        public void process<event1>(<args1>) {
            // ***TODO***: User code for handling events of kind <event1> with format <args1>.
        }
        ...
        @Override
        public void process<eventN>(<argsN>) {
            // ***TODO***: User code for handling events of kind <eventN> with format <argsN>.
        }
    }
\end{verbatim}
}
\end{framed}
\end{center}

\section{Compiling and Running a Dynamic Analysis}
\label{sec:running-dynamic-analysis}

Compile the analysis by placing the directory containing class \code{MyDynamicAnalysis} created
above in the path defined by property \code{chord.ext.java.analysis.path}.

Provide the IDs of program runs to be generated (say 1, 2, ..., M) and the command-line arguments to be
used for the program in each of those runs (say \code{<args1>}, ..., \code{<argsM>}) via properties
\code{chord.run.ids=1,2,...,N} and \code{chord.args.1=<args1>}, ..., \code{chord.args.M=<argsM>}.
By default, \code{chord.run.ids=0} and \code{chord.args.0=""}, that is, the program will be run only
once (using run ID 0) with no command-line arguments.

To run the analysis, set property \code{chord.run.analyses} to \code{<name-of-analysis>}
(recall that \code{<name-of-analysis>} is the name provided in the \code{@Chord} annotation for class
\code{MyDynamicAnalysis} created above).

{\bf Note:} The IBM J9 JVM on Linux is highly recommended if you intend to use Chord for
dynamic program analysis, as it allows you to instrument the entire JDK; using any other
platform will likely require excluding large parts of the JDK from being instrumented.
Additionally, if you intend to use online (load-time) bytecode instrumentation in
your dynamic program analysis, then you will need JDK 6 or higher, since this functionality
requires the \code{java.lang.instrument} API with class retransformation support (the latter
support is available only in JDK 6 and higher).

You can change the default values of various properties for configuring your dynamic analysis;
see Section \ref{sec:scope-props} and Section \ref{sec:instr-props} in Chapter \ref{chap:properties}.
For instance:
\begin{itemize}
\item
You can set property \code{chord.scope.kind} to {\tt dynamic} so that the program scope
is computed dynamically (i.e., by running the program) instead of statically.
\item
You can exclude certain classes (e.g., JDK classes) from being instrumented by setting
properties \code{chord.std.scope.exclude}, \code{chord.ext.scope.exclude}, and
\code{chord.scope.exclude}.
\item
You can choose between online (i.e. load-time) and offline bytecode instrumentation by
setting property \code{chord.instr.kind} to {\tt online} or {\tt offline}.
\item
You can require the event-generating and event-handling JVMs to be one and the same (by setting
property \code{chord.trace.kind} to {\tt none}),
or to be separate (by setting property {\tt chord.trace.kind} to {\tt full} or {\tt pipe}, depending upon
whether you want the two JVMs to exchange events by a regular file or a POSIX pipe, respectively).
Using a single JVM can cause correctness/performance issues if event-handling Java code itself is instrumented
(e.g., say the event-handling code uses class \code{java.util.ArrayList} which is not excluded from program scope).
Using separate JVMs prevents such issues since the event-handling JVM runs uninstrumented bytecode (only
the event-generating JVM runs instrumented bytecode).
If a regular file is used to exchange events between the two JVMs, then the JVMs run serially:
the event-generating JVM first runs to completion, dumps the entire dynamic trace to the regular file,
and then the event-handling JVM processes the dynamic trace.
If a POSIX pipe is used to exchange events between the two JVMs, then the JVMs run in lockstep.
Obviously, a pipe is more efficient for very long traces, but it 
not portable (e.g., it does not currently work on Windows/Cygwin), and the traces cannot be reused
across Chord runs (see the following item).
\item
You can reuse dynamic traces from a previous Chord run by setting property
\code{chord.reuse.traces} to {\tt true}.  In this case, you must also set property
{\tt chord.trace.kind} to {\tt full}.
\item
You can set property \code{chord.dynamic.haltonerr} to {\tt false} to prevent Chord from terminating
even if the program on which dynamic analysis is being performed crashes.
\end{itemize}

Chord offers much more flexibility in crafting dynamic analyses.
You can define your own instrumentor (by subclassing \code{chord.instr.CoreInstrumentor}
instead of using the default \code{chord.instr.Instrumentor}) and your own event handler (by subclassing \code{chord.runtime.CoreEventHandler}
instead of using the default \code{chord.runtime.EventHandler}).
You can ask the dynamic analysis to use your custom instrumentor and/or your custom event handler by overriding
methods \code{getInstrumentor()} and \code{getEventHandler()}, respectively, defined in \code{chord.project.analyses.CoreDynamicAnalysis}.
Finally, you can define your own dynamic analysis template by subclassing \code{chord.project.analyses.CoreDynamicAnalysis}
instead of subclassing the default \code{chord.project.analyses.DynamicAnalysis}.

\section{Common Dynamic Analysis Events}
\label{sec:instr-events}

Chord provides support for instrumenting common dynamic analysis events.
The below table describes these events.

\begin{table}
\begin{center}
\htmlattributes*{table}{border="1" cellpadding="10"}
\begin{tabular}{|l|p{4.3in}|} 
\hline
{\bf Event Kind} & {\bf Description} \\
\hline
EnterMainMethod(\bt) & After thread \bt\ enters method \bm\ (in domain M).
\\
\hline
EnterMethod(\bm, \bt) & After thread \bt\ enters method \bm\ (in domain M).
\\
\hline
LeaveMethod(\bm, \bt) & Before thread \bt\ leaves method \bm\ (in domain M).
\\
\hline
EnterLoop(\bw, \bt) & Before thread \bt\ begins loop \bw\ (in domain W).
\\
\hline
LoopIteration(\bw, \bt) & Before thread \bt\ starts a new iteration of loop \bw\ (in domain W).
\\
\hline
LeaveLoop(\bw, \bt) & After thread \bt\ finishes loop \bw\ (in domain W).
\\
\hline
BasicBlock(\bb, \bt) & Before thread \bt\ enters basic block \bb\ (in domain B).
\\
\hline
Quad(\bp, \bt) & Before thread \bt\ executes quad at program point \bp\ (in domain P).
\\
\hline
BefMethodCall(\bi, \bt, \bo) & 
\begin{tabular}{p{4.3in}}
Before thread \bt\ executes the method invocation statement at program point \bi\ (in domain I) with this argument as object \bo. \\
{\bf Note:} Not generated before constructor calls; use BefNew event.
\end{tabular}
\\
\hline
AftMethodCall(\bi, \bt, \bo) &
\begin{tabular}{p{4.3in}}
After thread \bt\ executes the method invocation statement at program point \bi\ (in domain I) with this argument as object \bo. \\
{\bf Note:} Not generated after constructor calls; use AftNew event.
\end{tabular}
\\
\hline
BefNew(\bh, \bt, \bo) & Before thread \bt\ executes a \code{new} bytecode instruction and allocates fresh object \bo\ at program point \bh\ (in domain H).
\\
\hline
AftNew(\bh, \bt, \bo) & After thread \bt\ executes a \code{new} bytecode instruction and allocates fresh object \bo\ at program point \bh\ (in domain H).
\\
\hline
NewArray(\bh, \bt, \bo) & After thread \bt\ executes a \code{newarray} bytecode instruction and allocates fresh object \bo\ at program point \bh\ (in domain H).
\\
\hline
GetstaticPrimitive(\be, \bt, \bg) & After thread \bt\ reads primitive-typed static field \bg\ (in domain F) at program point \be\ (in domain E).
\\
\hline
GetstaticReference(\be, \bt, \bg, \bo) & After thread \bt\ reads object \bo\ from reference-typed static field \bg\ (in domain F) at program point \be\ (in domain E).
\\
\hline
PutstaticPrimitive(\be, \bt, \bg) & After thread \bt\ writes primitive-typed static field \bg\ (in domain F) at program point \be\ (in domain E).
\\
\hline
PutstaticReference(\be, \bt, \bg, \bo) & After thread \bt\ writes object \bo\ to reference-typed static field \bg\ (in domain F) at program point \be\ (in domain E).
\\
\hline
GetfieldPrimitive(\be, \bt, \bb, \bg) & After thread \bt\ reads primitive-typed instance field \bg\ (in domain F) of object \bb\ at program point \be\ (in domain E).
\\
\hline
GetfieldReference(\be, \bt, \bb, \bg, \bo) & After thread \bt\ reads object \bo\ from reference-typed instance field \bg\ (in domain F) of object \bb\ at program point \be\ (in domain E).
\\
\hline
PutfieldPrimitive(\be, \bt, \bb, \bg) & After thread \bt\ writes primitive-typed instance field \bg\ (in domain F) of object \bb\ at program point \be\ (in domain E).
\\
\hline
PutfieldReference(\be, \bt, \bb, \bg, \bo) & After thread \bt\ writes object \bo\ to reference-typed instance field \bg\ (in domain F) of object \bb\ at program point \be\ (in domain E).
\\
\hline
AloadPrimitive(\be, \bt, \bb, \bi) & After thread \bt\ reads the primitive-typed element at index \bi\ of array object \bb\ at program point \be\ (in domain E).
\\
\hline
AloadReference(\be, \bt, \bb, \bi, \bo) & After thread \bt\ reads object \bo\ from the reference-typed element at index \bi\ of array object \bb\ at program point \be\ (in domain E).
\\
\hline
AstorePrimitive(\be, \bt, \bb, \bi) & After thread \bt\ writes the primitive-typed element at index \bi\ of array object \bb\ at program point \be\ (in domain E).
\\
\hline
AstoreReference(\be, \bt, \bb, \bi, \bo) & After thread \bt\ writes object \bo\ to the reference-typed element at index \bi\ of array object \bb\ at program point \be\ (in domain E).
\\
\hline
%ReturnPrimitive(\bp, \bt) & Not yet supported.
%\\
%ReturnReference(\bp, \bt) & Not yet supported.
%\\
%ExplicitThrow(\bp, \bt, \bo) & Not yet supported.
%\\
%ImplicitThrow(\bp, \bt, \bo) & Not yet supported.
%\\
ThreadStart(\bi, \bt, \bo) & Before thread \bt\ calls the \code{start()} method of \code{java.lang.Thread} at program point \bi\ (in domain I) and spawns a thread \bo.
\\
\hline
ThreadJoin(\bi, \bt, \bo) & Before thread \bt\ calls the \code{join()} method of \code{java.lang.Thread} at program point \bi\ (in domain I) to join with thread \bo.
\\
\hline
AcquireLock(\bl, \bt, \bo) & After thread \bt\ executes a statement of the form ``monitorenter \bo'' or enters a method synchronized on \bo\ at program point \bl\ (in domain L).
\\
\hline
ReleaseLock(\br, \bt, \bo) & Before thread \bt\ executes a statement of the form ``monitorexit \bo'' or leaves a method synchronized on \bo\ at program point \br\ (in domain R).
\\
\hline
Wait(\bi, \bt, \bo) & Before thread \bt\ calls the \code{wait()} method of \code{java.lang.Object} on object \bo\ at program point \bi\ (in domain I).
\\
\hline
Notify(\bi, \bt, \bo) & Before thread \bt\ calls the \code{notify()} or \code{notifyAll()} method of \code{java.lang.Object} on object \bo\ at program point \bi\ (in domain I).
\T \\
\hline
\end{tabular}
\end{center}
\end{table}

