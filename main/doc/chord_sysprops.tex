% TODO: write some text at beginning of each subsection
% TODO: do cross-referencing (i.e. put "See Also" for properties used by multiple subsections)

\section{Chord System Properties}
\label{sec:chord-sysprops}

Users can use system properties to control Chord's execution.
The following properties are recognized by Chord.
The separator for list-valued properties can be either a blank space, a comma, a colon, or a semi-colon.
Notation {\tt [<...>]} is used in this section to denote the value of the property named {\tt <...>}.

\paragraph{Program Properties} 

This section describes basic properties of the given program, such as
its main class, the location(s) of its class files and Java source
files, and command-line arguments to be used to run the program.
\\[10pt]
\code{chord.work.dir}
\begin{quote}
{\bf Type:} location \\
{\bf Description:} Working directory during Chord's execution.  This is
usually the top-level directory of the given program. \\
{\bf Default value:} current working directory
\end{quote}

\code{chord.props.file}
\begin{quote}
{\bf Type:} location \\
{\bf Description:} Properties file loaded by Chord at the
beginning before doing anything else.  Any of the below properties may
be defined in this file to avoid defining them on the command line
(using the \code{-D<key>=<value>} format) every time Chord is run.
Each relative file/directory name in the value of any property defined
in this file is treated relative to Chord's working directory (which
is specified by property \code{chord.work.dir}). \\
{\bf Default value:} \code{[chord.work.dir]/chord.properties}
\end{quote}

\code{chord.main.class}
\begin{quote}
{\bf Type:} class \\
{\bf Description:} Fully-qualified name of the main class of the given program (e.g., \code{com.example.Main}).
\end{quote}

\code{chord.class.path}
\begin{quote}
{\bf Type:} path \\
{\bf Description:} Classpath of the given program.  It does not need to include
boot classes (i.e., classes in \code{[sun.boot.class.path]}) or
standard extensions (i.e., classes in {\tt .jar} files in directory
\code{[java.home]/lib/ext/}). \\
{\bf Default value:} {\tt ""}
\end{quote}

\code{chord.src.path}
\begin{quote}
{\bf Type:} path \\
{\bf Description:} Java source path of the given program. \\
{\bf Default value:} {\tt ""} \\
{\bf Note:} Chord analyzes only Java bytecode, not Java source code.  This property is used only by the task of converting Java source files into HTML files by program analyses that need to present their analysis results at the Java source code level (by calling method \code{chord.program.Program.g().HTMLizeJavaSrcFiles())}.
\end{quote}

\code{chord.run.ids}
\begin{quote}
{\bf Type:} string list \\
{\bf Description:} List of IDs to identify runs of the given program. \\
{\bf Default value:} {\tt 0} \\
{\bf Note:} This property is used only when Chord executes the given program, namely, when it is asked to compute the program scope dynamically (i.e., when \code{[chord.scope.kind]=dynamic}) or when it is asked to execute a dynamic program analysis. 
\end{quote}

\code{chord.args.<id>}
\begin{quote}
{\bf Type:} string \\
{\bf Description:} Command-line arguments string to be used for the given program in the run having ID {\tt <id>}. \\
{\bf Default value:} {\tt ""} \\
{\bf Note:} This property is used only when Chord executes the given program, namely, when it is asked to compute the program scope dynamically (i.e., when \code{[chord.scope.kind]=dynamic}) or when it is asked to execute a dynamic program analysis.
\end{quote}

\code{chord.runtime.jvmargs}
\begin{quote}
{\bf Type:} string \\
{\bf Description:} Arguments to JVM which runs the given program. \\
{\bf Default value:} {\tt "-ea -Xmx1024m"} \\
{\bf Note:} This property is used only when Chord executes the given program, namely, when it is asked to compute the program scope dynamically (i.e., when \code{[chord.scope.kind]=dynamic}) or when it is asked to execute a dynamic program analysis. 
\end{quote}

\paragraph{Analysis Scope Properties}

This section describes properties that specify how the analysis scope of the given program is computed.
See Section \ref{sec:computing-scope} for more details.
\\[10pt]
\code{chord.scope.kind}
\begin{quote}
{\bf Type:} {\tt [dynamic|rta|cha]} \\
{\bf Description:} Algorithm to compute program scope.  The choices are {\tt dynamic} (dynamic analysis), {\tt rta} (Rapid Type Analysis), and {\tt cha} (Class Hierarchy Analysis). \\
{\bf Default value:} {\tt rta}
\end{quote}

\code{chord.reflect.kind}
\begin{quote}
{\bf Type:} {\tt [none|static|dynamic|static\_cast]} \\
{\bf Description:} TODO \\
{\bf Default value:} {\tt none}
\end{quote}

\code{chord.ch.kind}
\begin{quote}
{\bf Type:} {\tt [static|dynamic]} \\
{\bf Description:} Algorithm to build the class hierarchy.  If it is {\tt dynamic}, then the given program is executed and classes not loaded by the JVM while running the program are excluded while building the class hierarchy. \\
{\bf Default value:} {\tt static} \\
{\bf Note:} This property is relevant only if \code{chord.scope.kind} is {\tt cha} since only this scope computing algorithm queries the class hierarchy. 
\end{quote}

\code{chord.ssa}
\begin{quote}
{\bf Type:} bool  \\
{\bf Description:} Do SSA transformation for all methods deemed reachable by the algorithm used to compute program scope. \\
{\bf Default value:} {\tt true}
\end{quote}

% TODO: mention that below two properties are also used by instrumentor to decide which classes to exclude from instrumentation

\code{chord.std.scope.exclude}
\begin{quote}
{\bf Type:} string list \\
{\bf Description:} Partial list of prefixes of names of classes whose declared methods must be treated as no-ops.  Conventionally, this property is used to specify classes to be excluded from the scope of all programs to be analyzed, e.g., classes in the JDK standard library. \\
{\bf Default value:} {\tt ""}
\end{quote}

\code{chord.ext.scope.exclude}
\begin{quote}
{\bf Type:} string list \\
{\bf Description:} Partial list of prefixes of names of classes whose declared methods must be treated as no-ops.  Conventionally, this property is used to specify classes to be excluded from the scope of specific programs to be analyzed. \\
{\bf Default value:} {\tt ""}
\end{quote}

\code{chord.scope.exclude}
\begin{quote}
{\bf Type:} string list \\
{\bf Description:} Complete list of prefixes of names of classes whose declared methods must be treated as no-ops. \\
{\bf Default value:} \code{"[chord.std.scope.exclude],[chord.ext.scope.exclude]"}
\end{quote}

\code{chord.std.check.exclude}
\begin{quote}
{\bf Type:} string list \\
{\bf Description:} List of prefixes of names of classes and packages inside the standard library to be excluded by program analyses.  Interpretation of this property is analysis-specific. \\
{\bf Default value:} \code{"java.,javax.,sun.,com.sun.,com.ibm.,} \code{org.apache.harmony."}
\end{quote}

\code{chord.ext.check.exclude}
\begin{quote}
{\bf Type:} string list \\
{\bf Description:} List of prefixes of names of classes and packages outside the standard library to be excluded by program analyses.  Interpretation of this property is analysis-specific. \\
{\bf Default value:} {\tt ""}
\end{quote}

\code{chord.check.exclude}
\begin{quote}
{\bf Type:} string list \\
{\bf Description:} List of prefixes of names of classes and packages to be excluded by program analyses.  Interpretation of this property is analysis-specific. \\
{\bf Default value:} \code{"[chord.std.check.exclude],[chord.ext.check.exclude]"}
\end{quote}

\paragraph{Functionality Properties}

This section describes properties that dictate what Chord outputs during execution.

\code{chord.build.scope}
\begin{quote}
{\bf Type:} bool \\
{\bf Description:} Compute the scope of the given program using the algorithm specified by properties \code{chord.scope.kind} and \code{chord.reuse.scope}. \\
{\bf Default value:} {\tt false} \\
{\bf Note:} The program scope is computed regardless of the value of this property if another task (e.g., a program analysis specified via property \code{chord.run.analyses}) demands it by calling method \code{chord.program.Program.g()}. 
\end{quote}

\code{chord.run.analyses}
\begin{quote}
{\bf Type:} string list  \\
{\bf Description:} List of names of program analyses to be run in order. \\
{\bf Default value:} {\tt ""} \\
{\bf Note:} If the analysis is written in Java, its name is specified via statement {\tt name=<...>} in its {\tt @Chord} annotation, and is the name of the class itself if this statement is missing.  If the analysis is written in Datalog, its name is specified via a line of the form ``{\tt \# name=<...>}", and is the absolute location of the file itself if this line is missing. 
\end{quote}

\code{chord.print.methods}
\begin{quote}
{\bf Type:} string list  \\
{\bf Description:} List of methods in scope whose intermediate representation to print to standard output. \\
{\bf Default value:} {\tt ""} \\
{\bf Note:} Specify each method in format \code{<mname>:<mdesc>@<cname>} where {\tt <mname>} is the method's name, {\tt <mdesc>} is the method's descriptor, and {\tt <cname>} is the name of the method's declaring class. In {\tt <cname>}, use `.' instead of `/', and use {\tt \#} instead of the dollar character. 
\end{quote}

\code{chord.print.classes}
\begin{quote}
{\bf Type:} string list  \\
{\bf Description:} List of classes in scope whose intermediate representation to print to standard output. \\
{\bf Default value:} {\tt ""} \\
{\bf Note:} In class names, use `.' instead of `/', and use {\tt \#} instead of the dollar character. 
\end{quote}

\code{chord.print.all.classes}
\begin{quote}
{\bf Type:} bool \\
{\bf Description:} Print intermediate representation of all classes in scope to standard output. \\
{\bf Default value:} {\tt false}
\end{quote}

\code{chord.print.rels}
\begin{quote}
{\bf Type:} string list  \\
{\bf Description:} List of names of program relations whose contents must be printed to files \code{[chord.out.dir]/<...>.txt} where {\tt <...>} denotes the relation name. \\
{\bf Default value:} {\tt ""} \\
{\bf Note:} This functionality must be used with caution as certain program relations, albeit represented compactly as BDDs, may contain a large number (e.g., billions) of tuples, resulting in voluminous output when printed to a file.  See Section \ref{sec:tuning-datalog-analysis} for a more efficient way to query the contents of program relations (namely, by using the {\tt debug} target provided in file {\tt build.xml} in Chord's {\tt main/} directory). 
\end{quote}

\code{chord.print.project}
\begin{quote}
{\bf Type:} bool \\
{\bf Description:} Create files \code{targets_sortby_name.html}, \code{targets_sortby_kind.html}, and \code{targets_sortby_producers.html} in directory \code{[chord.out.dir]}, publishing all tasks and targets defined by program analyses in paths \code{[chord.java.analysis.path]} and \code{[chord.dlog.analysis.path]}.  \\
{\bf Default value:} {\tt false}
\end{quote}

\code{chord.print.results}
\begin{quote}
{\bf Type:} bool \\
{\bf Description:} Print the results of program analyses in HTML.  Interpretation of this property is analysis-specific.  \\
{\bf Default value:} {\tt true}
\end{quote}

\code{chord.verbose}
\begin{quote}
{\bf Type:} int in the range [0..6]  \\
{\bf Description:} Control the verbosity of logging during Chord's execution.  \\
{\bf Default value:} {\tt 2}
\end{quote}

\paragraph{Project Properties}

This section describes properties regarding program analyses executed by Chord.

\code{chord.classic}
\begin{quote}
{\bf Type:} bool \\
{\bf Description:} TODO \\
{\bf Default value:} TODO
\end{quote}

\code{chord.std.java.analysis.path}
\begin{quote}
{\bf Type:} path \\
{\bf Description:} TODO \\
{\bf Default value:} TODO
\end{quote}

\code{chord.ext.java.analysis.path}
\begin{quote}
{\bf Type:} path \\
{\bf Description:} TODO \\
{\bf Default value:} TODO
\end{quote}

\code{chord.java.analysis.path}
\begin{quote}
{\bf Type:} path \\
{\bf Description:} Classpath containing program analyses written in Java (i.e., {\tt @Chord}-annotated classes).  \\
{\bf Default value:} \code{[chord.main.dir]/classes/}
\end{quote}

\code{chord.std.dlog.analysis.path}
\begin{quote}
{\bf Type:} path \\
{\bf Description:} TODO \\
{\bf Default value:} TODO
\end{quote}

\code{chord.ext.dlog.analysis.path}
\begin{quote}
{\bf Type:} path \\
{\bf Description:} TODO \\
{\bf Default value:} TODO
\end{quote}

\code{chord.dlog.analysis.path}
\begin{quote}
{\bf Type:} path  \\
{\bf Description:} Path of directories containing program analyses written in Datalog (i.e., files with suffix {\tt .dlog} or {\tt .datalog}). \\
{\bf Default value:} \code{[chord.main.dir]/src/}
\end{quote}

\paragraph{Instrumentation Properties}

This section describes properties regarding execution of instrumented programs for dynamic analyses.

% TODO: mention chord.scope.exclude* here
\code{chord.use.jvmti}
\begin{quote}
{\bf Type:} bool \\
{\bf Description:} TODO \\
{\bf Default value:} TODO
\end{quote}

\code{chord.instr.kind}
\begin{quote}
{\bf Type:} {\tt [offline|online]}  \\
{\bf Description:} The kind of bytecode instrumentation.  The choices are offline and online (load-time).  \\
{\bf Default value:} {\tt offline}
\end{quote}

\code{chord.trace.kind}
\begin{quote}
{\bf Type:} {\tt [full|pipe]}  \\
{\bf Description:} The medium by which an event-generating JVM and an event-handling JVM communicate in a dynamic analysis.  The choices are regular file and POSIX pipe.  \\
{\bf Default value:} {\tt full} 
\end{quote}

\code{chord.trace.block.size}
\begin{quote}
{\bf Type:} int \\
{\bf Description:} Number of bytes to read/write in a single operation from/to the event file used by a multi-JVM dynamic analysis. \\
{\bf Default value:} {\tt 4096}
\end{quote}

\code{chord.dynamic.haltonerr}
\begin{quote}
{\bf Type:} bool \\
{\bf Description:} Whether or not to terminate Chord if the given program terminates abnormally during dynamic analysis. \\
{\bf Default value:} true
\end{quote}

\code{chord.dynamic.timeout}
\begin{quote}
{\bf Type:} int  \\
{\bf Description:} The amount of time, in milliseconds, after which to kill the process running the given program during dynamic analysis, or -1 if the process must never be killed. \\
{\bf Default value:} {\tt -1}
\end{quote}

\code{chord.max.cons.size}
\begin{quote}
{\bf Type:} int \\
{\bf Description:} Maximum number of bytes over which events generated during the execution of any constructor in the given program may span. \\
{\bf Default value:} {\tt 50000000} \\
{\bf Note:} This property is relevant only for dynamic analyses which want events of the form {\tt BEF\_NEW} $h$ $t$ $o$ to be generated (see Section \ref{sec:instr-events}).  The problem with generating such events at run-time is that the ID $o$ of the object freshly created by thread $t$ at object allocation site $h$ cannot be instrumented until the object is fully initialized (i.e., its constructor has finished executing).  Hence, Chord first generates a ``crude dynamic trace", which has events of the form {\tt BEF\_NEW} $h$ $t$ and {\tt AFT\_NEW} $h$ $t$ $o$ generated before and after the execution of the constructor, respectively.  A subsequent pass generates a ``final dynamic trace", which replaces each {\tt BEF\_NEW} $h$ $t$ event by {\tt BEF\_NEW} $h$ $t$ $o$.  For this purpose, however, Chord must buffer all events generated between the {\tt BEF\_NEW} and {\tt AFT\_NEW} events, and this property specifies the number of bytes over which these events may span.  If the actual number of bytes exceeds the value specified by this property (e.g., if the constructor throws an exception and the {\tt AFT\_NEW} event is not generated at all), then Chord simply generates event {\tt BEF\_NEW} $h$ $i$ $0$ (i.e., it treats the created object as having ID 0, which is the ID also used for {\tt null}). 
\end{quote}

\paragraph{Caching Properties}

TODO

\code{chord.reuse.scope}
\begin{quote}
{\bf Type:} bool \\
{\bf Description:} Compute program scope using the information in files specified by properties \code{chord.methods.file} and \code{chord.reflect.file}. \\
{\bf Default value:} {\tt false} \\
{\bf Note:} Property \code{chord.scope.kind} is ignored if this property is set to {\tt true} and the two files exist. 
\end{quote}

\code{chord.reuse.rels}
\begin{quote}
{\bf Type:} bool  \\
{\bf Description:} Construct program relations from BDDs stored on disk (from a previous run of Chord) whenever possible instead of re-computing them. \\
{\bf Default value:} {\tt false}
\end{quote}

\code{chord.reuse.traces}
\begin{quote}
{\bf Type:} bool \\
{\bf Description:} Reuse dynamic traces computed by a previous run of Chord if they exist.  Property \code{chord.trace.kind} must be set to {\tt full} if this property is set to {\tt true}. \\
{\bf Default value:} {\tt false}
\end{quote}

\paragraph{Chord JVM Properties}

TODO

\code{java.class.path}
\begin{quote}
{\bf Type:} path \\
{\bf Description:} TODO \\
{\bf Default value:} TODO
\end{quote}

\code{user.dir}
\begin{quote}
{\bf Type:} location
{\bf Description:} TODO \\
{\bf Default value:} TODO
\end{quote}

\code{chord.max.heap}
\begin{quote}
{\bf Type:} string \\
{\bf Description:} Maximum memory size of JVM running Chord. \\
{\bf Default value:} {\tt 1024m}
\end{quote}

\code{chord.max.stack}
\begin{quote}
{\bf Type:} string \\
{\bf Description:} Maximum thread stack size of JVM running Chord. \\
{\bf Default value:} {\tt 32m}
\end{quote}

\code{chord.jvmargs}
\begin{quote}
{\bf Type:} string \\
{\bf Description:} Arguments to JVM running Chord. \\
{\bf Default value:} \code{"-showversion} \code{-ea} \code{-Xmx[chord.max.heap]} \code{-Xss[chord.max.stack]"}
\end{quote}

\paragraph{BDD Properties}

TODO

\code{chord.use.buddy}
\begin{quote}
{\bf Type:} bool \\
{\bf Description:} TODO \\
{\bf Default value:} TODO
\end{quote}

\code{chord.bddbddb.max.heap}
\begin{quote}
{\bf Type:} string \\
{\bf Description:} Maximum memory size of JVM running Datalog solver bddbddb. \\
{\bf Default value:} {\tt 1024m} \\
{\bf Note:}  Each program analysis written in Datalog is run in a separate JVM.  This is primarily because there may be multiple invocations of bddbddb in a single run of Chord and it is difficult to reset the global state of bddbddb on each invocation. 
\end{quote}

\paragraph{Output Location Properties}

This section describes properties specifying the names of files and directories output by Chord.
Most users will not want to alter the default values of these properties.

\code{chord.out.file}
\begin{quote}
{\bf Type:} location \\
{\bf Description:} Absolute location of the file to which the standard output stream is redirected during Chord's execution. \\
{\bf Default value:} \code{null}
\end{quote}

\code{chord.err.file}
\begin{quote}
{\bf Type:} location  \\
{\bf Description:} Absolute location of the file to which the standard error stream is redirected during Chord's execution. \\
{\bf Default value:} \code{null}
\end{quote}

\code{chord.out.dir}
\begin{quote}
{\bf Type:} location \\
{\bf Description:} Absolute location of the directory to which Chord dumps all files. \\
{\bf Default value:} \code{[chord.work.dir]/chord_output/}
\end{quote}

\code{chord.reflect.file}
\begin{quote}
{\bf Type:} location  \\
{\bf Description:} Absolute location of the file from/to which resolved reflection information is read/written. \\
{\bf Default value:} \code{[chord.out.dir]/reflect.txt}
\end{quote}

\code{chord.methods.file}
\begin{quote}
{\bf Type:} location  \\
{\bf Description:} Absolute location of the file from/to which list of methods deemed reachable is read/written. \\
{\bf Default value:} \code{[chord.out.dir]/methods.txt}
\end{quote}

\code{chord.classes.file}
\begin{quote}
{\bf Type:} location  \\
{\bf Description:} Absolute location of the file from/to which list of classes deemed reachable is read/written.  \\
{\bf Default value:} \code{[chord.out.dir]/classes.txt}
\end{quote}

\code{chord.bddbddb.work.dir}
\begin{quote}
{\bf Type:} location \\
{\bf Description:} Absolute location of the directory used by the Datalog solver bddbddb as its input/output directory (namely, for program domain files {\tt *.dom} and {\tt *.map}, and program relation files {\tt *.bdd}). \\
{\bf Default value:} \code{[chord.out.dir]/bddbddb/}
\end{quote}

\code{chord.boot.classes.dir}
\begin{quote}
{\bf Type:} location \\
{\bf Description:} Absolute location of the directory from/to which instrumented JDK classes used by the given program are read/written by dynamic program analyses. \\
{\bf Default value:} \code{[chord.out.dir]/boot_classes/}
\end{quote}

\code{chord.user.classes.dir}
\begin{quote}
{\bf Type:} location  \\
{\bf Description:} Absolute location of the directory from/to which instrumented non-JDK classes of the given program are read/written by dynamic program analyses. \\
{\bf Default value:} \code{[chord.out.dir]/user_classes/}
\end{quote}

\code{chord.instr.scheme.file}
\begin{quote}
{\bf Type:} location \\
{\bf Description:} Absolute location of the file specifying the kind and format of events in dynamic trace files. \\
{\bf Default value:} \code{[chord.out.dir]/scheme.ser}
\end{quote}

\code{chord.trace.file}
\begin{quote}
{\bf Type:} location  \\
{\bf Description:} Absolute location of the dynamic trace file.  \\
{\bf Default value:} \code{[chord.out.dir]/trace} \\
{\bf Note:} Suffix \code{_full_verN.txt} or \code{_pipe_verN.txt} is appended to the name of the file, depending upon whether it is a regular file or a POSIX pipe, respectively, where {\tt N} is the version of the file (multiple versions are maintained if the trace is transformed by filters defined by the dynamic analysis; 0 is the final version).  If \code{chord.reuse.traces} is set to {\tt true}, then \code{_full_verN_runM.txt} is appended to the name of the file, where {\tt M} is the run ID. 
\end{quote}

%\item
%\code{chord.main.dir}
% location
%Absolute location of the {\tt main/} directory in Chord's installation.

%\item
%\code{chord.save.maps}
% bool
%Write to file \code{[chord.bddbddb.work.dir]/<...>.map} when saving program domain named {\tt <...>}.
%{\tt true}
%{\bf Note:} This functionality is useful for debugging Datalog programs using the {\tt debug} target provided in file {\tt build.xml} in Chord's {\tt main/} directory (see Section \ref{sec:tuning-datalog-analysis}).

