\xname{running}
\chapter{Running an Analysis}
\label{chap:running}

Chord provides many standard analyses.  Moreover, it allows users to define
their own analyses, possibly atop the predefined analyses. This chapter describes
how to run predefined as well as user-defined analyses.

Broadly, there are two kinds of analyses in Chord:

\begin{mytable}{|l||c|c|}
\hline
Kind:
	& imperative (see Chapter \ref{chap:writing})
	& declarative (see Chapter \ref{chap:datalog}) \\
\hline
Location:
	& \begin{tabular}{c}
	  a {\tt .class} file in the path denoted by property \\
      \code{chord.java.analysis.path} compiled \\
      from a \code{@Chord}-annotated class implementing \\
      interface \javadoc{chord.project.ITask}{chord/project/ITask.html}
      \end{tabular}
	& \begin{tabular}{c}
      a {\tt .dlog} file in the path denoted by property \\
      \code{chord.dlog.analysis.path}
      \end{tabular} \\
\hline
Name:
	& via stmt \verb+name="<NAME>"+ in {\tt @Chord} annotation
	& via line ``\verb+# name=<NAME>+'' in {\tt .dlog} file \T \\
\hline
\end{mytable}

In its most general form, the command for running an analysis is as follows:

\begin{framed}
\begin{verbatim}
ant -Dchord.work.dir=<WORK_DIR> -Dchord.run.analyses=<ANALYSIS_NAME> \
    -Dchord.dlog.analysis.path=<DLOG_ANALYSIS_PATH> \
    -Dchord.java.analysis.path=<JAVA_ANALYSIS_PATH> run
\end{verbatim}
\end{framed}

where:
\begin{itemize}
\item
\code{<WORK_DIR>} is a directory containing a file named
\code{chord.properties} that defines various properties of the program to be
analyzed that might be needed by the analysis being run, such as the program's
main class, the program's application classpath, etc.
(see Chapter \ref{chap:setup}).
\item
\code{<ANALYSIS_NAME>} is the name of the analysis to run.  More generally, it
can be a comma-separated list of names of analyses to run in order.
\item
\code{<JAVA_ANALYSIS_PATH>} is a path specifying all imperative analyses that
might be needed to run the desired analysis.
\item
\code{<DLOG_ANALYSIS_PATH>} is a path specifying all declarative analyses that
might be needed to run the desired analysis.
\end{itemize}

In order to facilitate heavy reuse and rapid prototyping, each analysis in
Chord is written modularly, independent of other analyses, along with
lightweight annotations specifying the inputs and outputs of the analysis.
Chord's runtime automatically computes producer-consumer relationships between
analyses (e.g., determines which analysis produces as output a result that is
needed as input by another analysis).  Before running the desired analysis
named \code{<ANALYSIS_NAME>}, Chord recursively runs other analyses until the
inputs to the desired analysis have been computed; it finally runs the desired
analysis to produce the outputs of that analysis.  Chapter \ref{chap:project}
explains this process in detail.  Each of these analyses must occur in the
path denoted by \code{<JAVA_ANALYSIS_PATH>} or \code{<DLOG_ANALYSIS_PATH>},
depending upon whether the analysis is written imperatively or declaratively,
respectively.

Chord provides shorthand ways for specifying analysis paths by means of the
following six properties:

\begin{mytable}{|c|l|l|l|}
\hline
Analysis Kind
	& \multicolumn{1}{c|}{Predefined}
	& \multicolumn{1}{c|}{User-defined}
	& \multicolumn{1}{c|}{All} \\
\hline
imperative
	& \code{chord.std.java.analysis.path}
	& \code{chord.ext.java.analysis.path}
	& \code{chord.java.analysis.path} \\
\hline
declarative
	& \code{chord.std.dlog.analysis.path}
	& \code{chord.ext.dlog.analysis.path}
	& \code{chord.dlog.analysis.path}
\T \\
\hline
\end{mytable}

The paths specified by the \code{chord.std.*.analysis.path} properties
conventionally include all ``standard'' analyses: analyses that are predefined
in Chord.  The default value of each of these properties is the absolute path
of file \code{chord.jar}.  Normally, users must not change the value of these
properties.

The paths specified by the \code{chord.ext.*.analysis.path} properties
conventionally include all ``external'' analyses: analyses that are written by
users.  The default value of each of these properties is the empty path.

The paths specified by the \code{chord.*.analysis.path} properties include
{\it all} analyses: both standard and external ones.  The default value of each
of these properties is simply the concatenation of the values of the
corresponding two properties above that specify the paths of standard and
external analyses.  Normally, users must not change the value of these
properties.

Running the above command can cause Chord to report a runtime error in the
following scenarios:

\begin{itemize}
\item
Either no included analysis or multiple included analyses are named
\code{<ANALYSIS_NAME>}.
\item
A result declared as consumed by the analysis named \code{<ANALYSIS_NAME>}
(or by some analysis on which the specified analysis is dependent directly or
transitively) is either not declared as produced by any included analysis or
is declared as produced by multiple included analyses.
\end{itemize}

To fix the error resulting from the ``missing analysis'' case in both the above
scenarios, simply include the missing analysis in the path specified by
properties \code{chord.*.analysis.path}.

To fix the error resulting from the ``ambiguous analysis'' case in both the
above scenarios, the names {\tt A1}, ..., {\tt An} of all analyses that were
involved in the ambiguity are provided in the error report.  Suppose {\tt Ai}
is the desired analysis from this set.  Then, one way to fix the error is to
exclude all analyses in that set except analysis {\tt Ai} from the path
specified by properties \code{chord.*.analysis.path}.  A better way is to
specify the names of {\it multiple} analyses in the value of property
\code{chord.run.analyses} (recall that this property allows a comma-separated
list of names of analyses to be run in order).  Specifically, the value of
this property must specify the name of the desired analysis {\tt Ai}
{\it before} the name of the analysis that caused the ambiguity error.

The above command is for users who have defined their own analyses and wish
to run them.  The following simpler command suffices for users who only want
to run analyses predefined in Chord (it uses the default values for
properties \code{chord.*.analysis.path}):

\begin{framed}
\begin{verbatim}
ant -Dchord.work.dir=<WORK_DIR> -Dchord.run.analyses=<ANALYSIS_NAME> run
\end{verbatim}
\end{framed}

For instance, to run a basic may-alias and call-graph analysis (called 0CFA),
run the following command:

\begin{framed}
\begin{verbatim}
ant -Dchord.work.dir=<WORK_DIR> -Dchord.run.analyses=cipa-0cfa-dlog run
\end{verbatim}
\end{framed}

This instructs Chord to run the declarative analysis named \code{cipa-0cfa-dlog},
which is defined in file \code{main/src/chord/analyses/alias/cipa_0cfa.dlog}.

The output of the above command is of the form:

\begin{framed}
{\small
\begin{verbatim}
Buildfile: build.xml

run:
     [java] Chord run initiated at: Mar 13, 2011 10:31:08 PM
     [java] ENTER: cipa-0cfa-dlog
     [java] ENTER: T
     [java] ENTER: RTA
     [java] Iteration: 0
     [java] Iteration: 1
     [java] Iteration: 2
     [java] LEAVE: RTA
     [java] SAVING dom T size: 1386
     [java] LEAVE: T
     [java] ENTER: F
     [java] SAVING dom F size: 4120
     [java] LEAVE: F
     ...
     [java] ENTER: MputStatFldInst
     [java] SAVING rel MputStatFldInst size: 739
     [java] LEAVE: MputStatFldInst
     [java] ENTER: statIM
     [java] SAVING rel statIM size: 3359
     [java] LEAVE: statIM
     [java] Starting command: 'java ... chord_analyses_alias_cipa_0cfa.dlog'
     [java] Relation VH: 541 nodes, 449.0 elements (V0,H0)
     [java] Relation FH: 137 nodes, 8.0 elements (H0,F0)
     [java] Relation HFH: 199 nodes, 35.0 elements (H0,F0,H1)
     [java] Relation IM: 590 nodes, 69.0 elements (I0,M0)
     [java] Finished command: 'java ... chord_analyses_alias_cipa_0cfa.dlog'
     [java] LEAVE: cipa-0cfa-dlog
     [java] Chord run completed at: Mar 13, 2011 10:31:36 PM
     [java] Total time: 00:00:27:671 hh:mm:ss:ms

BUILD SUCCESSFUL
Total time: 28 seconds
\end{verbatim}
}
\end{framed}

The 0CFA analysis consumes and produces multiple {\it program relations}.
The consumed program relations include \code{MputStatFldInst} and
\code{statIM}, each of which is produced by a separate imperative analysis
with the corresponding name, and the produced program relations include
\code{VH}, \code{FH}, \code{HFH}, and \code{IM}.  We next briefly discuss
these relations.

The program relations consumed by the 0CFA analysis contain basic program
facts.  For instance, \code{MputStatFldInst} is a relation containing
each tuple ($m$,$f$,$v$) such that method $m$ in the input Java program
contains a putstatic instruction of the form ``$f$ = $v$", while
\code{statIM} is a relation containing each tuple ($i$,$m$) such that $m$
is the target method of invokestatic instruction $i$.

The program relations produced by the 0CFA analysis represent points-to
information and the call graph of the input Java program as computed by
the analysis.  Specifically, relations \code{VH}, \code{FH}, and \code{HFH}
represent points-to information for local variables, static fields, and
instance fields, respectively, while relation \code{IM} represents the call
graph, namely, it contain each tuple ($i$,$m$) such that $m$ is a possible
target method of call site $i$.

Metavariables $m$, $f$, $i$, and $v$ above range over entities in so-called
{\it program domains} \code{M}, \code{F}, \code{I}, and \code{V}, respectively.
A program domain is a set of entities of a certain kind in the input Java
program.  For instance, \code{M} is the domain representing the set of all
methods in the input Java program, \code{F} is the domain representing the set
of all fields, \code{I} is the domain representing the set of all call sites
in methods in \code{M}, and \code{V} is the domain representing the set of all
local variables of reference type in methods in \code{M}.  Each of these
domains is produced by a separate Java analysis with the corresponding name.
Notice that the analyses producing these domains run upfront because these
domains are consumed by the analyses that produce relations such as
\code{MputStatFldInst} and \code{statIM}, which in turn are consumed by the
desired 0CFA analysis.

During execution, Chord dumps intermediate and final results to files in the
directory specified by property \code{chord.out.dir}, whose default value is
\code{[chord.work.dir]/chord_output/} and typically does not need to be changed
by users.  For the above example, this directory is
\code{<WORK_DIR>/chord_output/}.

The verbosity of Chord's output above is controlled by property
\code{chord.verbose}, whose default value is 1.  At verbosity level 0, the
above command produces less voluminous output of the form:

\begin{framed}
{\small
\begin{verbatim}
Buildfile: build.xml

run:
     [java] Chord run initiated at: Mar 13, 2011 10:35:01 PM
     [java] Chord run completed at: Mar 13, 2011 10:35:28 PM
     [java] Total time: 00:00:26:297 hh:mm:ss:ms

BUILD SUCCESSFUL
Total time: 26 seconds
\end{verbatim}
}
\end{framed}

