\section{Setting up a Java Program for Analysis}
\label{sec:program-setup}

Suppose the program to be analyzed has the following directory structure:

\begin{quote}
\begin{verbatim}
example/
    src/
        foo/
            Main.java
            ...
    classes/
        foo/
            Main.class
            ...
    lib/
        src/
            taz/
                ...
        jar/
            taz.jar
    chord.properties
\end{verbatim}
\end{quote}

The above structure is typical: the program's Java source files are under \verb+src/+,
its class files are under \verb+classes/+, the source and jar files of the libraries
used by the program are under \verb+lib/src/+ and \verb+lib/jar/+, respectively.

File \verb+chord.properties+ specifies properties to be passed to Chord (alternatively,
these properties may be passed on the command-line, in format \verb+-D<name>=<value>+).
A sample such file for the above program is as follows:

\begin{quote}
\begin{verbatim}
chord.main.class=foo.Main
chord.class.path=classes:lib/jar/taz.jar
chord.src.path=src:../lib/src
chord.run.ids=0,1
chord.args.0="-thread 1 -n 10"
chord.args.1="-thread 2 -n 50"
\end{verbatim}
\end{quote}

Each relative (instead of absolute) path element in the value of any property
named \verb+chord.<...>.path+ defined in this file is converted to an absolute path element with respect
to the directory containing this file, which in the above case is \verb+example/+.

Section \ref{sec:chord-sysprops} presents an exhaustive
listing of properties recognized by \Chord.
Here, we only describe those defined in the above
sample \verb+chord.properties+ file:

\begin{itemize}
\item
\verb+chord.main.class+ specifies the fully-qualified name of the main class of the
program.
\item
\verb+chord.class.path+ specifies the classpath of the program.
The value of pre-defined property \verb+${sun.boot.class.path}+ is implicitly appended
to it.
\item
\verb+chord.src.path+ specifies the Java source path of the program.
All program analyses in Chord operate on Java bytecode (specified by \verb+chord.class.path+).
The only use of this property is to HTMLize the Java source files of the program so that the
results of program analyses can be reported at the Java source code level.
\item
\verb+chord.run.ids+ specifies a list of IDs to identify runs of the program.
It is used by dynamic program analyses to determine how many times the program must be run.
An additional use of this property is to allow specifying the command-line arguments to use
in the run having ID \verb+XXX+ via property \verb+chord.args.XXX+, as illustrated by
properties \verb+chord.args.0+ and \verb+chord.args.1+ in the above example.
\end{itemize}

To run Chord on the above program, run the following command in Chord's \verb+main/+ directory:

\begin{quote}
\begin{verbatim}
prompt> ant -Dchord.work.dir=<...>/example run
\end{verbatim}
\end{quote}

where \verb+<...>+ denotes either the absolute path, or the path relative to Chord's main directory,
of the parent of directory {\tt example/}.
System property \verb+chord.work.dir+ specifies the working directory during Chord's execution.

The above command causes Chord to load the \verb+chord.properties+ file if present.
The location of this file may be specified explicitly via property
\verb+chord.props.file+ on the above command line.
By default, it is \verb+${chord.work.dir}/chord.properties+.

In the above case, the above command does not do much beyond loading the sample \verb+chord.properties+ file.
For Chord to do something interesting, you need to set additional properties, either
on the above command line or in the \verb+chord.properties+ file, specifying the task(s)
Chord must perform.
All tasks are summarized in the ``Chord task properties" portion of the table in Section \ref{sec:chord-sysprops}.
The two most common tasks, described next, are building program scope (Section \ref{sec:building-scope})
and running program analyses (Section \ref{sec:running-analysis}).

