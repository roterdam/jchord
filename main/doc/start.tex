\xname{start}
\chapter{Getting Started}
\label{chap:start}

This chapter describes how to download, install, and run Chord.  Section
\ref{sec:downloading-binaries} describes how to obtain pre-built binaries of
Chord.  Section \ref{sec:downloading-sources} describes how to obtain the source
code of Chord and Section \ref{sec:downloading-sources} explains how to build
it.  Finally, Section \ref{sec:running-chord} describes how to run Chord.

\section{Downloading Binaries}
\label{sec:downloading-binaries}

To obtain Chord's pre-built binaries, download and uncompress file
\chordbinfile.  It includes the following files:

\begin{enumerate}
\item
\code{chord.jar}, which contains the class files of Chord and of libraries used
by Chord.
\item
\code{libbuddy.so}, \code{buddy.dll}, and \code{libbuddy.dylib}: you can keep
one of these files depending upon whether you intend to run Chord on Linux,
Windows/Cygwin, or MacOS, respectively.  These files are needed only if you want
BDD library \buddy\ to be used when the BDD-based Datalog solver \bddbddb\ in
Chord runs analyses written in Datalog.
\item
\code{libchord_instr_agent.so}: this file is needed only if you want the
JVMTI-based bytecode instrumentation agent to be used when Chord runs dynamic
analyses.
\end{enumerate}

Novice users can ignore items (2) and (3) until they become more familiar with
Chord.  The binaries mentioned in items (2) and (3) might not be compatible with
your machine, in which case you can either forgo using them (with hardly any
noticeable difference in functionality), or you can download the sources (see
Section \ref{sec:downloading-sources}) and build them yourself (see Section
\ref{sec:compiling-sources}).

\section{Downloading Source Code}
\label{sec:downloading-sources}

To obtain Chord's source code, download and uncompress the following files:

\begin{itemize}
\item
Mandatory: file
\chordsrcfile, which contains Chord's source code and jars of libraries used by
Chord.
\item
Optional: file \chordlibfile, which contains the source code of libraries used
by Chord (e.g., joeq, javassist, bddbddb, etc.)
\end{itemize}

Alternatively, you can obtain the latest development snapshot from the SVN
repository by running the following command:

\begin{framed}
\begin{verbatim}
svn checkout http://jchord.googlecode.com/svn/trunk/ chord
\end{verbatim}
\end{framed}

Instead of checking out the entire \code{trunk/}, which contains several
sub-directories, you can check out specific sub-directories:

\begin{itemize}
\item
\code{main/} contains Chord's source code and jars of libraries used by Chord.
\item
\code{libsrc/} contains the source code of libraries used by Chord (e.g., joeq,
javassist, bddbddb, etc.).
\item
\code{test/} contains Chord's regression tests.
\item
many more; these might eventually move under \code{main/}.
\end{itemize}

Files \code{chord-2.0-src.tar.gz} and \code{chord-2.0-libsrc.tar.gz} mentioned
above are essentially stable releases of the \code{main/} and \code{libsrc/}
directories, respectively.

\section{Compiling the Source Code}
\label{sec:compiling-sources}

Compiling Chord's source code requires the following software:

\begin{itemize}
\item
JVM supporting Java 5 or higher, e.g. \ibmjvm\ or \sunjvm.
\item
\ant, a Java build tool.
\end{itemize}

Chord's main directory contains a file named {\tt build.xml} which is
interpreted by Apache Ant.  To see the various possible targets, simply run
command ``{\tt ant}" in that directory.

To compile Chord, run command ``{\tt ant compile}" in the same directory.  This
will compile Chord's Java sources from \code{src/} to class files in
\code{classes/}, as well as build a jar file \code{chord.jar} that contains
these class files as well as those in the jars of libraries that are used by
Chord and are provided under \code{lib/} (e.g., \code{joeq.jar},
\code{javassist.jar}, \code{bddbddb.jar}, etc.).  Additionally:

\begin{itemize}
\item

If system property \code{chord.use.buddy} is set to \code{true}, then the C
source code of BDD library \buddy\ from directory \code{bdd/} will be compiled
to a shared library named (\code{libbuddy.so} on Linux, \code{buddy.dll} on
Windows, and \code{libbuddy.dylib} on MacOS; this library is used by BDD-based
Datalog solver \bddbddb\ in Chord for running analyses written in Datalog.

\item

If system property \code{chord.use.jvmti} is set to \code{true}, then the C++
source code of the JVMTI-based bytecode instrumentation agent from directory
\code{agent/} will be compiled to a shared library named
\code{libchord_instr_agent.so} on all architectures; this agent is used in Chord
for computing analysis scope dynamically and for running dynamic analyses.
\end{itemize}

Properties \code{chord.use.buddy} and \code{chord.use.jvmti} are defined in a
file named \code{chord.properties} in Chord's main directory.  The default value
of both these properties is \code{false}.  If you set either of them to
\code{true}, then you will also need a utility like GNU Make (to run the
\code{Makefile}'s in directories \code{bdd/} and \code{agent/}) and a C++
compiler (to build the above shared libraries).

\section{Running Chord}
\label{sec:running-chord}

Running Chord requires a JVM supporting Java 5 or higher.  There are two
equivalent commands to run Chord.

One command, which is available in the source and binary installations of
Chord, is:

\begin{framed}
\begin{verbatim}
java -cp <CHORD_MAIN_DIR>/chord.jar -D<key1>=<val1> ... -D<keyN>=<valN> chord.project.Boot
\end{verbatim}
\end{framed}

where \code{<CHORD_MAIN_DIR>} denotes the directory containing file
\code{chord.jar}; that directory is also expected to contain any shared
libraries in Chord's installation (e.g., \code{libbuddy.so} and
\code{libchord_instr_agent.so}).

The alternate command, which is available only in the source installation of
Chord, is:

\begin{framed}
\begin{verbatim}
ant -f <CHORD_MAIN_DIR>/build.xml -D<key1>=<val1> ... -D<keyN>=<valN> run
\end{verbatim}
\end{framed}

This command requires \ant\ (a Java build tool) to be installed on your
machine.  This command is used throughout this guide.  Also, the
``\code{-f <CHORD_MAIN_DIR>/build.xml}" argument in the command is omitted
for brevity.

Each ``\code{-D<key>=<val>}" argument in either of the above commands sets the
system property named \code{<key>} to the value denoted by \code{<val>}.  The
only way to specify inputs to Chord is via system properties; there is no
command-line argument processing.  Chapter \ref{chap:properties} describes all
system properties recognized by Chord.

