\xname{program}
\chapter{Java Program Representation}
\label{chap:program}

Chord uses \xlink{Joeq}{http://joeq.sourceforge.net/} to translate
Java bytecode, one class file at a time, into a three-address-like
intermediate representation of the input Java program called {\it
quadcode}.  This chapter describes all aspects of quadcode and how it
relates to bytecode.  It first explains how to pretty-print bytecode
and quadcode (Section \ref{sec:pretty-printing}) which is useful for
debugging analyses and deciphering their output.  The remaining
sections describe the quadcode representation along with the API of
Joeq and Chord for navigating it.  Briefly, the representation
consists of a set of classes that may be loaded
(Section \ref{sec:program}).  The representation of each class consists
of a set of members (Section \ref{sec:class-members}) which are the fields and methods
of the class.  The representation of a concrete
method (Section \ref{sec:methods}) consists of a control-flow graph
(CFG).  The representation of a CFG (Section \ref{sec:cfgs}) consists
of a set of registers and a set of basic blocks linked by directed edges
denoting flow of control between basic blocks.
Each basic block contains zero or more primitive statements called quads
(Section \ref{sec:quads}).  Finally, the most common way to traverse
all quads is discussed (Section \ref{sec:traversing}).

\section{Pretty-Printing}
\label{sec:pretty-printing}

Consider the following Java program contained in
file \code{examples/hello_world/src/test/HelloWorld.java} in Chord's
main directory:
\begin{framed}
\begin{verbatim}
package test;

public class HelloWorld {
    public static void main(String[] args) {
        System.out.println("Hello World!");
    }
}
\end{verbatim}
\end{framed}

First compile this program by running command \code{ant} in
directory \code{examples/hello_world/}.

To pretty-print the bytecode representation of a class, run the following command:
\begin{framed}
\begin{verbatim}
javap -classpath <CLASS_PATH> -bootclasspath <BOOT_CLASS_PATH> -private -verbose <CLASS_NAME>
\end{verbatim}
\end{framed}

where:

\begin{itemize}
\item
\verb+<CLASS_NAME>+ is the fully-qualified name of the class whose
bytecode is to be printed (\code{test.HelloWorld} in our example).
\item
\verb+<CLASS_PATH>+ is the classpath of that class (\code{examples/hello_world/} in our example).
\item
\verb+<BOOT_CLASS_PATH>+ is the boot classpath; it is optional and must be supplied if
\verb+<CLASS_NAME>+ is a class from the JDK standard library (e.g., \code{java.util.ArrayList}) that has
been modified and written to a user-defined location (e.g., it has been instrumented by Chord and
written to \code{chord_output/boot_classes/}).
\end{itemize}

Program \code{javap} comes along with the JVM.  The output of the
above command for our example is as follows:

\begin{framed}
\begin{verbatim}
Compiled from "HelloWorld.java"
public class test.HelloWorld extends java.lang.Object
  SourceFile: "HelloWorld.java"
  minor version: 0
  major version: 49
  Constant pool:
const #1 = Method   #6.#20; //  java/lang/Object."<init>":()V
const #2 = Field    #21.#22;    //  java/lang/System.out:Ljava/io/PrintStream;
const #3 = String   #23;    //  Hello World!
const #4 = Method   #24.#25;    //  java/io/PrintStream.println:(Ljava/lang/String;)V
const #5 = class    #26;    //  test/HelloWorld
const #6 = class    #27;    //  java/lang/Object
const #7 = Asciz    <init>;
const #8 = Asciz    ()V;
const #9 = Asciz    Code;
const #10 = Asciz   LineNumberTable;
const #11 = Asciz   LocalVariableTable;
const #12 = Asciz   this;
const #13 = Asciz   Ltest/HelloWorld;;
const #14 = Asciz   main;
const #15 = Asciz   ([Ljava/lang/String;)V;
const #16 = Asciz   args;
const #17 = Asciz   [Ljava/lang/String;;
const #18 = Asciz   SourceFile;
const #19 = Asciz   HelloWorld.java;
const #20 = NameAndType #7:#8;//  "<init>":()V
const #21 = class   #28;    //  java/lang/System
const #22 = NameAndType #29:#30;//  out:Ljava/io/PrintStream;
const #23 = Asciz   Hello World!;
const #24 = class   #31;    //  java/io/PrintStream
const #25 = NameAndType #32:#33;//  println:(Ljava/lang/String;)V
const #26 = Asciz   test/HelloWorld;
const #27 = Asciz   java/lang/Object;
const #28 = Asciz   java/lang/System;
const #29 = Asciz   out;
const #30 = Asciz   Ljava/io/PrintStream;;
const #31 = Asciz   java/io/PrintStream;
const #32 = Asciz   println;
const #33 = Asciz   (Ljava/lang/String;)V;

{
public test.HelloWorld();
  Code:
   Stack=1, Locals=1, Args_size=1
   0:   aload_0
   1:   invokespecial   #1; //Method java/lang/Object."<init>":()V
   4:   return
  LineNumberTable:
   line 3: 0
  LocalVariableTable:
   Start  Length  Slot  Name   Signature
   0      5      0    this       Ltest/HelloWorld;

public static void main(java.lang.String[]);
  Code:
   Stack=2, Locals=1, Args_size=1
   0:   getstatic   #2; //Field java/lang/System.out:Ljava/io/PrintStream;
   3:   ldc #3; //String Hello World!
   5:   invokevirtual   #4; //Method java/io/PrintStream.println:(Ljava/lang/String;)V
   8:   return
  LineNumberTable:
   line 5: 0
   line 6: 8
  LocalVariableTable:
   Start  Length  Slot  Name   Signature
   0      9      0    args       [Ljava/lang/String;
}
\end{verbatim}
\end{framed}

To pretty-print the quadcode representation of a class, run the following command:

\begin{framed}
\begin{verbatim}
ant -Dchord.work.dir=<WORK_DIR> -Dchord.print.classes=<CLASS_NAME> \
    -Dchord.verbose=0 -Dchord.out.file=<OUT_FILE> run
\end{verbatim}
\end{framed}

where:

\begin{itemize}
\item
\code{<WORK_DIR>} is a directory (\code{examples/hello_world/} in our example) 
that contains a file named \code{chord.properties} which defines
properties \code{chord.main.class} and \code{chord.class.path}
specifying the main class and the classpath of the input Java program.
Alternatively, these two properties can be defined directly on the
above command-line.
\item
\code{<CLASS_NAME>} is the fully-qualified name of the class whose quadcode is to be printed
({\tt test.HelloWorld} in our example).  Each occurrence of a `{\tt \$}' in the class name
must be replaced by a `{\tt \#}'.
\item
\code{<OUT_FILE>} is the file to which the quadcode must be written; if left unspecified,
the quadcode is written to the standard output.
\end{itemize}

The output of the above command for our example is as follows:

\begin{framed}
\begin{verbatim}
*** Class: test.HelloWorld
Method: main:([Ljava/lang/String;)V@test.HelloWorld
    0#1
    5#3
    5#2
    8#4
Control flow graph for main:([Ljava/lang/String;)V@test.HelloWorld:
BB0 (ENTRY) (in: <none>, out: BB2)

BB2 (in: BB0 (ENTRY), out: BB1 (EXIT))
1: GETSTATIC_A T1, .out
3: MOVE_A T2, AConst: "Hello World!"
2: INVOKEVIRTUAL_V println:(Ljava/lang/String;)V@java.io.PrintStream, (T1, T2)
4: RETURN_V

BB1 (EXIT)  (in: BB2, out: <none>)

Exception handlers: []
Register factory: Registers: 3
\end{verbatim}
\end{framed}

\section{Whole Program}
\label{sec:program}

This and the following sections describe the quadcode representation
along with the API of Joeq and Chord for navigating it.
This API is contained in
packages \javadoc{chord.program}{chord/program/package-summary.html}, \javadoc{joeq.Class}{joeq/Class/package-summary.html},
and \javadoc{joeq.Compiler.Quad}{joeq/Compiler/Quad/package-summary.html}.

The quadcode representation of the whole program is a unique global
object of
class \javadoc{chord.program.Program}{chord/program/Program.html}
which can be obtained by calling static
method \javadoc{chord.program.Program.g()}{chord/program/Program.html\#g()}.
This class provides a rich API (in the form of public instance
methods) to access various parts of the representation, most notably:

\begin{mytable}{|l|l|}
\hline
\verb+IndexSet<jq_Type> getTypes()+ & All types referenced in classes that may be loaded. \\
\hline
\verb+IndexSet<jq_Reference> getClasses()+ & All classes that may be loaded.* \\
\hline
\verb+IndexSet<jq_Method> getMethods()+ & All methods that may be called. \T \\
\hline
\end{mytable}

* Includes both classes/interfaces and array types, represented as objects
of \code{jq_Class} and \code{jq_Array}, respectively; both these are
subclasses of \code{jq_Reference}.

See Chapter \ref{chap:scope} for how Chord determines which
classes may be loaded and which methods may be called.

The quadcode representation of each type is a unique object of the
appropriate subclass
of \javadoc{joeq.Class.jq_Type}{joeq/Class/jq_Type.html} in the following
hierarchy:

\begin{verbatim}
                    jq_Type
                       |
      ------------------------------------
      |                                  |
jq_Primitive                       jq_Reference
                                         |
                          -------------------------------
                          |                             |
                      jq_Class                       jq_Array
\end{verbatim}

\section{Class Members}
\label{sec:class-members}

Each primitive type (e.g., boolean, int, etc.) is represented by a
unique \code{jq_Primtive} object.  Each class and each interface type is
represented by a unique \code{jq_Class} object.  Each array type is
represented by a unique \code{jq_Array} object.

Members (i.e., fields and methods) of the class/interface represented
by an object of
class \javadoc{joeq.Class.jq_Class}{joeq/Class/jq_Class.html} can be
accessed using the following API provided by that class.

\begin{mytable}{|l|p{3.1in}|}
\hline
\verb+String getName()+ & Fully-qualified name of the class, e.g., ``\code{java.lang.String[]}''. \\
\hline
\verb+jq_InstanceField[] getDeclaredInstanceFields()+ & All instance fields declared in the class. \\
\hline
\verb+jq_StaticField[] getDeclaredStaticFields()+ & All static fields declared in the class. \\
\hline
\verb+jq_InstanceMethod[] getDeclaredInstanceMethods()+ & All instance methods declared in the class. \\
\hline
\verb+jq_StaticMethod[] getDeclaredStaticMethods()+ & All static methods declared in the class. \T \\
\hline
\end{mytable}

Chord uses format \code{mName:mDesc@cName}, described in
class \javadoc{chord.program.MethodSign}{chord/program/MethodSign.html},
to uniquely identify each field and each method in the input Java
program, where
\code{mName} denotes the name of the field/method,
\code{mDesc} denotes the descriptor of the field/method (see below),
and \code{cName} denotes the fully-qualified name of the class
declaring the field/method.
For instance, ``\code{main:[Ljava/lang/String;@test.HelloWorld}''
uniquely identifies the main method in the example above.
We next review field descriptors and method descriptors from the Java
bytecode specification.

A field descriptor represents the type of a local variable or a
(static or instance) field.  It is a series of characters generated by
the grammar:

\begin{verbatim}
FieldDescriptor : FieldType
      FieldType : BaseType | ObjectType | ArrayType
       BaseType : B | C | D | F | I | J | S | Z
     ObjectType : L <classname> ;
      ArrayType : [ ComponentType
  ComponentType : FieldType
\end{verbatim}

The characters of \code{BaseType}, the `{\tt L}' and `{\tt ;}'
of \code{ObjectType}, and the `{\tt [}' of
\code{ArrayType} are all ASCII characters.
The {\tt <classname>} represents a fully qualified class or interface
name.  The interpretation of the field types is as shown in the below
table:

\begin{mytable}{l|l|l}
\hline
BaseType Character	& Type	& Interpretation \\
\hline
{\tt B} &byte& signed byte \\
{\tt C} &char& Unicode character \\
{\tt D} &double& double-precision floating-point value \\
{\tt F} &float& single-precision floating-point value \\
{\tt I} &int& integer \\
{\tt J} &long& long integer \\
{\tt L<classname>;}	&reference& an instance of class {\tt <classname>} \\
{\tt S}	&short& signed short \\
{\tt Z}	&boolean& true or false \\
{\tt [}	&reference& one array dimension \T \\
\hline
\end{mytable}

For example, the descriptor of type {\tt int}
 is simply {\tt I}.  The descriptor of an instance variable of
 type \code{Object} is ``\code{Ljava/lang/Object;}''.  Note that the
 internal form of the fully qualified name for class \code{Object} is
 used. The descriptor of a multidimensional double array of type
 ``\code{double[][][]}'' is ``\code{[[[D}''.

A method descriptor represents the types of the arguments and 
return result of a method:

\begin{verbatim}
   MethodDescriptor : ( ParameterDescriptor* ) ReturnDescriptor
ParameterDescriptor : FieldType
   ReturnDescriptor : FieldType | V
\end{verbatim}

A parameter descriptor represents the type of an argument of a method.
A return descriptor represents the type of the return result of a
method.  The character `{\tt V}' indicates that the method returns no
value (its return type is void).

The method descriptor is the same whether it is a static or an instance
method.  Although an instance method is passed \code{this}, a reference
to the current class instance, in addition to its intended arguments,
that fact is not reflected in the method descriptor.

For example, the method descriptor
for the method ``\code{Object foo(int i, double d, Thread t)}'' is
``\code{(IDLjava/lang/Thread;)Ljava/lang/Object;}''.  Note that
internal forms of the fully qualified names of \code{Thread} and
\code{Object} are used in the method descriptor.

\section{Methods}
\label{sec:methods}

The quadcode representation of each method is a unique object of
class \javadoc{joeq.Class.jq_Method}{joeq/Class/jq_Method.html}.  Components of the method,
most notably its control-flow graph,
can be accessed using the following API provided
by that class.

\begin{mytable}{|l|p{4in}|}
\hline
\verb+String getName()+ & Name of the method. \\
\hline
\verb+String getDesc().toString()+ & Descriptor of the method, e.g., ``\code{(Ljava/lang/String;)V}''. \\
\hline
\verb+jq_Class getDeclaringClass()+ & Declaring class of the method. \\
\hline
\verb+ControlFlowGraph getCFG()+ & Control-flow graph of the method.* \\
\hline
\verb+int getLineNumber(int bci)+ & Line number of the given bytecode offset (-1 if not found). \\
\hline
\verb+Quad getQuad(int bci)+ & First quad at the given bytecode
offset (null if not found). \\
\hline
\verb+Quad getQuad(int bci, Class kind)+ & First quad of the given
kind at the given bytecode offset (null if not found). \\
\hline
\verb+Quad getQuad(int bci, Class[] kind)+ & First quad of any
given kind at the given bytecode offset (null if not found). \\
\hline
\verb+String toString()+ & Unique identifier of the method in format \code{mName:mDesc@cName}. \T \\
\hline
\end{mytable}

* The control-flow graph must not be asked if the method is abstract (which can be determined by calling instance method \code{isAbstract()} of \code{jq_Method}).

\section{Control-Flow Graphs}
\label{sec:cfgs}

The control-flow graph (CFG) of each method consists of a set of registers,
called the register factory, and a directed graph whose nodes are basic blocks and
whose edges denote flow of control between basic blocks.

The CFG of each method is a unique object of
class \javadoc{joeq.Compiler.Quad.ControlFlowGraph}{joeq/Compiler/Quad/ControlFlowGraph.html}.
Components of the CFG can be accessed using the following API provided
by that class.

\begin{mytable}{|l|l|}
\hline
\verb+getRegisterFactory()+ & Set of all local variables. \\
\hline
\verb+EntryOrExitBasicBlock entry()+ & Unique entry basic block. \\
\hline
\verb+EntryOrExitBasicBlock exit()+ & Unique exit basic block. \\
\hline
\verb+ListIterator.BasicBlock reversePostOrderIterator()+ & Iterator over all basic blocks in reverse post-order. \\
\hline
\verb+jq_Method getMethod()+ & Containing method of the CFG. \T \\
\hline
\end{mytable}

%\subsection{Register Factory}

The register factory contains one register per argument of the method
(called {\it local variables}) and one register per temporary 
in the method body (called {\it stack variables}).
Temporaries include those declared by programmers as well as those
generated by Joeq.  The reason Joeq can generate temporaries is that the quadcode
representation, which is register-based, is constructed from Java
bytecode, which is stack-based; moreover, Joeq does the Static Single
Assignment (SSA) transformation by default, which introduces
temporaries to ensure that there is at most one static assignment to
any variable.  Registers corresponding to local variables are named {\tt R0}, {\tt R1}, ..., {\tt Rn},
while those corresponding to stack variables are named {\tt Tn+1}, {\tt Tn+2}, ..., {\tt Tm}.

For instance, the register factory of the main method in the example above 
has 3 registers: {\tt R0} denoting the {\tt args} argument of the method
and {\tt T1} and {\tt T2} denoting temporaries generated by Joeq.

Each register factory is a unique object
of class \javadoc{joeq.Compiler.Quad.RegisterFactory}{joeq/Compiler/Quad/RegisterFactory.html}.

Besides the register factory, a CFG has a directed graph whose nodes are basic blocks and whose edges
denote flow of control between basic blocks.
Each basic block contains a straight-line sequence of zero or more primitive statements called quads
(Section \ref{sec:quads}).  Each CFG is guaranteed to contain at least
two basic blocks: a unique entry basic block with no incoming edges
and a unique exit block with no outgoing edges.  The entry and exit
basic blocks do not contain any quads.

Each basic block is a unique object of
class \javadoc{joeq.Compiler.Quad.BasicBlock}{joeq/Compiler/Quad/BasicBlock.html}
(the entry and exit basic blocks are instances of a
subclass \javadoc{joeq.Compiler.Quad.EntryOrExitBasicBlock}{joeq/Compiler/Quad/EntryOrExitBsaicBlock.html}).
Components of the basic block can be accessed using the following API
provided by that class.

\begin{mytable}{|l|l|}
\hline
\verb+int size()+ & Number of quads contained in the basic block. \\
\hline
\verb+Quad getQuad(int index)+ & Quad at the given 0-based index. \\
\hline
\verb+List.BasicBlock getPredecessors()+ & List of immediate predecessor basic blocks. \\
\hline
\verb+List.BasicBlock getSuccessors()+ & List of immediate successor basic blocks. \\
\hline
\verb+jq_Method getMethod()+ & Containing method of the basic block. \T \\
\hline
\end{mytable}

\section{Quads}
\label{sec:quads}

Chord uses format \code{offset!mName:mDesc@cName}, described in
class \javadoc{chord.program.MethodElem}{chord/program/MethodElem.html},
to uniquely identify each bytecode instruction in the input Java
program, where \code{offset} is the (0-based) bytecode offset of the
instruction in its containing method, \code{mName} is the name of the
method, \code{mDesc} is the descriptor of the method, and \code{cName}
is the fully-qualified name of the class declaring the method.  For
instance, ``\code{8!main:[Ljava/lang/String;@test.HelloWorld}''
uniquely identifies the return instruction in the main method
in the example above.

The quadcode representation is register-based, as opposed to Java
bytecode that is used to construct it, which is stack-based.  As a
result, it uses {\it quads} to represent bytecode instructions.  A
quad is a primitive statement that consists of an operator and upto
four operands.  There is no one-to-one correspondence between bytecode
instructions and quads: certain bytecode instructions generate a
sequence of more than one quads while others do not generate any quad.
The API of class \code{jq_Method} provides various \code{getQuad(...)}
methods to access the quad(s)
corresponding to a bytecode instruction (see Section \ref{sec:methods}).

Each quad is a unique object of
class \javadoc{joeq.Compiler.Quad.Quad}{joeq/Compiler/Quad/Quad.html}.
Components of the quad can be accessed using the following API
provided by that class.

\begin{mytable}{|l|l|}
\hline
\verb+Operator getOperator()+ & Kind of the quad. \\
\hline
\verb+int getBCI()+ & Bytecode offset of the quad in its containing
method. \\
\hline
\verb+String toByteLocStr()+ & Unique identifier of the quad in
format \code{offset!mName:mDesc@cName}. \\
\hline
\verb+String toJavaLocStr()+ & Location of the quad in
format \code{fileName:lineNum} in Java source code. \\
\hline
\verb+String toLocStr()+ & Location of the quad in both Java bytecode and source code. \\
\hline
\verb+String toVerboseStr()+ & Verbose description of the quad (its location plus contents). \\
\hline
\verb+jq_Method getMethod()+ & Containing method of the quad. \T \\
\hline
\end{mytable}

The kind of each quad is determined by its operator which is a unique object of
the appropriate subclass
of \javadoc{joeq.Compiler.Quad.Operator}{joeq/Compiler/Quad/Operator.html}
in the following hierarchy:

\begin{verbatim}
Operator
    |
    |--- Move
    |--- Phi
    |--- Unary
    |--- Binary
    |--- New
    |--- NewArray
    |--- MultiNewArray
    |--- Getstatic
    |--- Putstatic
    |--- ALoad
    |--- AStore
    |--- Getfield
    |--- Putfield
    |--- CheckCast
    |--- InstanceOf
    |--- ALength
    |--- Return
    |
    |--- Branch
    |       |
    |       |--- IntIfCmp
    |       |--- Goto
    |       |--- Jsr
    |       |--- Ret
    |       |--- LookupSwitch
    |       |--- TableSwitch
    |
    |--- Invoke
    |       |
    |       |--- InvokeVirtual
    |       |--- InvokeStatic
    |       |--- InvokeInterface
    |
    |--- Monitor
            |
            |--- MONITORENTER
            |--- MONITOREXIT
\end{verbatim}

The number and kinds of operands of each quad depends upon the kind of
the operator.  Each of the above subclasses of \code{Operator}
provides an API to access the operands of the quad.  For instance,
the components of a \code{Getfield} quad {\tt q} of the form ``l = b.f'' can be accessed as follows:

\begin{framed}
\begin{verbatim}
Operand lo = Getfield.getSrc(q);
Operand bo = Getfield.getBase(q);
if (lo instanceof RegisterOperand && bo instanceof RegisterOperand) {
    Register l = ((RegisterOperand) lo).getRegister();
    Register b = ((RegisterOperand) bo).getRegister();
    jq_Field f = Getfield.getField(q).getField();
    ...
}
\end{verbatim}
\end{framed}

\section{Traversing Quadcode}
\label{sec:traversing}

A common way to traverse all quads in the input Java program is as follows:

\texonly{\newpage}

\begin{framed}
\begin{verbatim}

import chord.program.Program;
import joeq.Compiler.Quad.QuadVisitor;
import joeq.Class.jq_Method;
import joeq.Compiler.Quad.ControlFlowGraph;
import joeq.Util.Templates.ListIterator;
import joeq.Compiler.Quad.BasicBlock;
import joeq.Compiler.Quad.Quad;

QuadVisitor qv = new QuadVisitor.EmptyVisitor() {
    public void visitMove(Quad q) { ... }
    public void visitPhi(Quad q) { ... }
    public void visitUnary(Quad q) { ... }
    ...
};
Program program = Program.g();
for (jq_Method m : program.getMethods()) {
    if (!m.isAbstract()) {
        ControlFlowGraph cfg = m.getCFG();
        ListIterator.BasicBlock it = cfg.reversePostOrderIterator();
        while (it.hasNext()) {
            BasicBlock b = it.nextBasicBlock();
            for (int i = 0; i < b.size(); i++) {
                Quad q = b.getQuad(i);
                q.accept(qv);
            }
        }
    }
}
\end{verbatim}
\end{framed}

