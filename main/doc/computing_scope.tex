\xname{computing_scope}
\chapter{Computing Analysis Scope}
\label{chap:computing-scope}

A pre-requisite to analyzing a Java program using any program analysis framework,
including Chord, is to compute the {\it analysis scope}: the set of methods
of the program to be analyzed.  Several scope construction algorithms (so-called
call-graph algorithms) exist in the literature that differ in scalability (i.e., how large
a program they can handle with the available resources) and precision (i.e., how small a
set of methods they deem is reachable).

Chord implements several standard scope construction algorithms.  Besides scalability and precision,
an additional metric of these algorithms in Chord that can be controlled by users is usability,
which concerns aspects such as excluding certain methods from being analyzed even if they
are reachable, and modeling Java features such as reflection,
dynamic class loading, and native methods.
These features affect which methods are
reachable but, in general, they cannot be modeled soundly by any program analysis framework.  The best
a framework can do is
 provide stubs for commonly-used native methods in the standard JDK library (e.g., the \code{arraycopy}
method of class \code{java.lang.System}), 
offer users a range of options on how to resolve reflection (e.g.,
an option might be running the program and observing how reflection is resolved), etc.

Chord computes the analysis scope of the given
program either if property \code{chord.build.scope} is set to {\tt true} or if some other task
(e.g., a program analysis specified via property \code{chord.run.analyses}) demands it.
The following sections describe Chord's analysis scope 
computation in detail.  Section \ref{sec:scope-reuse} describes how to reuse the analysis scope
computed in a previous run of Chord for a given program.  Section \ref{sec:scope-algos} describes
Chord's analysis scope construction algorithms.  Finally, Section \ref{sec:scope-exclude} describes how users
can exclude certain classes from the analysis scope.

%Users can set the following properties to control scope computation:
%
%\begin{itemize}
%\item \code{chord.main.class}
%\item \code{chord.class.path}
%\item \code{chord.reuse.scope}
%\item \code{chord.methods.file}
%\item \code{chord.reflect.file}
%\item \code{chord.scope.kind}
%\item \code{chord.ch.kind}
%\item \code{chord.reflect.kind}
%\item \code{chord.std.scope.exclude}
%\item \code{chord.ext.scope.exclude}
%\item \code{chord.scope.exclude}
%\end{itemize}
%
%The meaning of these properties is explained on demand in

\section{Scope Reuse}
\label{sec:scope-reuse}

If property \code{chord.reuse.scope} has value {\tt true} and both
files specified by properties \code{chord.methods.file} and
\code{chord.reflect.file} exist, then Chord regards those files as
specifying which methods to consider reachable and how to resolve
reflection, respectively.

The format of the file specified by property \code{chord.methods.file}
is a list of zero or more lines, where each line is of the form
\code{mname:mdesc@cname}
specifying the method's name {\tt mname}, the method's descriptor
{\tt mdesc}, and the method's declaring class {\tt cname} (e.g.,
\code{main:([Ljava/lang/String;)V@foo.bar.Main}).

The format of the file specified by property \code{chord.reflect.file}
is of the form:

\begin{framed}
\begin{verbatim}
# resolvedClsForNameSites
...
# resolvedObjNewInstSites
...
# resolvedConNewInstSites
...
# resolvedAryNewInstSites
...
\end{verbatim}
\end{framed}

where each of the above ``{\tt ...}'' is a list of zero or more lines, where
each line is of the form
\code{bci!mname:mdesc@cname->type1,type2,...typeN}
meaning the call site at bytecode offset {\tt bci} in the
method denoted by {\tt mname:mdesc@cname} may resolve to any of
reference types {\tt type1}, {\tt type2}, ..., {\tt typeN}.
The meaning of the above four sections is as follows.
\begin{itemize}
\item {\tt resolvedClsForNameSites} lists
each call to static method {\tt forName(String)} defined in class
\code{java.lang.Class}, along with a list of the types of the named
classes.
\item {\tt resolvedObjNewInstSites} lists
each call to instance method {\tt newInstance()} defined in class
\code{java.lang.Class}, along with a list of the types of the
instantiated classes.
\item {\tt resolvedConNewInstSites} lists
each call to instance method {\tt newInstance(Object[])} defined in class
\code{java.lang.reflect.Constructor}, along with a list of the types of the
instantiated classes.
\item {\tt resolvedAryNewInstSites} lists
each call to instance method {\tt newInstance(Class,int)} defined in class
\code{java.lang.reflect.Array}, along with a list of the types of the
instantiated classes.
\end{itemize}
The default value of property \code{chord.reuse.scope} is {\tt false}.
The default value of properties \code{chord.methods.file} and
\code{chord.reflect.file} is \code{[chord.out.dir]/methods.txt} and
\code{[chord.out.dir]/reflect.txt}, respectively.
Property \code{chord.out.dir} denotes the output directory of Chord;
its default value is \code{[chord.work.dir]/chord_output/}.
Property \code{chord.work.dir} denotes the working directory during
Chord's execution; its default value is the current directory.

\section{Scope Construction Algorithms}
\label{sec:scope-algos}

If property \code{chord.reuse.scope} has value {\tt false} or the
files specified by properties \code{chord.methods.file} or
\code{chord.reflect.file} do not exist, then Chord computes analysis
scope using the algorithm specified by property
\code{chord.scope.kind} and then writes the list of methods deemed
reachable and the reflection resolved by that algorithm to the files
specified by properties \code{chord.methods.file} and
\code{chord.reflect.file}, respectively.

The possible values of property \code{chord.scope.kind} are
[\code{dynamic}$|$\code{rta}$|$\code{cha}] (the default value is {\tt rta}).
The following subsections describe the scope construction algorithm
that Chord runs in each of these three cases.
In each case, Chord at
least expects properties \code{chord.main.class} and
\code{chord.class.path} to be set to the fully-qualified name of the
program's main class (e.g., \code{com.example.Main}) and the
program's application classpath, respectively.

\subsection{Dynamic Scope Construction}

If property \code{chord.scope.kind} has value {\tt dynamic}, then Chord
computes analysis scope
dynamically, by running the program and observing the
classes that are loaded at run-time.  The number of times the program
is run and the command-line arguments to be supplied to the program in
each run is specified by properties \code{chord.run.ids} and
\code{chord.args.<id>} for each run ID {\tt <id>}.  By default, the
program is run only once, using run ID {\tt 0}, and without any
command-line arguments.  Only classes loaded in some run are regarded
as reachable but {\it all} methods of each loaded class are regarded
as reachable regardless of whether they were invoked in the run.  The
rationale behind this decision is to both reduce the run-time instrumentation
overhead and increase the predictive power of program analyses
performed using the computed analysis scope.

\subsection{RTA Scope Construction}

If property \code{chord.scope.kind} has value {\tt rta}, then Chord
computes analysis scope statically using Rapid Type Analysis (RTA).
RTA is an iterative fixed-point algorithm.  It maintains a set of
reachable methods $M$.  The initial iteration starts by assuming that
only the main method in the main class is reachable (Chord also
handles class initializer methods but we ignore them here for brevity;
we also ignore the set of reachable classes maintained besides the set
of reachable methods).  All object allocation sites $H$ contained in
methods in $M$ are deemed reachable (i.e., control-flow within method
bodies is ignored).  Whenever a dynamically-dispatching method call
site (i.e., an invokevirtual or invokeinterface site) with receiver of
static type $t$ is encountered in a method in $M$, only subtypes of
$t$ whose objects are allocated at some site in $H$ are considered to
determine the possible target methods, and each such target method is
added to $M$.  The process terminates when no more methods can be
added.

%RTA is a relatively inexpensive and precise algorithm in practice.
%Its key shortcoming is that it makes no attempt to resolve
%reflection, which is rampant in real-world Java programs, and can
%therefore be unsound (i.e., underestimate the set of reachable
%classes and methods).  The next option attempts to overcome this
%problem.  \item The \code{rta_reflect} value instructs Chord to
%compute analysis scope statically using Rapid Type Analysis and,
%moreover, to resolve a common reflection
%pattern: \begin{quote} \begin{verbatim} String s = ...; Class c =
%Class.forName(s); Object o = c.newInstance(); T t = (T)
%o; \end{verbatim} \end{quote} This analysis is identical to RTA
%except that it additionally inspects every cast statement in the
%program, such as the last statement in the above snippet, and queries
%the class hierarchy to find all concrete classes that subclass
%\code{T} (if \code{T} is a class) or that implement \code{T} (if
%\code{T} is an interface).  Chord allows users to control which
%classes are included in the class hierarchy (see Section
%\ref{sec:cha}).

\subsection{CHA Scope Construction}

If property \code{chord.scope.kind} has value {\tt cha}, then Chord
computes analysis scope statically using Class Hierarchy Analysis (CHA).
The key difference between CHA and RTA is that for invokevirtual and
invokeinterface sites with receiver of static type $t$, CHA considers
{\it all} subtypes of $t$ in the class hierarchy to determine the
possible target methods, whereas RTA restricts them to types of
objects allocated in methods deemed reachable so far.  As a result,
CHA is highly imprecise in practice, and also expensive since it
grossly overestimates the set of reachable classes and methods.
Nevertheless, Chord allows users to control which classes are
included in the class hierarchy, and thereby control the
precision and cost of CHA, by setting property \code{chord.ch.kind},
whose possible values are [\code{static}$|$\code{dynamic}] (the default
value is {\tt static}).

Chord first constructs the entire classpath of the given program by
concatenating in order the following classpaths:

\begin{enumerate}
\item
The boot classpath, specified by property \code{sun.boot.class.path}.
\item
The library extensions classpath, comprising all jar files in
directory \code{[java.home]/lib/dir/}.
\item
The application classpath of the given program, specified by property \code{chord.class.path},
which is empty by default.
\end{enumerate}

All classes in the entire classpath (resulting from items 1--3 above)
are included in the class hierarchy with the following exceptions:
\begin{itemize}
\item
Duplicate classes, i.e., classes with the same name occurring in more
than one classpath element; in this case, all occurrences except the
first are excluded.
\item
Any class whose name's prefix is specified in the value of property
\code{chord.scope.exclude} (see Section \ref{sec:scope-exclude}).
\item
If property \code{chord.ch.kind} has value {\tt dynamic}, then
Chord runs the given program and observes the set of all classes the
JVM loads; any class not in this set is excluded.
\item
If the superclass of a class C is missing or if an interface
implemented/extended by a class/interface C is missing, where
``missing" means that it is either not in the classpath resulting from
items 1--3 above or it is excluded by one of these rules, then C
itself is excluded.  Note that this rule is recursive, e.g., if C has
superclass B which in turn has superclass A, and A is missing, then
both B and C are excluded.
\end{itemize}

\section{Scope Exclusion}
\label{sec:scope-exclude}

Chord can be instructed to exclude certain classes in a given program 
from being analyzed.
This functionality might be desirable, for instance, if the given program
contains a larger framework (e.g., Hadoop or Android) which
must not be analyzed.
Chord provides three properties for this purpose.
The value of each of these properties is a comma-separated list of prefixes
of names of classes.  Chord treats the body of each method defined in
each such class as a no-op.

\begin{itemize}
\item
Property \code{chord.std.scope.exclude} is intended to specify
classes to be excluded from the scope of {\it all} programs to be
analyzed, e.g., classes in the JDK standard library.  Its default
value is the empty list.
\item
Property \code{chord.ext.scope.exclude} is intended to specify
classes to be excluded from the scope of specific programs to be
analyzed.  Its default value is the empty list.
\item
Property \code{chord.scope.exclude} specifies the final list of
classes to be excluded from scope.  Its default value is
\code{[chord.std.scope.exclude]:[chord.ext.scope.exclude]}.
\end{itemize}

{\bf Note:} The value of each of the above properties is a list of {\it prefixes},
not {\it regular expressions}.  A valid value is \code{java.,com.sun.},
but not \code{java.*,com.sun.*}.

%\begin{tabular}
%chord.scope.kind,chord.ch.dynamic & runs program? & \# reachable classes & \# reachable methods & running time \\
%dynamic,-         & yes &    314 &  4,491 &   25s \\
%cha,true          & yes &    427 &  1,532 &   28s \\
%rta,-             &  no &    849 &  4,836 &    9s \\
%rta_reflect,true  & yes &    849 &  4,836 &   34s \\
%rta_reflect,false &  no &  9,871 & 58,726 &  7m6s \\
%cha,false         &  no & 14,121 & 74,613 & 4m11s
%\end{tabular}

