\section{Building Analysis Scope}
\label{sec:building-scope}

Chord computes the analysis scope (i.e., reachable classes and methods) of a given Java program
either if property \code{chord.build.scope} is set to \code{true} or if some other task (e.g.,
a program analysis specified via property \code{chord.run.analyses}) demands it by
calling static method \code{chord.program.Program.v()}.

The algorithm used to compute the analysis scope is as follows.

\begin{itemize}
\item
If property \code{chord.reuse.scope} has value \code{true} and the files specified by properties
\code{chord.classes.file} and \code{chord.methods.file} exist,
then Chord regards those files as specifying the classes and methods, respectively,
to be regarded as reachable.  The format
of the classes file is a fully-qualified class name per line (e.g., \code{foo.bar.Main}).  The format
of the methods file is an entry of the form \code{<method name>:<method descriptor>@<class name>} per line,
specifying the method's name, the method's descriptor, and the method's declaring class
(e.g., \code{main:([Ljava/lang/String;)V@foo.bar.Main}).
Chord throws a runtime exception if the declaring class of a method in the methods file is not
specified in the classes file.

\item
If property \code{chord.reuse.scope} has value \code{false} or the file specified by
property \code{chord.classes.file} or \code{chord.methods.file} does not exist,
then Chord computes the analysis scope
using the algorithm specified by property \code{chord.scope.kind} and then
writes the classes and methods deemed reachable by that algorithm to those files.

The possible legal values of property \code{chord.scope.kind} are \code{rta} and \code{dynamic}.
In both cases, Chord expects properties \code{chord.main.class} and \code{chord.class.path}
to be set.

\begin{itemize}
\item
The \code{rta} value instructs Chord to compute the analysis scope statically using Rapid Type Analysis.

\item
The \code{dynamic} value instructs Chord to compute the analysis scope dynamically, by running the program
and observing using JVMTI the classes that are loaded at run-time.
The number of times the program is run and the command-line arguments to be supplied to
the program in each run is specified by properties \code{chord.run.ids} and
\code{chord.args.<id>} for each run ID \code{<id>}.  By default, the program is run only once, using run ID \code{0},
and without any command-line arguments.
Only classes loaded in some run are regarded as reachable but {\it all} methods of each loaded class are regarded
as reachable regardless of whether they were invoked in the run.
The rationale behind this decision is to both reduce run-time overhead of JVMTI and to increase the
predictive power of program analyses performed using the computed analysis scope.
\end{itemize}
\end{itemize}

The default value of property \code{chord.reuse.scope} is \code{false} and that of
property \code{chord.scope.kind} is \code{rta}.
The default values of properties \code{chord.classes.file} and \code{chord.methods.file} are
\code{[chord.out.dir]/classes.txt} and \code{[chord.out.dir]/methods.txt}, respectively,
where property \code{chord.out.dir} defaults to \code{[chord.work.dir]/chord_output/},
and property \code{chord.work.dir} defaults to the current working directory.

