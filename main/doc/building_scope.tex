\section{Building Analysis Scope}
\label{sec:building-scope}

\Chord\ computes the analysis scope (i.e., reachable classes and methods) of the given program
either if property \code{chord.build.scope} is set to \code{true} or if some other task (e.g.,
a program analysis specified via property \code{chord.run.analyses}) demands it by
calling method \code{chord.program.Program.v()}.

The algorithm used to compute the analysis scope is as follows.

\begin{itemize}
\item
If property \code{chord.reuse.scope} has value \code{true} and the files specified by properties
\code{chord.classes.file} and \code{chord.methods.file} exist,
then \Chord\ regards those files as specifying the classes and methods, respectively,
to be regarded as reachable.  The format
of the classes file is a fully-qualified class name per line (e.g., \code{foo.bar.Main}).  The format
of the methods file is an entry of the form \code{<method name>:<method descriptor>@<class name>} per line,
specifying the method's name, the method's descriptor, and the method's declaring class
(e.g., \code{main:([Ljava/lang/String;)V@foo.bar.Main}).

The default value of property \code{chord.reuse.scope} is \code{false}, and
the default values of properties \code{chord.classes.file} and \code{chord.methods.file} are
\code{[chord.out.dir]/classes.txt} and \code{[chord.out.dir]/methods.txt}, respectively,
where property \code{chord.out.dir} defaults to \code{[chord.work.dir]/chord_output/},
and property \code{chord.work.dir} defaults to the current working directory.

\item
If property \code{chord.reuse.scope} has value \code{false} or the file specified by
property \code{chord.classes.file} or \code{chord.methods.file} does not exist,
then \Chord\ computes the analysis scope
using the algorithm specified by property \code{chord.scope.kind} and
then writes the classes and methods deemed reachable by that algorithm to those files.

The possible legal values of property \code{chord.scope.kind} are [\code{dynamic}$|$\code{rta}$|$\code{rta_reflect}$|$\code{cha}].
In each case, \Chord\ at least expects properties \code{chord.main.class} and \code{chord.class.path}
to be set.

\begin{itemize}
\item
The \code{dynamic} value instructs \Chord\ to compute the analysis scope dynamically, by running the program
and observing using JVMTI the classes that are loaded at run-time.
The number of times the program is run and the command-line arguments to be supplied to
the program in each run is specified by properties \code{chord.run.ids} and
\code{chord.args.<id>} for each run ID \code{<id>}.  By default, the program is run only once, using run ID \code{0},
and without any command-line arguments.
Only classes loaded in some run are regarded as reachable but {\it all} methods of each loaded class are regarded
as reachable regardless of whether they were invoked in the run.
The rationale behind this decision is to both reduce run-time overhead of JVMTI and to increase the
predictive power of program analyses performed using the computed analysis scope.

\item
The \code{rta} value instructs \Chord\ to compute the analysis scope statically using Rapid Type Analysis (RTA).
In this case, no attempt is made to resolve reflection.

RTA is an iterative fixed-point algorithm.  It maintains a set of reachable methods $M$.
The initial iteration starts by assuming that only the main method in the main class is reachable
(\Chord\ also handles class initializer methods but we ignore them here for brevity; we also ignore
the set of reachable classes maintained besides the set of reachable methods).
All object allocation sites $H$ contained in methods in $M$ are deemed
reachable (i.e., control-flow within method bodies is ignored).  Whenever a dynamically-dispatching
method call site (i.e., an invokevirtual or invokeinterface site) with receiver of static
type $t$ is encountered in a method in $M$, only subtypes of $t$ whose objects are allocated at some site in $H$
are considered to determine the possible target methods, and each such target
method is added to $M$.  The process terminates when no more methods can be added.

RTA is a relatively inexpensive and precise algorithm in practice.  Its key shortcoming is that it makes
no attempt to resolve reflection, which is rampant in real-world Java programs, and can therefore be
unsound (i.e., underestimate the set of reachable classes and methods).
The next option attempts to overcome this problem.

\item
The \code{rta_reflect} value instructs \Chord\ to compute the analysis scope statically using Rapid Type Analysis
and, moreover, to resolve a common reflection pattern:

\begin{quote}
\begin{verbatim}
String s = ...;
Class c = Class.forName(s);
Object o = c.newInstance();
T t = (T) o;
\end{verbatim}
\end{quote}

This analysis is identical to RTA except that it additionally inspects every cast statement in the program,
such as the last statement in the above snippet,
and queries the class hierarchy to find all concrete classes that subclass \code{T} (if \code{T} is a class)
or that implement \code{T} (if \code{T} is an interface).
\Chord\ allows users to control which classes are included in the class hierarchy (see below).

\item
The \code{cha} value instructs \Chord\ to compute the analysis scope statically using Class Hierarchy Analysis (CHA).
The key difference between CHA and RTA is that for invokevirtual and invokeinterface sites with receiver of
static type $t$, CHA considers {\it all} subtypes of $t$ in the class hierarchy to determine the possible
target methods, whereas RTA restricts them to types of objects allocated in methods deemed reachable so far.
As a result, CHA is highly imprecise in practice, and also expensive since it grossly overestimates the set
of reachable classes and methods.
Nevertheless, \Chord\ allows users to control which classes are included in the class hierarchy (see below)
and thereby control the precision and cost of CHA.

\end{itemize}
The default value of property \code{chord.scope.kind} is \code{rta}.
\end{itemize}

The class hierarchy is built if property \code{chord.scope.kind} has value \code{rta_reflect} or \code{cha}.
Users can control which classes are included in building the class hierarchy.
\Chord\ first constructs the entire classpath of the given program by concatenating in order the following
classpaths:

\begin{enumerate}
\item
The boot classpath, specified by property \code{sun.boot.class.path}.
\item
The library extensions classpath, comprising all jar files in directory \code{[java.home]/lib/dir/}.
\item
\Chord's classpath, specified by property \code{chord.main.class.path}.
\item
The classpath of user-defined Java analyses, specified by user-defined property \code{chord.java.analysis.path}
(it is empty by default).
\item
The classpath of the given program, specified by user-defined property \code{chord.class.path}
(it is empty by default).
\end{enumerate}

All classes in the entire classpath (resulting from items 1--5 above) are included in the class hierarchy
with the following exceptions:
\begin{itemize}
\item
Duplicate classes, i.e., classes with the same name occurring in more than one classpath element;
in this case, all occurrences except the first are excluded. 
\item
All classes in \Chord's classpath are excluded, i.e., all classes in the classpath specified by property \code{chord.main.class.path},
such as those in packages and sub-packages of \code{chord}, \code{joeq}, \code{net.sf.bddbddb}, \code{net.sf.javabdd}, \code{javassist}, \code{gnu.trove}, \code{net.sf.saxon}, etc.
\item
If property \code{chord.ch.kind} has value \code{dynamic}, then \Chord\ runs the given program and
observes the set of all classes the JVM loads; any class not in this set is excluded.
By default, property \code{chord.ch.dynamic} has value \code{static}.
\item
If the superclass of a class $c$ is missing or if an interface implemented/extended by a class/interface
$c$ is missing, where ``missing" means that it is either not in the classpath resulting from items 1--5 above
or it is excluded by one of these rules, then $c$ itself is excluded.  Note that this rule is recursive, e.g.,
if $c$ has superclass $b$ which in turn has superclass $a$, and $a$ is missing, then both $b$ and $c$ are
excluded.
\end{itemize}

%TODO: describe chord.scope.exclude?

%\begin{tabular}
%chord.scope.kind,chord.ch.dynamic & runs program? & \# reachable classes & \# reachable methods & running time \\
%dynamic,-         & yes &    314 &  4,491 &   25s \\
%cha,true          & yes &    427 &  1,532 &   28s \\
%rta,-             &  no &    849 &  4,836 &    9s \\
%rta_reflect,true  & yes &    849 &  4,836 &   34s \\
%rta_reflect,false &  no &  9,871 & 58,726 &  7m6s \\
%cha,false         &  no & 14,121 & 74,613 & 4m11s
%\end{tabular}

