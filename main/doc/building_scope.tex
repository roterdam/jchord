\section{Computing Program Scope}
\label{sec:building-scope}

\Chord\ computes the scope (i.e., reachable methods) of the given
program either if property \code{chord.build.scope} is set to {\tt
  true} or if some other task (e.g., a program analysis specified via
property \code{chord.run.analyses}) demands it by calling method
\code{chord.program.Program.g()}.

Users can set the following properties to control scope computation:

\begin{itemize}
\item \code{chord.main.class}
\item \code{chord.class.path}
\item \code{chord.reuse.scope}
\item \code{chord.methods.file}
\item \code{chord.reflect.file}
\item \code{chord.scope.kind}
\item \code{chord.ch.kind}
\item \code{chord.reflect.kind}
\item \code{chord.std.scope.exclude}
\item \code{chord.ext.scope.exclude}
\item \code{chord.scope.exclude}
\end{itemize}

\subsection{Reusing Program Scope}

If property \code{chord.reuse.scope} has value {\tt true} and both
files specified by properties \code{chord.methods.file} and
\code{chord.reflect.file} exist, then \Chord\ regards those files as
specifying which methods to consider reachable and how to resolve
reflection, respectively.

The format of the file specified by property \code{chord.methods.file}
is a list of zero or more lines, where each line is of the form
\code{mname:mdesc@cname}
specifying the method's name {\tt mname}, the method's descriptor
{\tt mdesc}, and the method's declaring class {\tt cname} (e.g.,
\code{main:([Ljava/lang/String;)V@foo.bar.Main}).

The format of the file specified by property \code{chord.reflect.file}
is of the form:
\begin{verbatim}
# resolvedClsForNameSites
...
# resolvedObjNewInstSites
...
# resolvedConNewInstSites
...
# resolvedAryNewInstSites
...
\end{verbatim}
where each of the above ``{\tt ...}'' is a list of zero or more lines, where
each line is of the form
\code{bci!mname:mdesc@cname->type1,type2,...typeN}
meaning the call site at bytecode offset {\tt bci} in the
method denoted by {\tt mname:mdesc@cname} may resolve to any of
reference types {\tt type1}, {\tt type2}, ..., {\tt typeN}.
The meaning of the above four sections is as follows.
\begin{itemize}
\item {\tt resolvedClsForNameSites} lists
each call to static method {\tt forName(String)} defined in class
{\tt java.lang.Class}, along with a list of the types of the named
classes.
\item {\tt resolvedObjNewInstSites} lists
each call to instance method {\tt newInstance()} defined in class
{\tt java.lang.Class}, along with a list of the types of the
instantiated classes.
\item {\tt resolvedConNewInstSites} lists
each call to instance method {\tt newInstance(Object[])} defined in class
{\tt java.lang.reflect.Constructor}, along with a list of the types of the
instantiated classes.
\item {\tt resolvedAryNewInstSites} lists
each call to instance method {\tt newInstance(Class,int)} defined in class
{\tt java.lang.reflect.Array}, along with a list of the types of the
instantiated classes.
\end{itemize}
The default value of property \code{chord.reuse.scope} is {\tt false}.
The default value of properties \code{chord.methods.file} and
\code{chord.reflect.file} is \code{[chord.out.dir]/methods.txt} and
\code{[chord.out.dir]/reflect.txt}, respectively.
Property \code{chord.out.dir} defaults to
\code{[chord.work.dir]/chord_output/} and property
\code{chord.work.dir} defaults to the working directory during Chord's
execution.

\subsection{Scope Construction Algorithms}

If property \code{chord.reuse.scope} has value {\tt false} or the
files specified by properties \code{chord.methods.file} or
\code{chord.reflect.file} do not exist, then \Chord\ computes program
scope using the algorithm specified by property
\code{chord.scope.kind} and then writes the list of methods deemed
reachable and the reflection resolved by that algorithm to the files
specified by properties \code{chord.methods.file} and
\code{chord.reflect.file}, respectively.

The possible legal values of property \code{chord.scope.kind} are
[\code{dynamic}$|$\code{rta}$|$\code{cha}].  In each case, \Chord\ at
least expects properties \code{chord.main.class} and
\code{chord.class.path} to be set.

\begin{itemize}
\item
The {\tt dynamic} value instructs \Chord\ to compute program scope
dynamically, by running the program and observing using JVMTI the
classes that are loaded at run-time.  The number of times the program
is run and the command-line arguments to be supplied to the program in
each run is specified by properties \code{chord.run.ids} and
\code{chord.args.<id>} for each run ID {\tt <id>}.  By default, the
program is run only once, using run ID {\tt 0}, and without any
command-line arguments.  Only classes loaded in some run are regarded
as reachable but {\it all} methods of each loaded class are regarded
as reachable regardless of whether they were invoked in the run.  The
rationale behind this decision is to both reduce run-time overhead of
JVMTI and to increase the predictive power of program analyses
performed using the computed program scope.

\item
The {\tt rta} value instructs \Chord\ to compute program scope
statically using Rapid Type Analysis (RTA).

RTA is an iterative fixed-point algorithm.  It maintains a set of
reachable methods $M$.  The initial iteration starts by assuming that
only the main method in the main class is reachable (\Chord\ also
handles class initializer methods but we ignore them here for brevity;
we also ignore the set of reachable classes maintained besides the set
of reachable methods).  All object allocation sites $H$ contained in
methods in $M$ are deemed reachable (i.e., control-flow within method
bodies is ignored).  Whenever a dynamically-dispatching method call
site (i.e., an invokevirtual or invokeinterface site) with receiver of
static type $t$ is encountered in a method in $M$, only subtypes of
$t$ whose objects are allocated at some site in $H$ are considered to
determine the possible target methods, and each such target method is
added to $M$.  The process terminates when no more methods can be
added.

%RTA is a relatively inexpensive and precise algorithm in practice.  Its key shortcoming is that it makes
%no attempt to resolve reflection, which is rampant in real-world Java programs, and can therefore be
%unsound (i.e., underestimate the set of reachable classes and methods).
%The next option attempts to overcome this problem.
%
%\item
%The \code{rta_reflect} value instructs \Chord\ to compute program scope statically using Rapid Type Analysis
%and, moreover, to resolve a common reflection pattern:
%
%\begin{quote}
%\begin{verbatim}
%String s = ...;
%Class c = Class.forName(s);
%Object o = c.newInstance();
%T t = (T) o;
%\end{verbatim}
%\end{quote}
%
%This analysis is identical to RTA except that it additionally inspects every cast statement in the program,
%such as the last statement in the above snippet,
%and queries the class hierarchy to find all concrete classes that subclass \code{T} (if \code{T} is a class)
%or that implement \code{T} (if \code{T} is an interface).
%\Chord\ allows users to control which classes are included in the class hierarchy (see below).
%
\item
The {\tt cha} value instructs \Chord\ to compute program scope
statically using Class Hierarchy Analysis (CHA).

The key difference between CHA and RTA is that for invokevirtual and
invokeinterface sites with receiver of static type $t$, CHA considers
{\it all} subtypes of $t$ in the class hierarchy to determine the
possible target methods, whereas RTA restricts them to types of
objects allocated in methods deemed reachable so far.  As a result,
CHA is highly imprecise in practice, and also expensive since it
grossly overestimates the set of reachable classes and methods.
Nevertheless, \Chord\ allows users to control which classes are
included in the class hierarchy (see below) and thereby control the
precision and cost of CHA.

\end{itemize}
The default value of property \code{chord.scope.kind} is {\tt rta}.

\subsection{Class Hierarchy Construction}

The class hierarchy is built if property \code{chord.scope.kind} has
value {\tt cha}.  Users can control which classes are included in
building the class hierarchy by setting property \code{chord.ch.kind},
whose possible legal values are [\code{static}$|$\code{dynamic}].
\Chord\ first constructs the entire classpath of the given program by
concatenating in order the following classpaths:

\begin{enumerate}
\item
The boot classpath, specified by property \code{sun.boot.class.path}.
\item
The library extensions classpath, comprising all jar files in
directory \code{[java.home]/lib/dir/}.
\item
\Chord's classpath, specified by property
\code{chord.main.class.path}.  This includes the classpath of any
user-defined Java analyses, specified by user-defined property
\code{chord.java.analysis.path}, which is empty by default.
\item
The classpath of the given program, specified by user-defined property
\code{chord.class.path}, which is empty by default.
\end{enumerate}

All classes in the entire classpath (resulting from items 1--4 above)
are included in the class hierarchy with the following exceptions:
\begin{itemize}
\item
Duplicate classes, i.e., classes with the same name occurring in more
than one classpath element; in this case, all occurrences except the
first are excluded.
\item
Any class whose name's prefix is specified in the value of property
\code{chord.scope.exclude}.
\item
If property \code{chord.ch.kind} has value {\tt dynamic}, then
\Chord\ runs the given program and observes the set of all classes the
JVM loads; any class not in this set is excluded.  By default,
property \code{chord.ch.kind} has value {\tt static}.
\item
If the superclass of a class C is missing or if an interface
implemented/extended by a class/interface C is missing, where
``missing" means that it is either not in the classpath resulting from
items 1--4 above or it is excluded by one of these rules, then C
itself is excluded.  Note that this rule is recursive, e.g., if C has
superclass B which in turn has superclass A, and A is missing, then
both B and C are excluded.
\end{itemize}

\subsection{Scope Exclusion}

%TODO: describe chord.scope.exclude?

%\begin{tabular}
%chord.scope.kind,chord.ch.dynamic & runs program? & \# reachable classes & \# reachable methods & running time \\
%dynamic,-         & yes &    314 &  4,491 &   25s \\
%cha,true          & yes &    427 &  1,532 &   28s \\
%rta,-             &  no &    849 &  4,836 &    9s \\
%rta_reflect,true  & yes &    849 &  4,836 &   34s \\
%rta_reflect,false &  no &  9,871 & 58,726 &  7m6s \\
%cha,false         &  no & 14,121 & 74,613 & 4m11s
%\end{tabular}

