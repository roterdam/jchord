\section{Building the Analysis Scope of a Java Program}
\label{sec:building-scope}

Chord computes the analysis scope (i.e. reachable classes and methods) of a given Java program
either if property \verb+chord.build.scope+ is set to \verb+true+ or if some other task (e.g.,
a program analysis specified via property \verb+chord.run.analyses+) demands it by
calling static method \verb+chord.program.Program.v()+.

The algorithm used to compute the analysis scope is specified by 
properties \verb+chord.scope.kind+ and \verb+chord.reuse.scope+.

\begin{itemize}
\item
If property \verb+chord.reuse.scope+ is set to \verb+true+ and the files specified by
properties \verb+chord.classes.file+ and \verb+chord.methods.file+ exist,
then Chord regards those files as specifying the classes and methods, respectively,
to be regarded as reachable.  The format
of the classes file is a class name per line (e.g. \verb+foo.Main+).  The format
of the methods file is an entry of the form \verb+<method name>:<method descriptor>@<class name>+ per line,
specifying the method's name, the method's descriptor, and the method's declaring class
(e.g. \verb+main:([Ljava/lang/String;)V@foo.Main+).
Chord throws a runtime exception if the declaring class of a method in the methods file is not
specified in the classes file.
\item
If property \verb+chord.reuse.scope+ is set to \verb+false+ or the file specified by
property \verb+chord.classes.file+ or \verb+chord.methods.file+ does not exist,
then Chord computes the analysis scope
using the algorithm specified by property \verb+chord.scope.kind+ and then
writes the classes and methods deemed reachable by that algorithm to those files.

The possible legal values of property \verb+chord.scope.kind+ are \verb+rta+ and \verb+dynamic+.
In both cases, Chord expects properties \verb+chord.main.class+ and \verb+chord.class.path+
to be set.
\begin{itemize}
\item
The \verb+rta+ value instructs Chord to compute the analysis scope statically using Rapid Type Analysis.
\item
The \verb+dynamic+ value instructs Chord to compute the analysis scope dynamically, by running the program
and observing using JVMTI the classes that are loaded at run-time.
The number of times the program is run and the command-line arguments to be supplied to
the program in each run is specified by properties \verb+chord.run.ids+ and
\verb+chord.args.XXX+ for each run ID \verb+XXX+.  By default, the program is run only once, using run ID \verb+0+,
and without any command-line arguments.
Only classes loaded in some run are regarded as reachable but {\it all} methods of each loaded class are regarded
as reachable regardless of whether they were invoked in the run.
The rationale behind this decision is to both reduce run-time overhead of JVMTI and to increase the
predictive power of program analyses performed using the computed analysis scope.
\end{itemize}
\end{itemize}

The default value of property \verb+chord.reuse.scope+ is \verb+false+ and that of
property \verb+chord.scope.kind+ is \verb+rta+.
The default values of properties \verb+chord.classes.file+ and \verb+chord.methods.file+ are
\verb+${chord.out.dir}/classes.txt+ and \verb+${chord.out.dir}/methods.txt+, respectively,
where property \verb+chord.out.dir+ defaults to \verb+${chord.work.dir}/chord_output/+,
and property \verb+chord.work.dir+ defaults to the current working directory.

