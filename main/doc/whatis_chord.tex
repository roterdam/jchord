\section{What is Chord?}
\label{sec:whatis-chord}

Chord is a program analysis platform for Java designed with the
primary goal of helping advance the state-of-the-art in the field by
enabling researchers to productively design, implement, and evaluate a
broad variety of program analyses. To this end, Chord has the
following key characteristics:

\begin{itemize}
\item
Stand-alone: it provides various off-the-shelf analyses (e.g., various
call-graph analyses; points-to analyses; thread-escape analysis;
static slicing; static and dynamic concurrency analyses for finding
races, deadlocks, and atomicity violations; etc.)
\item
Extensible: it allows users to build their own analyses atop the
provided analyses.
\item
Compositional: it allows users to write each analysis independently
yet allow it to interact in complex ways with other analyses by
specifying lightweight data- and control-dependencies.
\item
Efficient: It executes analyses in a demand-driven fashion, caches
results computed by each analysis for reuse by other analyses without
re-computation, and can execute analyses without dependencies between
them in parallel.
\item
Deterministic: It guarantees that the result is the same regardless of
the order in which different analyses are executed; moreover, results
can be shared across different runs.
\item
Flexible: It allows a broad range of analyses to be expressed,
including both static and dynamic analyses, analyses written
imperatively in Java or declaratively in Datalog, summary-based as
well as cloning-based inter-procedural context-sensitive analyses,
iterative refinement-based analyses, client-driven analyses, and
combined static/dynamic analyses.
\end{itemize}

\noindent Chord is intended to work on a variety of platforms,
including Linux, Windows/Cygwin, and MacOS.  It has been tested most
extensively on Linux.  It is open source software distributed under
the \xlink{New BSD License}{http://www.opensource.org/licenses/bsd-license.php}.
Improvements by users are welcome and encouraged.  The project website
is \url{http://berkeley.intel-research.net/chord/} and the software
website is \url{http://code.google.com/p/jchord/}.


