\section{What is Chord?}
\label{sec:whatis-chord}

\Chord\ is a static and dynamic program analysis framework for Java.
It has the following key characteristics:

\begin{itemize}
\item
Stand-alone: various off-the-shelf program analyses are provided (e.g., may-alias, thread-escape, datarace, deadlock, etc.).
\item
Extensible: users can build their own program analyses on top of the provided ones.
\item
Compositional: each program analysis can be written independently and yet made to interact in complex ways with other program analyses.
\item
Efficient: results computed by each program analysis are cached for reuse by other program analyses without re-computation.
\item
Flexible: a broad range of program analyses can be expressed, including those
written imperatively in Java or declaratively in Datalog, summary-based as well as cloning-based context-sensitive analyses,
iterative refinement-based analyses, client-driven analyses, and combined static and dynamic analyses.
\end{itemize}

\noindent \Chord\ is intended for use by a broad range of users:

\begin{itemize}
\item
Program analysis writers: Program analysis researchers wanting to implement and evaluate new program analysis algorithms.
\item
Program analysis appliers: Researchers with possibly a limited program analysis background wanting to build applications on
top of program analyses in Chord used as black boxes.
\item
Software engineers: Programmers wanting to use program analyses in Chord to assist with software development and testing.
\end{itemize}

\noindent \Chord\ is intended to work on a variety of platforms, including Linux, Windows/Cygwin, and MacOS.
It has been tested most extensively on Linux.
It is open source software distributed under the \htmladdnormallink{New BSD License}{http://www.opensource.org/licenses/bsd-license.php}.
Improvements from users are welcome and encouraged.
The project website is located at \url{http://code.google.com/p/jchord/}.

