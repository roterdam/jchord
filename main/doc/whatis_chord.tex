\chapter{What is Chord?}
\label{chap:whatis-chord}

Chord is a program analysis platform that enables users to
productively design, implement, evaluate, and combine a
broad variety of program analyses for Java bytecode. It has the
following key features:

\begin{itemize}
\item

{\bf Stand-alone:} It provides various off-the-shelf analyses (e.g., various
may-alias and call-graph analyses; thread-escape analysis;
static slicing analysis; static and dynamic concurrency analyses for finding
races, deadlocks, and atomicity violations; etc.)

\item

{\bf Flexible:} It allows users to express a broad range of analyses,
including both static and dynamic analyses, analyses written
imperatively in Java or declaratively in Datalog, summary-based as
well as cloning-based inter-procedural context-sensitive analyses,
iterative refinement-based analyses, client-driven analyses, and
combined static/dynamic analyses.

\item

{\bf Efficient:} It executes analyses in a demand-driven fashion, caches
results computed by each analysis for reuse by other analyses without
re-computation, and can execute analyses without dependencies between
them in parallel.

\item

{\bf Deterministic:} It guarantees that the result is the same regardless of
the order in which different analyses are executed; moreover, results
can be shared across different runs.
\end{itemize}

Chord is intended to work on a variety of platforms,
including Linux, Windows/Cygwin, and MacOS.  It has been tested most
extensively on Linux.  It is open source software distributed under
the \xlink{New BSD License}{http://www.opensource.org/licenses/bsd-license.php}.
Improvements by users are welcome and encouraged.  The project website
is \url{http://chord.stanford.edu/} and the software development website is
\url{http://jchord.googlecode.com/}.


