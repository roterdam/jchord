\xname{project}
\chapter{A Chord Project: Tasks, Targets, and Dependencies}
\label{chap:project}

In order to facilitate heavy reuse and rapid prototyping, each analysis in Chord is
written modularly, independent of other analyses, along with lightweight annotations
specifying the inputs and outputs of the analysis.
In each run, upon startup, Chord organizes all analyses and their inputs and outputs
(collectively called analysis results) using a global entity
called a {\it project}.  More concretely, a project consists of a set of
analyses called {\it tasks}, a set of analysis results called {\it targets}, and
a set of data/control dependencies between tasks and targets.

The project built in a particular run is of either of the following two kinds,
depending upon whether the value of property \code{chord.classic} is true or false, respectively.
\begin{itemize}
\item
a {\it classic project}, represented as an object of
class \javadoc{chord.project.ClassicProject}{chord/project/ClassicProject.html}.
\item
a {\it modern project}, represented as an object of
class \javadoc{chord.project.ModernProject}{chord/project/ModernProject.html}.
\end{itemize}
The project representation can be obtained by calling static method \code{g()} of the
corresponding class.
A classic project is built by default.
The two kinds of projects differ
primarily in that the only kind of dependencies in a classic
project are data dependencies whereas both data and control dependencies are
allowed in a modern project.  The key advantage of a modern project is that it
can schedule independent tasks in parallel whereas a classic project always
runs tasks sequentially.  This chapter focusses on classic projects as the
runtime for modern projects is still under development.
We next explain how Chord builds a classic project
(a set of tasks, a set of targets, and a set of dependencies between them).

{\bf Tasks:}
There are two kinds of tasks corresponding to the two broad kinds of analyses in Chord.
They are summarized in the following table:

\begin{mytable}{|l||c|c|}
\hline
 Kind: & imperative (see Chapter \ref{chap:writing-analysis}) & declarative (see Chapter \ref{chap:datalog-analysis}) \\
\hline
 Location: &
	\begin{tabular}{c}
	a {\tt .class} file in the path denoted by property \\
	\code{chord.java.analysis.path} compiled \\
	from a \code{@Chord}-annotated class implementing \\
	interface \javadoc{chord.project.ITask}{chord/project/ITask.html} 
	\end{tabular} &
	\begin{tabular}{c}
	a {\tt .dlog} file in the path denoted by property \\
	\code{chord.dlog.analysis.path}
	\end{tabular} \\
\hline
 Name: & via stmt \verb+name="<NAME>"+ in {\tt @Chord} annotation & via line ``\verb+# name=<NAME>+'' in {\tt .dlog} file \\
\hline
 Form: &
	an instance of the \code{@Chord}-annotated class &
	\begin{tabular}{c}
	an instance of class \\
    \javadoc{chord.project.analyses.DlogAnalysis}{chord/project/analyses/DlogAnalysis.html}
	\end{tabular}
\T \\
\hline
\end{mytable}

Each task in Chord is of the form ``\code{\{ C1, ..., Cn \} T \{ P1, ..., Pm \}}" where:
\begin{itemize}
\item
{\tt T} is the code provided by the user to be executed when the task is executed,
\item
{\tt C1}, ..., {\tt Cn} are the names of zero or more targets specified by the user as being
consumed by the task, and
\item
{\tt P1}, ..., {\tt Pm} are the names of zero or more targets specified by the user as being
produced by the task.
\end{itemize}
The consumed targets may be produced by other tasks and, likewise, the produced
targets may be consumed by other tasks.

{\bf Targets:}
The set of targets in a project includes each target that is specified as
consumed/produced by some task in the project.  When defining tasks, the user implicitly or
explicitly provides the class (type) of each target.
Chord reports a runtime error if a target has no type or has multiple types.
Otherwise, it creates a separate instance of that class to represent that target.

{\bf Dependencies:}
Chord computes a dependency graph as a directed graph whose
nodes are all tasks and targets computed as above, and:
\begin{itemize}
\item
There is an edge from a target C to a task T if the user has specified that T consumes C.
\item
There is an edge from a task T to a target P is the user has specified that T produces P.
\end{itemize}

We next present an example project to illustrate various concepts in the rest of
this chapter:

\begin{framed}
\begin{verbatim}
{} T1 { R1 }
{} T2 { R2 }
{ R4} T3 { R2 }
{ R1, R2 } T4 { R3, R4 }
\end{verbatim}
\end{framed} 

The set of tasks in this project is \{ {\tt T1}, {\tt T2}, {\tt T3}, {\tt T4} \}
and the set of targets in the project is \{ {\tt R1}, {\tt R2}, {\tt R3}, {\tt R4} \}.
The dependency graph is as follows:

\begin{center}
\includeimage{0.3}{dependency_graph.png}
\end{center}

Class \javadoc{chord.project.ClassicProject}{chord/project/ClassicProject.html}
provides a rich API (in the form of public instance methods) for accessing tasks
and targets in the project, for running tasks, and for resetting tasks and
targets.  The most commonly used methods are as follows:

\begin{mytable}{|l|l|}
\hline
\verb+ITask getTask(String name)+ & Representation of the task named {\tt name}. \\
\hline
\verb+Object getTrgt(String name)+ & Representation of the target named {\tt name}. \\
\hline
\verb+ITask runTask(String name)+ & Execute the task named {\tt name}. \\
\hline
\verb+boolean isTaskDone(String name)+ & Whether task named {\tt name} has alread been executed. \\
\hline
\verb+boolean isTrgtDone(String name)+ & Whether target named {\tt name} has already been computed. \\
\hline
\verb+void setTaskDone(String name)+ & Force task named {\tt name} to not be executed
the next time it is demanded. \\
\hline
\verb+void setTrgtDone(String name)+ & Force target named {\tt name} to
not be computed the next time it is demanded.  \\
\hline
\verb+void resetTaskDone(String name)+ & Force task named {\tt name} to be executed
the next time it is demanded. \\
\hline
\verb+void resetTrgtDone(String name)+ & Force target named {\tt name} to
be computed the next time it is demanded.  \T \\
\hline
\end{mytable}

We next explain the above methods.

The \code{getTask(name)} provides the representation
of the unique task with the specified name, if it exists,
and a runtime error otherwise.

The \code{getTrgt(name)} provides the representation
of the unique target with the specified name, if it exists,
and a runtime error otherwise.

A ``done'' bit, initialized to false, is kept with each task and each target in the project. 
The \code{runTask(name)} method runs the task with the specified name, if it exists,
and reports a runtime error otherwise.  Running a task proceeds as follows.
If the done bit of the task is true, no action is taken.  Otherwise, 
suppose the task is of the form ``\code{\{ C1, ..., Cn \} T \{ P1, ..., Pm \}}''.  Then,
the following two actions are taken in order:

\begin{enumerate}
\item
For each of the consumed targets {\tt C1}, ..., {\tt Cn} 
whose done bit is false, the unique task in the project producing that target 
is run recursively.
A runtime error is reported if no such task exists or if multiple such tasks exist.
\item
Once all consumed targets are done, the code {\tt T} of this task itself
is run.
\item
Finally, the done bit of this task as well as of each of its produced tasks
{\tt P1}, ..., {\tt Pn} is set to true.
\end{enumerate}

It is the user's responsibility to ensure termination in the case in which there are
cycles in the dependency graph.  The \code{isTaskDone(name)} and \code{isTrgtDone(name)}
methods can be used in the code {\tt T} of any task to enquire
whether the ``done'' bit of the task or target with the specified name is set to true.
Moreover, methods \code{setTaskDone(name)}, \code{setTrgtDone(name)} can be used to set
the done bit of the task or target with the specified name to true, and likewise,
methods \code{resetTaskDone(name)} and \code{resetTrgtDone(name)} can be used
to set the done bit to false.

It is possible to run tasks from the command-line of Chord by specifying the value of property
\code{chord.run.analyses} as a comma-separated list of the names of tasks to be run in order
(see Chapter \ref{chap:running-analysis}).

We next illustrate the above concepts using the above example.
Suppose Chord is run with the value of property \code{chord.run.analyses} as ``{\tt T4}''.
This causes \code{runTask(T4)} to be called.  The done bit of task {\tt T4} is initialized to false.
Hence, the done bit of its first consumed target {\tt R1} is checked.
Since it is also initialized to false, the unique task producing
target {\tt R1} is demanded.  However, multiple tasks {\tt T1} and {\tt T2} 
producing target {\tt R1} are found in the project, resulting in a runtime error which
reports the ambiguity between tasks {\tt T1} and {\tt T2}.

To resolve the ambiguity (say in favor of task {\tt T1}), the user can
specify the value of property \code{chord.run.analyses} as ``{\tt T1},{\tt T4}''.  This
time, \code{runTask(T1)} is called followed by \code{runTask(T4)}.
Since the done bit of task {\tt T1} is initialized to false and it has no consumed
targets, \code{runTask(T1)} simply executes the code of task {\tt T1}, and sets
the done bit of task {\tt T1} and of its only produced target {\tt R1} to true.
Next, the call to \code{runTask(T4)} proceeds as described in the previous run
above, but this time the done bit of target {\tt R1} consumed by task {\tt T4} is
set to true.  Hence, the demand for the unique task producing target {\tt R1} (and
the ensuing ambiguity runtime error) is averted.  However, this time a different
problem occurs: the done bit of the other target {\tt R2} consumed by task {\tt T4} is
initialized to false, which results in a call to \code{runTask(T3)}
(since task {\tt T3} is the unique task that produces target {\tt R2}), which in turn
results in a call to \code{runTask(T4)}.  The result is infinite 
mutually-recursive calls to \code{runTask(T4)} and \code{runTask(T3)} unless the
code of task {\tt T3} or {\tt T4} averts it by
calling \code{setTaskDone} or \code{setTrgtDone} on some task or target
in the cycle.
This scenario resulting from a cycle in the dependence graph is rare in practice.
It typically occurs in the case of iterative refinement-based client-driven analyses:
the output of such an analysis in one iteration is fed as an input to the same analysis
in a subsequent iteration.  The code of such an analysis (task) must explicitly
control execution as described above to avert infinite recursion.

