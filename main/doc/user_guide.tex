\documentclass[openany]{book}
\usepackage{hyperlatex}
\usepackage{hyperref}
\texonly{
	\usepackage{graphicx}
	\usepackage{framed}
}

\texonly{
% put borders around figures
	%\usepackage{float}
	%\floatstyle{boxed} 
	%\restylefloat{figure}
% set dimensions of page/text
	\parindent=0in
	\parskip=10pt
	\oddsidemargin=-0.3in
	\evensidemargin=-0.3in
	\textwidth=7in
	\textheight=8.0in
% redefine certain commands
	\renewcommand\code[1]{\path{#1}}
	\renewcommand\xlink[2]{\href{#2}{#1}}
}

\htmlonly{
	\newcommand\code[1]{{\tt #1}}
	\newcommand\url[1]{\xlink{#1}{#1}}
	\newenvironment{framed}
		{\xmlattributes*{table}{border=1 frame=box rules=none cellpadding=8 bgcolor=CCFFFF width=100%
		}\begin{tabular}{l}}{\end{tabular}}
}

\newcommand\javadoc[2]{\xlink{\code{#1}}{http://chord.stanford.edu/javadoc/#2}}

\newenvironment{mytable}[1]{\htmlattributes*{table}{border=1 cellpadding=10}\begin{tabular}{#1}}{\end{tabular}}

\newcommand\includeimage[2]{\texonly{\includegraphics[scale=#1]{#2}}\htmlonly{{\htmlimg{#2}}}}

\newcommand\bh{{\bf h}}
\newcommand\bi{{\bf i}}
\newcommand\bo{{\bf o}}
\newcommand\bt{{\bf t}}
\newcommand\bm{{\bf m}}
\newcommand\be{{\bf e}}
\newcommand\bg{{\bf f}}
\newcommand\bb{{\bf b}}
\newcommand\bl{{\bf l}}
\newcommand\bp{{\bf p}}
\newcommand\br{{\bf r}}
\newcommand\bw{{\bf w}}

\newcommand\bsdlicense{\xlink{New BSD License}{http://www.opensource.org/licenses/bsd-license.php}}
\newcommand\chordweb{\url{http://jchord.googlecode.com/}}

\setcounter{htmldepth}{3}
\setcounter{htmlautomenu}{4}

\htmltitle{Chord}
\htmladdress{mhn@cs.stanford.edu}

\title{Chord: A Versatile Platform for Program Analysis}
\author{Mayur Naik}
\date{\today}
\begin{document}
\maketitle

%\texonly{\tableofcontents}

\xname{preface}
\part{Preface}

\xname{organization}
\chapter{Organization of this Guide}
\label{chap:organization}

Chord is a program analysis platform that is both {\it stand-alone}, in that it
provides many standard analyses for users to run, and {\it extensible}, in that
it allows users to write and run their own analyses, possibly atop the provided
analyses.  As a result, Chord has two kinds of users: {\it end-users}, who only
wish to run predefined analyses, and {\it developers}, who additionally wish to
write and run their own analyses.

For convenience, this user guide consists of two parts: a
\xlink{guide for end-users}{enduser\_guide.html} and a
\xlink{guide for developers}{developer\_guide.html}.  Unlike end-users,
developers need to understand Chord's source code and API, and code written by
them is executed as part of a Chord run.  Hence, the guide for end-users
concerns how to run Chord, and the guide for developers concerns how to extend
Chord.


\xname{acks}
\chapter{Acknowledgments}
\label{chap:acks}

Chord would not be possible without the following open-source software:

\begin{itemize}
\item
\joeq, a Java compiler framework
\item
\javassist, a Java bytecode manipulation framework
\item
\bddbddb, a BDD-based Datalog solver
\end{itemize}

Chord additionally relies on the following open-source tools and libraries:
\begin{itemize}
\item
\xlink{Ant-Contrib}{http://ant-contrib.sourceforge.net/}, a collection of useful Ant tasks  
\item
\buddy, a BDD library
\item
\xlink{GNU Trove}{http://trove4j.sourceforge.net/}, a primitive collections library for Java
\item
\xlink{Java2HTML}{http://www.java2html.com/} and
\xlink{Java2Html}{http://www.java2html.de/}, Java-to-HTML tools
\item
\xlink{Saxon}{http://saxon.sourceforge.net/}, an XSLT processor
\end{itemize}

Chord was supported in part by grants from the National Science Foundation,
an equipment grant from Intel, and a Microsoft fellowship during 2005-2007.



\xname{enduser_guide}
\part{Guide for End-Users}

\xname{whatis}
\chapter{What is Chord?}
\label{chap:whatis}

Chord is a program analysis platform that enables users to productively design,
implement, evaluate, and combine a broad variety of static and dynamic program
analyses for Java bytecode. It has the following key features:

\begin{itemize}
\item

It provides various off-the-shelf analyses (e.g., various may-alias and
call-graph analyses; thread-escape analysis; static slicing analysis; static and
dynamic concurrency analyses for finding races, deadlocks, and atomicity
violations; etc.)

\item

It allows users to express a broad range of analyses, including both static and
dynamic analyses, analyses written imperatively in Java or declaratively in
Datalog, summary-based as well as cloning-based inter-procedural
context-sensitive analyses, iterative refinement-based analyses, client-driven
analyses, and combined static/dynamic analyses.

\item

It executes analyses in a demand-driven fashion, caches results computed by each
analysis for reuse by other analyses without re-computation, and can execute
analyses without dependencies between them in parallel.

\item

It guarantees that the result is the same regardless of the order in which
different analyses are executed; moreover, results can be shared across
different runs.
\end{itemize}

Chord is intended to work on a variety of platforms, including Linux,
Windows/Cygwin, and MacOS.  It is open-source software distributed under
the \bsdlicense.  Improvements by users are welcome and encouraged.  The project
website is \chordweb.

\xname{start}
\chapter{Getting Started}
\label{chap:start}

This chapter describes how to download, install, and run Chord.  Section
\ref{sec:downloading-binaries} describes how to obtain pre-built binaries of
Chord.  Section \ref{sec:downloading-sources} describes how to obtain the source
code of Chord and Section \ref{sec:downloading-sources} explains how to build
it.  Finally, Section \ref{sec:running-chord} describes how to run Chord.

\section{Downloading Binaries}
\label{sec:downloading-binaries}

To obtain Chord's pre-built binaries, download and uncompress file
\chordbinfile.  It includes the following files:

\begin{enumerate}
\item
\code{chord.jar}, which contains the class files of Chord and of libraries used
by Chord.
\item
\code{libbuddy.so}, \code{buddy.dll}, and \code{libbuddy.dylib}: you can keep
one of these files depending upon whether you intend to run Chord on Linux,
Windows/Cygwin, or MacOS, respectively.  These files are needed only if you want
BDD library \buddy\ to be used when the BDD-based Datalog solver \bddbddb\ in
Chord runs analyses written in Datalog.
\item
\code{libchord_instr_agent.so}: this file is needed only if you want the
JVMTI-based bytecode instrumentation agent to be used when Chord runs dynamic
analyses.
\end{enumerate}

Novice users can ignore items (2) and (3) until they become more familiar with
Chord.  The binaries mentioned in items (2) and (3) might not be compatible with
your machine, in which case you can either forgo using them (with hardly any
noticeable difference in functionality), or you can download the sources (see
Section \ref{sec:downloading-sources}) and build them yourself (see Section
\ref{sec:compiling-sources}).

\section{Downloading Source Code}
\label{sec:downloading-sources}

To obtain Chord's source code, download and uncompress the following files:

\begin{itemize}
\item
Mandatory: file
\chordsrcfile, which contains Chord's source code and jars of libraries used by
Chord.
\item
Optional: file \chordlibfile, which contains the source code of libraries used
by Chord (e.g., joeq, javassist, bddbddb, etc.)
\end{itemize}

Alternatively, you can obtain the latest development snapshot from the SVN
repository by running the following command:

\begin{framed}
\begin{verbatim}
svn checkout http://jchord.googlecode.com/svn/trunk/ chord
\end{verbatim}
\end{framed}

Instead of checking out the entire \code{trunk/}, which contains several
sub-directories, you can check out specific sub-directories:

\begin{itemize}
\item
\code{main/} contains Chord's source code and jars of libraries used by Chord.
\item
\code{libsrc/} contains the source code of libraries used by Chord (e.g., joeq,
javassist, bddbddb, etc.).
\item
\code{test/} contains Chord's regression tests.
\item
many more; these might eventually move under \code{main/}.
\end{itemize}

Files \code{chord-2.0-src.tar.gz} and \code{chord-2.0-libsrc.tar.gz} mentioned
above are essentially stable releases of the \code{main/} and \code{libsrc/}
directories, respectively.

\section{Compiling the Source Code}
\label{sec:compiling-sources}

Compiling Chord's source code requires the following software:

\begin{itemize}
\item
JVM supporting Java 5 or higher, e.g. \ibmjvm\ or \sunjvm.
\item
\ant, a Java build tool.
\end{itemize}

Chord's main directory contains a file named {\tt build.xml} which is
interpreted by Apache Ant.  To see the various possible targets, simply run
command ``{\tt ant}" in that directory.

To compile Chord, run command ``{\tt ant compile}" in the same directory.  This
will compile Chord's Java sources from \code{src/} to class files in
\code{classes/}, as well as build a jar file \code{chord.jar} that contains
these class files as well as those in the jars of libraries that are used by
Chord and are provided under \code{lib/} (e.g., \code{joeq.jar},
\code{javassist.jar}, \code{bddbddb.jar}, etc.).  Additionally:

\begin{itemize}
\item

If system property \code{chord.use.buddy} is set to \code{true}, then the C
source code of BDD library \buddy\ from directory \code{bdd/} will be compiled
to a shared library named (\code{libbuddy.so} on Linux, \code{buddy.dll} on
Windows, and \code{libbuddy.dylib} on MacOS; this library is used by BDD-based
Datalog solver \bddbddb\ in Chord for running analyses written in Datalog.

\item

If system property \code{chord.use.jvmti} is set to \code{true}, then the C++
source code of the JVMTI-based bytecode instrumentation agent from directory
\code{agent/} will be compiled to a shared library named
\code{libchord_instr_agent.so} on all architectures; this agent is used in Chord
for computing analysis scope dynamically and for running dynamic analyses.
\end{itemize}

Properties \code{chord.use.buddy} and \code{chord.use.jvmti} are defined in a
file named \code{chord.properties} in Chord's main directory.  The default value
of both these properties is \code{false}.  If you set either of them to
\code{true}, then you will also need a utility like GNU Make (to run the
\code{Makefile}'s in directories \code{bdd/} and \code{agent/}) and a C++
compiler (to build the above shared libraries).

\section{Running Chord}
\label{sec:running-chord}

Running Chord requires a JVM supporting Java 5 or higher.  There are two
equivalent commands to run Chord.

One command, which is available in the source and binary installations of
Chord, is:

\begin{framed}
\begin{verbatim}
java -cp <CHORD_MAIN_DIR>/chord.jar -D<key1>=<val1> ... -D<keyN>=<valN> chord.project.Boot
\end{verbatim}
\end{framed}

where \code{<CHORD_MAIN_DIR>} denotes the directory containing file
\code{chord.jar}; that directory is also expected to contain any shared
libraries in Chord's installation (e.g., \code{libbuddy.so} and
\code{libchord_instr_agent.so}).

The alternate command, which is available only in the source installation of
Chord, is:

\begin{framed}
\begin{verbatim}
ant -f <CHORD_MAIN_DIR>/build.xml -D<key1>=<val1> ... -D<keyN>=<valN> run
\end{verbatim}
\end{framed}

This command requires \ant\ (a Java build tool) to be installed on your
machine.  This command is used throughout this guide.  Also, the
``\code{-f <CHORD_MAIN_DIR>/build.xml}" argument in the command is omitted
for brevity.

Each ``\code{-D<key>=<val>}" argument in either of the above commands sets the
system property named \code{<key>} to the value denoted by \code{<val>}.  The
only way to specify inputs to Chord is via system properties; there is no
command-line argument processing.  Chapter \ref{chap:properties} describes all
system properties recognized by Chord.


\chapter{Chord Properties}
\label{chap:properties}

The only way to specify inputs to Chord is by means of system properties.
There is no command-line argument processing in Chord and any command-line arguments are ignored.
Section \ref{sec:properties-setting} explains how to set properties and
Section \ref{sec:properties-meaning} explains the meaning of properties recognized by Chord.

\section{How to Set Properties}
\label{sec:properties-setting}

A property can be passed to Chord in any of several ways.
The reason for providing multiple ways is to provide users shorthand ways for defining properties
once and for all for a particular input Java program, or even once and for all across all Chord runs. 
The following are the different ways by which a property can be passed to Chord in decreasing order of precedence:

\begin{enumerate}
\item

On the command-line via the \code{-D<key>=<value>} format.

Use this option to specify properties specific to the current run of Chord.

Typical usage of this option is by running the following command in Chord's \code{main/} directory:
\begin{verbatim}
    prompt> ant -D<key1>=<value1> ... -D<keyN>=<valueN> run
\end{verbatim}

\item

Via the file specified by property \code{chord.props.file}.

Use this option to specify once and for all properties of the program to
be analyzed (e.g., property \code{chord.main.class} specifying the name of that program's main class, property \code{chord.class.path}
specifying the application classpath of that program, etc.).

The default value of this property is \code{[chord.work.dir]/chord.properties} and typically
does not need to be changed by users.
Property \code{chord.work.dir} specifies the directory in which Chord will run.
The default value of this property is the current directory but users must typically set
it on the command line, as follows:

\begin{verbatim}
    prompt> ant -Dchord.work.dir=<...> run
\end{verbatim}
where \code{<...>} denotes the absolute or relative path of the top-level directory of
the program to be analyzed, which contains a file named \code{chord.properties} that users
must create themselves and populate with properties that are recognized by chord and
are specific to that program.

\item

Via file \code{main/chord.properties}.

Use this option to specify once and for all properties you would like to hold in every run of Chord
(e.g., property \code{chord.max.heap} specifying the maximum heap memory size to be used by the
JVM running Chord).
\end{enumerate}

\section{Meaning of Properties}
\label{sec:properties-meaning}

The following properties are recognized by Chord.
The separator for list-valued properties can be either a blank space, a comma, a colon, or a semi-colon.
Notation {\tt [<...>]} is used in this section to denote the value of the property named {\tt <...>}.

\subsection{Java Program Properties} 
\label{sec:program-props}

This section describes properties of the Java program to be analyzed, such as
its main class, the location(s) of its class files and Java source
files, and command-line arguments to be used when running the program.

\code{chord.work.dir}
\begin{quote}
{\bf Type:} location \\
{\bf Description:} Working directory during Chord's execution.  This is
usually the top-level directory of the input Java program. \\
{\bf Default value:} current working directory
\end{quote}

\code{chord.props.file}
\begin{quote}
{\bf Type:} location \\
{\bf Description:} Properties file loaded by Chord at the
beginning before doing anything else.  Any of the below properties may
be defined in this file to avoid defining them on the command line
(using the \code{-D<key>=<value>} format) every time Chord is run.
Each relative file/directory name in the value of any property defined
in this file is treated relative to Chord's working directory (which
is specified by property \code{chord.work.dir}). \\
{\bf Default value:} \code{[chord.work.dir]/chord.properties}
\end{quote}

\code{chord.main.class}
\begin{quote}
{\bf Type:} class \\
{\bf Description:} Fully-qualified name of the main class of the input Java program (e.g., \code{com.example.Main}).
\end{quote}

\code{chord.class.path}
\begin{quote}
{\bf Type:} path \\
{\bf Description:} Classpath of the input Java program.  It does not need to include
boot classes (i.e., classes in \code{[sun.boot.class.path]}) or
standard extensions (i.e., classes in jar files in directory \code{[java.home]/lib/ext/}). \\
{\bf Default value:} {\tt ""}
\end{quote}

\code{chord.src.path}
\begin{quote}
{\bf Type:} path \\
{\bf Description:} Source path of the input Java program. \\
{\bf Default value:} {\tt ""} \\
{\bf Note:} Chord analyzes only Java bytecode, not Java source code.  This property is used only by the task of converting Java source files into HTML files by analyses that need to present their results at the Java source code level (by calling method \code{chord.program.Program.g().HTMLizeJavaSrcFiles())}.
\end{quote}

\code{chord.run.ids}
\begin{quote}
{\bf Type:} string list \\
{\bf Description:} List of IDs to identify runs of the input Java program. \\
{\bf Default value:} {\tt 0} \\
{\bf Note:} This property is used only when Chord runs the input Java program, namely, when it is asked to compute the analysis scope dynamically (i.e., when \code{[chord.scope.kind]=dynamic}) or when it is asked to run a dynamic analysis. 
\end{quote}

\code{chord.args.<id>}
\begin{quote}
{\bf Type:} string \\
{\bf Description:} Command-line arguments string to be used for the input Java program in the run having ID {\tt <id>}. \\
{\bf Default value:} {\tt ""} \\
{\bf Note:} This property is used only when Chord runs the input Java program, namely, when it is asked to compute the analysis scope dynamically (i.e., when \code{[chord.scope.kind]=dynamic}) or when it is asked to run a dynamic analysis.
\end{quote}

\code{chord.runtime.jvmargs}
\begin{quote}
{\bf Type:} string \\
{\bf Description:} Arguments to JVM which runs the input Java program. \\
{\bf Default value:} {\tt "-ea -Xmx1024m"} \\
{\bf Note:} This property is used only when Chord runs the input Java program, namely, when it is asked to compute the analysis scope dynamically (i.e., when \code{[chord.scope.kind]=dynamic}) or when it is asked to run a dynamic analysis. 
\end{quote}

\subsection{Analysis Scope Properties}
\label{sec:scope-props}

This section describes properties that specify how the analysis scope of the input Java program is computed.
See Chapter \ref{chap:computing-scope} for more details.

\code{chord.scope.kind}
\begin{quote}
{\bf Type:} {\tt [dynamic|rta|cha]} \\
{\bf Description:} Algorithm to compute analysis scope.  The choices are {\tt dynamic} (dynamic analysis), {\tt rta} (Rapid Type Analysis), and {\tt cha} (Class Hierarchy Analysis). \\
{\bf Default value:} {\tt rta} \\
{\bf Note:} This property is ignored if property \code{chord.reuse.scope} is set to {\tt true} and the files specified by properties \code{chord.methods.file} and \code{chord.reflect.file} exist. 
\end{quote}

\code{chord.reflect.kind}
\begin{quote}
{\bf Type:} {\tt [none|dynamic|static|static\_cast]} \\
{\bf Description:} Algorithm to resolve reflection.  The choices are {\tt none} (do not resolve any reflection),
{\tt dynamic} (run the program and observe how reflection is resolved),
{\tt static} (resolve reflection statically but without analyzing casts), and
{\tt static\_cast} (resolve reflection statically by analyzing casts). \\
{\bf Default value:} {\tt none}
\end{quote}

\code{chord.ch.kind}
\begin{quote}
{\bf Type:} {\tt [static|dynamic]} \\
{\bf Description:} Algorithm to build the class hierarchy.  If it is {\tt dynamic}, then the input Java program is executed
and classes not loaded by the JVM while running the program are excluded while building the class hierarchy. \\
{\bf Default value:} {\tt static} \\
{\bf Note:} This property is relevant only if \code{chord.scope.kind} is {\tt cha} since only this
scope computing algorithm queries the class hierarchy. 
\end{quote}

\code{chord.ssa}
\begin{quote}
{\bf Type:} bool  \\
{\bf Description:} Do SSA (Static Single Assignment) transformation of the bodies of all methods deemed reachable by the algorithm used to compute analysis scope. \\
{\bf Default value:} {\tt true}
\end{quote}

% TODO: mention that below two properties are also used by instrumentor to decide which classes to exclude from instrumentation

\code{chord.std.scope.exclude}
\begin{quote}
{\bf Type:} string list \\
{\bf Description:} Partial list of prefixes of names of classes, typically inside the JDK standard library, whose methods must be treated as no-ops. \\
{\bf Default value:} {\tt ""}
\end{quote}

\code{chord.ext.scope.exclude}
\begin{quote}
{\bf Type:} string list \\
{\bf Description:} Partial list of prefixes of names of classes, typically outside the JDK standard library, whose methods must be treated as no-ops. \\
{\bf Default value:} {\tt ""}
\end{quote}

\code{chord.scope.exclude}
\begin{quote}
{\bf Type:} string list \\
{\bf Description:} Complete list of prefixes of names of classes whose methods must be treated as no-ops. \\
{\bf Default value:} \code{"[chord.std.scope.exclude],[chord.ext.scope.exclude]"}
\end{quote}

\code{chord.std.check.exclude}
\begin{quote}
{\bf Type:} string list \\
{\bf Description:} Partial list of prefixes of names of classes, typically inside the JDK standard library, to be excluded by analyses.  Interpretation of this property is analysis-specific. \\
{\bf Default value:} \code{"java.,javax.,sun.,com.sun.,com.ibm.,} \code{org.apache.harmony."}
\end{quote}

\code{chord.ext.check.exclude}
\begin{quote}
{\bf Type:} string list \\
{\bf Description:} Partial list of prefixes of names of classes, typically outside the JDK standard library, to be excluded by analyses.  Interpretation of this property is analysis-specific. \\
{\bf Default value:} {\tt ""}
\end{quote}

\code{chord.check.exclude}
\begin{quote}
{\bf Type:} string list \\
{\bf Description:} Complete list of prefixes of names of classes to be excluded by analyses.  Interpretation of this property is analysis-specific. \\
{\bf Default value:} \code{"[chord.std.check.exclude],[chord.ext.check.exclude]"}
\end{quote}

\subsection{Functionality Properties}
\label{sec:func-props}

This section describes properties that dictate what task(s) Chord must perform.

\code{chord.build.scope}
\begin{quote}
{\bf Type:} bool \\
{\bf Description:} Compute the analysis scope of the input Java program using the algorithm. \\
{\bf Default value:} {\tt false} \\
{\bf Note:} The analysis scope is computed regardless of the value of this property if another task (e.g., an analysis specified via property \code{chord.run.analyses}) demands it.
\end{quote}

\code{chord.run.analyses}
\begin{quote}
{\bf Type:} string list  \\
{\bf Description:} List of names of analyses to be run in order. \\
{\bf Default value:} {\tt ""} \\
{\bf Note:} If the analysis is written in Java, its name is specified via statement {\tt name=<...>} in its {\tt @Chord} annotation.  If the analysis is written in Datalog, its name is specified via a line of the form ``{\tt \# name=<...>}".
\end{quote}

\code{chord.print.methods}
\begin{quote}
{\bf Type:} string list  \\
{\bf Description:} List of methods whose intermediate representation to print to standard output. \\
{\bf Default value:} {\tt ""} \\
{\bf Note:} Specify each method in format \code{<mname>:<mdesc>@<cname>} where {\tt <mname>} is the method's name, {\tt <mdesc>} is the method's descriptor, and {\tt <cname>} is the name of the method's declaring class. In {\tt <cname>}, use `.' instead of `/', and use {\tt \#} instead of the dollar character. 
\end{quote}

\code{chord.print.classes}
\begin{quote}
{\bf Type:} string list  \\
{\bf Description:} List of classes whose intermediate representation to print to standard output. \\
{\bf Default value:} {\tt ""} \\
{\bf Note:} In class names, use `.' instead of `/', and use {\tt \#} instead of the dollar character. 
\end{quote}

\code{chord.print.all.classes}
\begin{quote}
{\bf Type:} bool \\
{\bf Description:} Print intermediate representation of all classes in scope to standard output. \\
{\bf Default value:} {\tt false}
\end{quote}

\code{chord.print.rels}
\begin{quote}
{\bf Type:} string list  \\
{\bf Description:} List of names of program relations whose contents must be printed to files \code{[chord.out.dir]/<...>.txt} where {\tt <...>} denotes the relation name. \\
{\bf Default value:} {\tt ""} \\
{\bf Note:} This functionality must be used with caution as certain program relations, albeit represented compactly as BDDs, may contain a large number (e.g., millions) of tuples, resulting in voluminous output when printed in explicit form to a text file.  See Section \ref{sec:tuning-datalog-analysis} for a more efficient way to query the contents of program relations (namely, by using the {\tt debug} target provided in file {\tt build.xml} in Chord's \code{main/} directory).
\end{quote}

\code{chord.print.project}
\begin{quote}
{\bf Type:} bool \\
{\bf Description:} Create files \code{targets_sortby_name.html}, \code{targets_sortby_kind.html}, and \code{targets_sortby_producers.html} in directory \code{[chord.out.dir]}, publishing all tasks and targets defined by analyses in paths \code{[chord.java.analysis.path]} and \code{[chord.dlog.analysis.path]}.  \\
{\bf Default value:} {\tt false}
\end{quote}

\code{chord.print.results}
\begin{quote}
{\bf Type:} bool \\
{\bf Description:} Print the results of analyses in HTML.  Interpretation of this property is analysis-specific.  \\
{\bf Default value:} {\tt true}
\end{quote}

\code{chord.verbose}
\begin{quote}
{\bf Type:} int in the range [0..5]  \\
{\bf Description:} Control the verbosity of messages during Chord's execution.  \\
{\bf Default value:} {\tt 1}
\end{quote}

\subsection{Project Properties}
\label{sec:project-props}

This section describes properties regarding analyses executed by Chord.

\code{chord.classic}
\begin{quote}
{\bf Type:} bool \\
{\bf Description:} Whether to use the classic project (as opposed to the modern project).  See Chapter\ref{chap:analyses} for the difference between the two kinds of projects. \\
{\bf Default value:} \code{true}
\end{quote}

\code{chord.std.java.analysis.path}
\begin{quote}
{\bf Type:} path \\
{\bf Description:} Partial classpath of analyses written in Java (i.e., {\tt @Chord}-annotated classes).
Conventionally, it includes all such analyses that are predefined in Chord.  \\
{\bf Default value:} The absolute path of file \code{chord.jar}
\end{quote}

\code{chord.ext.java.analysis.path}
\begin{quote}
{\bf Type:} path \\
{\bf Description:} Partial classpath of analyses written in Java (i.e., {\tt @Chord}-annotated classes).
Conventionally, it includes all such external analyses that users want to run. \\
{\bf Default value:} {\tt ""}
\end{quote}

\code{chord.java.analysis.path}
\begin{quote}
{\bf Type:} path \\
{\bf Description:} Complete classpath of analyses written in Java (i.e., {\tt @Chord}-annotated classes). \\
{\bf Default value:} \code{[chord.std.java.analysis.path]:[chord.ext.java.analysis.path]}
\end{quote}

\code{chord.std.dlog.analysis.path}
\begin{quote}
{\bf Type:} path \\
{\bf Description:} Partial classpath of analyses written in Datalog (i.e., files with suffix {\tt .dlog} or {\tt .datalog}).
Conventionally, it includes all such analyses that are predefined in Chord.  \\
{\bf Default value:} The absolute path of file \code{chord.jar}
\end{quote}

\code{chord.ext.dlog.analysis.path}
\begin{quote}
{\bf Type:} path \\
{\bf Description:} Partial classpath of analyses written in Datalog (i.e., files with suffix {\tt .dlog} or {\tt .datalog}).
Conventionally, it includes all such external analyses that users want to run. \\
{\bf Default value:} {\tt ""}
\end{quote}

\code{chord.dlog.analysis.path}
\begin{quote}
{\bf Type:} path  \\
{\bf Description:} Complete classpath of analyses written in Datalog (i.e., files with suffix {\tt .dlog} or {\tt .datalog}). \\
{\bf Default value:} \code{[chord.std.dlog.analysis.path]:[chord.ext.dlog.analysis.path]}
\end{quote}

\subsection{Instrumentation Properties}
\label{sec:instr-props}

This section describes properties regarding bytecode instrumentation and dynamic analysis.

% TODO: mention chord.scope.exclude* here
\code{chord.use.jvmti}
\begin{quote}
{\bf Type:} bool \\
{\bf Description:} Whether the JVMTI-based bytecode instrumentation agent from \code{main/agent/} must be used for running dynamic analyses. \\
{\bf Default value:} \code{false}
\end{quote}

\code{chord.instr.kind}
\begin{quote}
{\bf Type:} {\tt [offline|online]}  \\
{\bf Description:} The kind of bytecode instrumentation.  The choices are offline and online (load-time).  \\
{\bf Default value:} {\tt offline}
\end{quote}

\code{chord.trace.kind}
\begin{quote}
{\bf Type:} {\tt [full|pipe]}  \\
{\bf Description:} The medium by which an event-generating JVM and an event-handling JVM communicate in a dynamic analysis.  The choices are regular file and POSIX pipe.  \\
{\bf Default value:} {\tt full} 
\end{quote}

\code{chord.trace.block.size}
\begin{quote}
{\bf Type:} int \\
{\bf Description:} Number of bytes to read/write in a single operation from/to the event trace file in a multi-JVM dynamic analysis. \\
{\bf Default value:} {\tt 4096}
\end{quote}

\code{chord.dynamic.haltonerr}
\begin{quote}
{\bf Type:} bool \\
{\bf Description:} Whether to terminate Chord if the input Java program terminates abnormally during dynamic analysis. \\
{\bf Default value:} true
\end{quote}

\code{chord.dynamic.timeout}
\begin{quote}
{\bf Type:} int  \\
{\bf Description:} The amount of time, in milliseconds, after which to kill the process running the given program during dynamic analysis, or -1 if the process must never be killed. \\
{\bf Default value:} {\tt -1}
\end{quote}

\code{chord.max.cons.size}
\begin{quote}
{\bf Type:} int \\
{\bf Description:} Maximum number of bytes over which events generated during the execution of any constructor in the given program may span. \\
{\bf Default value:} {\tt 50000000} \\
{\bf Note:} This property is relevant only for dynamic analyses which want events of the form {\tt BEF\_NEW} $h$ $t$ $o$ to be generated (see Section \ref{sec:instr-events}).  The problem with generating such events at run-time is that the ID $o$ of the object freshly created by thread $t$ at object allocation site $h$ cannot be instrumented until the object is fully initialized (i.e., its constructor has finished executing).  Hence, Chord first generates a ``crude dynamic trace", which has events of the form {\tt BEF\_NEW} $h$ $t$ and {\tt AFT\_NEW} $h$ $t$ $o$ generated before and after the execution of the constructor, respectively.  A subsequent pass generates a ``final dynamic trace", which replaces each {\tt BEF\_NEW} $h$ $t$ event by {\tt BEF\_NEW} $h$ $t$ $o$.  For this purpose, however, Chord must buffer all events generated between the {\tt BEF\_NEW} and {\tt AFT\_NEW} events, and this property specifies the number of bytes over which these events may span.  If the actual number of bytes exceeds the value specified by this property (e.g., if the constructor throws an exception and the {\tt AFT\_NEW} event is not generated at all), then Chord simply generates event {\tt BEF\_NEW} $h$ $i$ $0$ (i.e., it treats the created object as having ID 0, which is the ID also used for {\tt null}). 
\end{quote}

\subsection{Caching Properties}
\label{sec:caching-props}

This section describes properties that specify what must be reused by Chord, if available, from previous runs instead of recomputing.

\code{chord.reuse.scope}
\begin{quote}
{\bf Type:} bool \\
{\bf Description:} Compute analysis scope using the information in files specified by properties \code{chord.methods.file} and \code{chord.reflect.file}, if both of those files exist. \\
{\bf Default value:} {\tt false} \\
{\bf Note:} Property \code{chord.scope.kind} is ignored if this property is set to {\tt true} and the two files exist. 
\end{quote}

\code{chord.reuse.rels}
\begin{quote}
{\bf Type:} bool  \\
{\bf Description:} Load each desired program relation named \code{<name>} from the BDD stored in file \code{[chord.bddbddb.work.dir]/<name>.bdd}, if the file exists. \\
{\bf Default value:} {\tt false}
\end{quote}

\code{chord.reuse.traces}
\begin{quote}
{\bf Type:} bool \\
{\bf Description:} Reuse event traces stored in file(s) \code{chord.trace.file]_full_ver0_runM.txt} for dynamic analysis, if those files exist,
where \code{M} ranges over run IDs specified by property \code{chord.run.ids}.
Property \code{chord.trace.kind} must be set to {\tt full} if this property is set to {\tt true}. \\
{\bf Default value:} {\tt false}
\end{quote}

\subsection{Chord JVM Properties}
\label{sec:jvm-props}

This section describes properties regarding the JVM that runs Chord.

\code{chord.max.heap}
\begin{quote}
{\bf Type:} string \\
{\bf Description:} Maximum heap memory size of the JVM running Chord. \\
{\bf Default value:} {\tt 1024m}
\end{quote}

\code{chord.max.stack}
\begin{quote}
{\bf Type:} string \\
{\bf Description:} Maximum thread stack size of the JVM running Chord. \\
{\bf Default value:} {\tt 32m}
\end{quote}

\code{chord.jvmargs}
\begin{quote}
{\bf Type:} string \\
{\bf Description:} Arguments to the JVM running Chord. \\
{\bf Default value:} \code{"-showversion} \code{-ea} \code{-Xmx[chord.max.heap]} \code{-Xss[chord.max.stack]"}
\end{quote}

\subsection{BDD Properties}
\label{sec:bdd-props}

This section describes properties concerning BDD-based Datalog solver bddbddb that is used by Chord to run analyses written in Datalog.

\code{chord.use.buddy}
\begin{quote}
{\bf Type:} bool \\
{\bf Description:} Whether BDD library BuDDy from \code{main/bdd/} must be used by bddbddb. \\
{\bf Default value:} \code{false}
\end{quote}

\code{chord.bddbddb.max.heap}
\begin{quote}
{\bf Type:} string \\
{\bf Description:} Maximum heap memory size of JVM running bddbddb. \\
{\bf Default value:} {\tt 1024m} \\
{\bf Note:} bddbddb is invoked in a separate JVM for each analysis written in Datalog that is executed.
This is primarily because multiple Datalog analyses may be executed in a single run of Chord,
resulting in multiple invocations of bddbddb, and it is difficult to reset the state of bddbddb on each invocation. 
\end{quote}

\subsection{Output Location Properties}
\label{sec:output-props}

This section describes properties specifying the names of files and directories output by Chord.
Most users will not need to alter the default values of these properties.

\code{chord.out.file}
\begin{quote}
{\bf Type:} location \\
{\bf Description:} Absolute location of the file to which the standard output stream is redirected during Chord's execution. \\
{\bf Default value:} \code{null}
\end{quote}

\code{chord.err.file}
\begin{quote}
{\bf Type:} location  \\
{\bf Description:} Absolute location of the file to which the standard error stream is redirected during Chord's execution. \\
{\bf Default value:} \code{null}
\end{quote}

\code{chord.out.dir}
\begin{quote}
{\bf Type:} location \\
{\bf Description:} Absolute location of the directory to which Chord dumps all files. \\
{\bf Default value:} \code{[chord.work.dir]/chord_output/}
\end{quote}

\code{chord.reflect.file}
\begin{quote}
{\bf Type:} location  \\
{\bf Description:} Absolute location of the file from/to which resolved reflection information is read/written. \\
{\bf Default value:} \code{[chord.out.dir]/reflect.txt}
\end{quote}

\code{chord.methods.file}
\begin{quote}
{\bf Type:} location  \\
{\bf Description:} Absolute location of the file from/to which list of methods deemed reachable is read/written. \\
{\bf Default value:} \code{[chord.out.dir]/methods.txt}
\end{quote}

\code{chord.classes.file}
\begin{quote}
{\bf Type:} location  \\
{\bf Description:} Absolute location of the file from/to which list of classes deemed reachable is read/written.  \\
{\bf Default value:} \code{[chord.out.dir]/classes.txt}
\end{quote}

\code{chord.bddbddb.work.dir}
\begin{quote}
{\bf Type:} location \\
{\bf Description:} Absolute location of the directory used by BDD-based Datalog solver bddbddb as its input/output directory (namely, for program domain files {\tt *.dom} and {\tt *.map}, and program relation files {\tt *.bdd}). \\
{\bf Default value:} \code{[chord.out.dir]/bddbddb/}
\end{quote}

\code{chord.boot.classes.dir}
\begin{quote}
{\bf Type:} location \\
{\bf Description:} Absolute location of the directory from/to which instrumented classes of the input Java program inside the JDK standard library are read/written by dynamic analyses. \\
{\bf Default value:} \code{[chord.out.dir]/boot_classes/}
\end{quote}

\code{chord.user.classes.dir}
\begin{quote}
{\bf Type:} location  \\
{\bf Description:} Absolute location of the directory from/to which instrumented classes of the input Java program outside the JDK standard library are read/written by dynamic analyses. \\
{\bf Default value:} \code{[chord.out.dir]/user_classes/}
\end{quote}

\code{chord.instr.scheme.file}
\begin{quote}
{\bf Type:} location \\
{\bf Description:} Absolute location of the file specifying the kind and format of events in trace files used by dynamic analyses. \\
{\bf Default value:} \code{[chord.out.dir]/scheme.ser}
\end{quote}

\code{chord.trace.file}
\begin{quote}
{\bf Type:} location  \\
{\bf Description:} Absolute location of trace files used by dynamic analyses. \\
{\bf Default value:} \code{[chord.out.dir]/trace} \\
{\bf Note:} Suffix \code{_full_verN.txt} or \code{_pipe_verN.txt} is appended to the name of the file, depending upon whether it is a regular file or a POSIX pipe, respectively, where {\tt N} is the version of the file (multiple versions are maintained if the trace is transformed by filters defined by the dynamic analysis; 0 is the final version).  If \code{chord.reuse.traces} is set to {\tt true}, then \code{_full_verN_runM.txt} is appended to the name of the file, where {\tt M} is the run ID. 
\end{quote}

%\item
%\code{chord.main.dir}
% location
%Absolute location of the {\tt main/} directory in Chord's installation.

%\item
%\code{chord.save.maps}
% bool
%Write to file \code{[chord.bddbddb.work.dir]/<...>.map} when saving program domain named {\tt <...>}.
%{\tt true}
%{\bf Note:} This functionality is useful for debugging Datalog programs using the {\tt debug} target provided in file {\tt build.xml} in Chord's {\tt main/} directory (see Section \ref{sec:tuning-datalog-analysis}).


\xname{setup}
\chapter{Setting Up a Program for Analysis}
\label{chap:setup}

This chapter describes how to setup a Java program for analysis using Chord.
Suppose the program has the following directory structure:

\begin{framed}
\begin{verbatim}
example/
    src/
        foo/
            Main.java
            ...
    classes/
        foo/
            Main.class
            ...
    lib/
        src/
            taz/
                ...
        jar/
            taz.jar
    chord.properties
\end{verbatim}
\end{framed}

The above structure is typical: the program's Java source
files are under {\tt src/}, its class files are under {\tt classes/},
and the source and jar files of the libraries used by the program are
under \code{lib/src/} and \code{lib/jar/}, respectively.  The
purpose of the \code{chord.properties} file is explained below.

The only way to specify inputs to Chord, including the program
to be analyzed, is via system properties.
Section \ref{sec:properties-setting} describes various ways by which
properties can be passed to Chord.  Here, we describe the
simplest approach, in which all properties of the program to be analyzed
that might be needed by Chord are defined in a file named \code{chord.properties} 
that is located in the top-level directory of the program (directory \code{example/} above).
Then, Chord can be applied to the program by running the following command:

\begin{framed}
\begin{verbatim}
ant -Dchord.work.dir=<WORK_DIR> run
\end{verbatim}
\end{framed}

This command instructs Chord to run in the directory denoted by \code{<WORK_DIR>}, where it searches for a file
named \code{chord.properties} and
loads all properties defined in that file, if it exists.
Thus, for the above program, \code{<WORK_DIR>} must be the absolute or relative path of the
\code{example/} directory.  A sample \code{chord.properties} file for the above program is as follows:

\begin{framed}
\begin{verbatim}
chord.main.class=foo.Main
chord.class.path=classes:lib/jar/taz.jar
chord.src.path=src:lib/src
chord.run.ids=0,1
chord.args.0="-thread 1 -n 10"
chord.args.1="-thread 2 -n 50"
\end{verbatim}
\end{framed}

Each relative file/directory name in the value of any property
defined in this file (e.g., the \code{lib/src} directory name in the value of
property \code{chord.src.path} above) is treated relative to the directory
specified by property \code{chord.work.dir}, whose default value
is the current directory.
Section \ref{sec:program-props} presents all program properties that are
recognized by Chord.  Here, we only describe those that are most commonly
used, namely, those defined in the above sample properties file:

\begin{itemize}
\item
\code{chord.main.class} specifies the fully-qualified name of the main
class of the program.
\item
\code{chord.class.path} specifies the application classpath
of the program (the JDK standard library classpath is implicitly
included).
\item
\code{chord.src.path} specifies the Java source path of the program.
All analyses in Chord operate on Java bytecode.  The only use
of this property is to HTMLize the Java source files of the program so
that the results of analyses can be reported at the Java
source code level.
\item
\code{chord.run.ids} specifies a list of IDs to identify runs of the
program.  It is used by dynamic analyses to determine how many
times the program must be run.  An additional use of this property is
to allow specifying the command-line arguments to use in the run
having ID {\tt <id>} via property \code{chord.args.<id>}, as
illustrated by properties \code{chord.args.0} and \code{chord.args.1}
above.
\end{itemize}

The above command does not do much beyond making Chord load the above
properties file.  For Chord to do something interesting,
additional properties must be set that specify the function(s)
Chord must perform.  All functions are summarized in Section \ref{sec:func-props}.
The most common function is to run one or more analyses on the input program;
it is described in Chapter \ref{chap:running}.


\xname{scope}
\chapter{Analysis Scope Construction}
\label{chap:scope}

A pre-requisite to analyzing a Java program using any program analysis framework,
including Chord, is to compute the {\it analysis scope}: which parts of the
program to analyze.  Several scope construction algorithms (so-called
call-graph algorithms) exist in the literature that differ in scalability (i.e.,
how large a program they can handle with the available resources) and precision (i.e., 
how much of the program they deem is reachable).

Chord implements several standard scope construction algorithms.  Besides scalability and precision,
an additional metric of these algorithms in Chord that can be controlled by users is usability,
which concerns aspects such as excluding certain code from being analyzed even if it
is reachable, and modeling Java features such as reflection, dynamic class loading, and native methods.
These features affect which code are
reachable but, in general, they cannot be modeled soundly by any program analysis framework.  The best
a framework can do is
 provide stubs for commonly-used native methods in the standard JDK library (e.g., the \code{arraycopy}
method of class \code{java.lang.System}), 
offer users a range of options on how to resolve reflection (e.g.,
an option might be running the program and observing how reflection is resolved), etc.

Chord computes the analysis scope of the given
program either if property \code{chord.build.scope} is set to {\tt true} or if some other task
(e.g., a program analysis specified via property \code{chord.run.analyses}) demands it.
The following sections describe Chord's analysis scope 
computation in detail.  Section \ref{sec:scope-reuse} describes how to reuse the analysis scope
computed in a previous run of Chord for a given program.  Section \ref{sec:scope-algos} describes
Chord's analysis scope construction algorithms.  Finally, Section \ref{sec:scope-exclude} describes how users
can exclude certain classes from the analysis scope.

%Users can set the following properties to control scope computation:
%
%\begin{itemize}
%\item \code{chord.main.class}
%\item \code{chord.class.path}
%\item \code{chord.reuse.scope}
%\item \code{chord.methods.file}
%\item \code{chord.reflect.file}
%\item \code{chord.scope.kind}
%\item \code{chord.ch.kind}
%\item \code{chord.reflect.kind}
%\item \code{chord.std.scope.exclude}
%\item \code{chord.ext.scope.exclude}
%\item \code{chord.scope.exclude}
%\end{itemize}
%
%The meaning of these properties is explained on demand in

\section{Scope Reuse}
\label{sec:scope-reuse}

If property \code{chord.reuse.scope} has value {\tt true} and both
files specified by properties \code{chord.methods.file} and
\code{chord.reflect.file} exist, then Chord regards those files as
specifying which methods to consider reachable and how to resolve
reflection, respectively.

The format of the file specified by property \code{chord.methods.file}
is a list of zero or more lines, where each line is of the form
\code{mname:mdesc@cname}
specifying the method's name {\tt mname}, the method's descriptor
{\tt mdesc}, and the method's declaring class {\tt cname} (e.g.,
\code{main:([Ljava/lang/String;)V@foo.bar.Main}).

The format of the file specified by property \code{chord.reflect.file}
is of the form:

\begin{framed}
\begin{verbatim}
# resolvedClsForNameSites
...
# resolvedObjNewInstSites
...
# resolvedConNewInstSites
...
# resolvedAryNewInstSites
...
\end{verbatim}
\end{framed}

where each of the above ``{\tt ...}'' is a list of zero or more lines, where
each line is of the form
\code{bci!mname:mdesc@cname->type1,type2,...typeN}
meaning the call site at bytecode offset {\tt bci} in the
method denoted by {\tt mname:mdesc@cname} may resolve to any of
reference types {\tt type1}, {\tt type2}, ..., {\tt typeN}.
The meaning of the above four sections is as follows.
\begin{itemize}
\item {\tt resolvedClsForNameSites} lists
each call to static method {\tt forName(String)} defined in class
\code{java.lang.Class}, along with a list of the types of the named
classes.
\item {\tt resolvedObjNewInstSites} lists
each call to instance method {\tt newInstance()} defined in class
\code{java.lang.Class}, along with a list of the types of the
instantiated classes.
\item {\tt resolvedConNewInstSites} lists
each call to instance method {\tt newInstance(Object[])} defined in class
\code{java.lang.reflect.Constructor}, along with a list of the types of the
instantiated classes.
\item {\tt resolvedAryNewInstSites} lists
each call to instance method {\tt newInstance(Class,int)} defined in class
\code{java.lang.reflect.Array}, along with a list of the types of the
instantiated classes.
\end{itemize}
The default value of property \code{chord.reuse.scope} is {\tt false}.
The default value of properties \code{chord.methods.file} and
\code{chord.reflect.file} is \code{[chord.out.dir]/methods.txt} and
\code{[chord.out.dir]/reflect.txt}, respectively.
Property \code{chord.out.dir} denotes the output directory of Chord;
its default value is \code{[chord.work.dir]/chord_output/}.
Property \code{chord.work.dir} denotes the working directory during
Chord's execution; its default value is the current directory.

\section{Scope Construction Algorithms}
\label{sec:scope-algos}

If property \code{chord.reuse.scope} has value {\tt false} or the
files specified by properties \code{chord.methods.file} or
\code{chord.reflect.file} do not exist, then Chord computes analysis
scope using the algorithm specified by property
\code{chord.scope.kind} and then writes the list of methods deemed
reachable and the reflection resolved by that algorithm to the files
specified by properties \code{chord.methods.file} and
\code{chord.reflect.file}, respectively.

The possible values of property \code{chord.scope.kind} are
[\code{rta}$|$\code{cha}$|$\code{dynamic}] (the default value is {\tt rta}).
The following subsections describe the scope construction algorithm
that Chord runs in each of these three cases.
In each case, Chord at
least expects properties \code{chord.main.class} and
\code{chord.class.path} to be set to the fully-qualified name of the
program's main class (e.g., \code{com.example.Main}) and the
program's application classpath, respectively.

\subsection{Rapid Type Analysis}

If property \code{chord.scope.kind} has value {\tt rta}, then Chord
computes analysis scope statically using Rapid Type Analysis (RTA).
RTA is an iterative fixed-point algorithm.  It maintains a set of
reachable methods $M$.  The initial iteration starts by assuming that
only the main method in the main class is reachable (Chord also
handles class initializer methods but we ignore them here for brevity;
we also ignore the set of reachable classes maintained besides the set
of reachable methods).  All object allocation sites $H$ contained in
methods in $M$ are deemed reachable (i.e., control-flow within method
bodies is ignored).  Whenever a dynamically-dispatching method call
site (i.e., an invokevirtual or invokeinterface site) with receiver of
static type $t$ is encountered in a method in $M$, only subtypes of
$t$ whose objects are allocated at some site in $H$ are considered to
determine the possible target methods, and each such target method is
added to $M$.  The process terminates when no more methods can be
added.

%RTA is a relatively inexpensive and precise algorithm in practice.
%Its key shortcoming is that it makes no attempt to resolve
%reflection, which is rampant in real-world Java programs, and can
%therefore be unsound (i.e., underestimate the set of reachable
%classes and methods).  The next option attempts to overcome this
%problem.  \item The \code{rta_reflect} value instructs Chord to
%compute analysis scope statically using Rapid Type Analysis and,
%moreover, to resolve a common reflection
%pattern: \begin{quote} \begin{verbatim} String s = ...; Class c =
%Class.forName(s); Object o = c.newInstance(); T t = (T)
%o; \end{verbatim} \end{quote} This analysis is identical to RTA
%except that it additionally inspects every cast statement in the
%program, such as the last statement in the above snippet, and queries
%the class hierarchy to find all concrete classes that subclass
%\code{T} (if \code{T} is a class) or that implement \code{T} (if
%\code{T} is an interface).  Chord allows users to control which
%classes are included in the class hierarchy (see Section
%\ref{sec:cha}).

\subsection{Class Hierarchy Analysis}

If property \code{chord.scope.kind} has value {\tt cha}, then Chord
computes analysis scope statically using Class Hierarchy Analysis (CHA).
The key difference between CHA and RTA is that for invokevirtual and
invokeinterface sites with receiver of static type $t$, CHA considers
{\it all} subtypes of $t$ in the class hierarchy to determine the
possible target methods, whereas RTA restricts them to types of
objects allocated in methods deemed reachable so far.  As a result,
CHA is highly imprecise in practice, and also expensive since it
grossly overestimates the set of reachable classes and methods.
Nevertheless, Chord allows users to control which classes are
included in the class hierarchy, and thereby control the
precision and cost of CHA, by setting property \code{chord.ch.kind},
whose possible values are [\code{static}$|$\code{dynamic}] (the default
value is {\tt static}).

Chord first constructs the entire classpath of the given program by
concatenating in order the following classpaths:

\begin{enumerate}
\item
The boot classpath, specified by property \code{sun.boot.class.path}.
\item
The library extensions classpath, comprising all jar files in
directory \code{[java.home]/lib/dir/}.
\item
The application classpath of the given program, specified by property \code{chord.class.path},
which is empty by default.
\end{enumerate}

All classes in the entire classpath (resulting from items 1--3 above)
are included in the class hierarchy with the following exceptions:
\begin{itemize}
\item
Duplicate classes, i.e., classes with the same name occurring in more
than one classpath element; in this case, all occurrences except the
first are excluded.
\item
Any class whose name's prefix is specified in the value of property
\code{chord.scope.exclude} (see Section \ref{sec:scope-exclude}).
\item
If property \code{chord.ch.kind} has value {\tt dynamic}, then
Chord runs the given program and observes the set of all classes the
JVM loads; any class not in this set is excluded.
\item
If the superclass of a class C is missing or if an interface
implemented/extended by a class/interface C is missing, where
``missing" means that it is either not in the classpath resulting from
items 1--3 above or it is excluded by one of these rules, then C
itself is excluded.  Note that this rule is recursive, e.g., if C has
superclass B which in turn has superclass A, and A is missing, then
both B and C are excluded.
\end{itemize}

\subsection{Dynamic Analysis}

If property \code{chord.scope.kind} has value {\tt dynamic}, then Chord
computes analysis scope
dynamically, by running the program and observing the
classes that are loaded at run-time.  The number of times the program
is run and the command-line arguments to be supplied to the program in
each run is specified by properties \code{chord.run.ids} and
\code{chord.args.<id>} for each run ID {\tt <id>}.  By default, the
program is run only once, using run ID {\tt 0}, and without any
command-line arguments.  Only classes loaded in some run are regarded
as reachable but {\it all} methods of each loaded class are regarded
as reachable regardless of whether they were invoked in the run.  The
rationale behind this decision is to both reduce the run-time instrumentation
overhead and increase the predictive power of program analyses
performed using the computed analysis scope.

\section{Scope Exclusion}
\label{sec:scope-exclude}

Chord can be instructed to exclude certain classes in a given program 
from being analyzed.
This functionality might be desirable, for instance, if the given program
contains a larger framework (e.g., Hadoop or Android) which
must not be analyzed.
Chord provides three properties for this purpose.
The value of each of these properties is a comma-separated list of prefixes
of names of classes.  Chord treats the body of each method defined in
each such class as a no-op.

\begin{itemize}
\item
Property \code{chord.std.scope.exclude} is intended to specify
classes to be excluded from the scope of {\it all} programs to be
analyzed, e.g., classes in the JDK standard library.  Its default
value is the empty list.
\item
Property \code{chord.ext.scope.exclude} is intended to specify
classes to be excluded from the scope of specific programs to be
analyzed.  Its default value is the empty list.
\item
Property \code{chord.scope.exclude} specifies the final list of
classes to be excluded from scope.  Its default value is
\code{[chord.std.scope.exclude],[chord.ext.scope.exclude]}.
\end{itemize}

{\bf Note:} The value of each of the above properties is a list of {\it prefixes},
not {\it regular expressions}.  A valid value is ``\code{java.,com.sun.}",
but not ``\code{java.*,com.sun.*}".

%\begin{tabular}
%chord.scope.kind,chord.ch.dynamic & runs program? & \# reachable classes & \# reachable methods & running time \\
%dynamic,-         & yes &    314 &  4,491 &   25s \\
%cha,true          & yes &    427 &  1,532 &   28s \\
%rta,-             &  no &    849 &  4,836 &    9s \\
%rta_reflect,true  & yes &    849 &  4,836 &   34s \\
%rta_reflect,false &  no &  9,871 & 58,726 &  7m6s \\
%cha,false         &  no & 14,121 & 74,613 & 4m11s
%\end{tabular}


\xname{predefined}
\chapter{Predefined Analyses}
\label{chap:predefined}

This chapter describes various predefined analyses in Chord.

\section{Points-to and Call-Graph Analyses}

Chord offers several choices for computing points-to and call-graph information of Java programs.
In each of these choices, points-to and call-graph information is computed simultaneously
(called ``on-the-fly call-graph construction" in the literature in contrast to ``ahead-of-time call-graph
construction" in which the call graph is computed first followed by points-to information).
On-the-fly approaches are more precise because, in a dynamically dispatching language like Java,
as more points-to facts are discovered, more (dynamically dispatched) methods are deemed reachable,
thereby growing the call graph; the code in these newly added methods in turn results in more
points-to facts.

Flow-insensitive analysis computes a single abstract heap whereas flow-sensitive analysis computes
per-program-point abstract heaps.  Context-insensitive analysis analyzes each method at most once (i.e. in a single abstract context),
whereas context-sensitive analyses potentially analyze each method multiple times, in different abstract contexts.
Thus, flow- and context-sensitive analyses are more precise but less scalable than flow- and context-insensitive
analyses, respectively.

Flow-sensitive analysis does not offer much precision over flow-insensitive analysis in
practice, especially in the absence of strong updates and in the presence of SSA (Static Single Assignment form),
a program representation that renders a flow-insensitive analysis almost as precise as a flow-sensitive analysis.
Since analyses in Chord currently perform only weak updates, and since they all operate on an SSA form of the
input Java program by default, the rest of this section focuses only on flow-insensitive analysis, which is the predominant
kind of points-to/call-graph analysis in Chord.

We describe context-insensitive analysis first because understanding the concepts behind it will help understand
the more sophisticated context-sensitive analyses.
We first recall some relevant program domains:

\texonly{\newpage}

\begin{itemize}
\item M is the domain of all methods.
\item I is the domain of all method call sites.
\item F is the domain of all (instance and static) fields.
\item V is the domain of all local variables of reference type.
\item H is the domain of all object allocation sites.
\end{itemize}

\subsection{Context-Insensitive Analysis}

The context-insensitive points-to/call-graph analysis 
treats each object allocation site as a separate abstract memory location; in other words, it
can distinguish objects created at different sites but not those created at the same site.
Additionally, it is field-sensitive, in that it distinguishes between different instance fields of the same object,
but array-insensitive, in that it cannot distinguish between different elements of the same array;
all array elements are modeled using a distinguished hypothetical instance field (that has index 0 in domain F).

To run this analysis, run the following command:

\begin{framed}
\begin{verbatim}
ant -Dchord.work.dir=<WORK_DIR> -Dchord.run.analyses=cipa-0cfa-dlog run
\end{verbatim}
\end{framed}

where \code{<WORK_DIR>} is a directory
containing a file named \code{chord.properties} that defines
properties \code{chord.main.class} and \code{chord.class.path} specifying
the main class and the application classpath, respectively, of the program
to be analyzed.

This analysis outputs the following relations:

%is implemented in \code{chord/analyses/alias/cipa-0cfa.dlog}.

\begin{itemize}
\item
{\it Call-graph information:}
\begin{itemize}
\item rootM subset M contains the set of entry methods that may be reachable; this includes the program's main method
as well as each static initializer method that may be reachable from the main
method.
\item reachableM subset M contains the set of methods that may be reachable from the program's main method.
\item IM subset (I $\times$ M) contains tuples \texttt{(i,m)} such that call site \texttt{i} may call method \texttt{m}.
%\item MM subset (M $\times$ M) contains tuples \texttt{(m1,m2)} such that method \texttt{m1} may call method \texttt{m2}.
%\item reachableI subset I contains the set of call sites that may be reachable.
%\item reachableT subset T contains the set of classes that may be reachable.
\end{itemize}
\item
{\it Points-to information:}
\begin{itemize}
\item FH subset (F $\times$ H) contains tuples \texttt{(f,h)} such that static field (i.e. global variable) \texttt{f} may point to an object allocated at site \texttt{h}.
\item VH subset (V $\times$ H) contains tuples \texttt{(v,h)} such that local variable \texttt{v} may point to an object allocated at site \texttt{h}.
\item HFH subset (H $\times$ F $\times$ H) contains tuples \texttt{(h1,f,h2)} such that instance field \texttt{f} of some object allocated at
site \texttt{h1} may point to some object allocated at site \texttt{h2}.
\end{itemize}
\end{itemize}

\subsection{Context-Sensitive Analysis}

In a context-sensitive analysis, there is no longer just one abstract memory
location per object allocation site.  Rather, the set of objects a reference can
point to depends on the \textit{context} in which the method containing the
reference is called.  Whereas a context-insensitive analysis talks about the
domain of methods (M) and the domain of allocation sites (H), a
context-sensitive analysis talks about the domain of abstract contexts, labeled
C.  Elements of domain C contain both abstract calling contexts and abstract
objects. (These are merged for reasons described below.)

Chord has several context-sensitive analyses, but they all expose the same
relations, which are described below:

\begin{itemize}
\item
{\it Context information:}
\begin{itemize}
\item C is the domain of all abstract calling contexts and abstract objects.
Each element in this domain is a sequence of zero or more sites, where each site
can be a call site or an object allocation site.  A sequence may have mixed call
sites and object allocation sites.  The most significant site (i.e. the first
site) in each sequence is called the head; the remaining sub-sequence is called
the tail.  The below three relations relate a sequence with its head and tail.
\item CC subset (C $\times$ C) contains tuples \texttt{(c1,c2)} such that context \texttt{c2} is the tail of context \texttt{c1}.
\item CH subset (C $\times$ H) contains tuples \texttt{(c,h)} such that object allocation site \texttt{h} is the head of context \texttt{c}.
\item CI subset (C $\times$ I) contains tuples \texttt{(c,i)} such that call site \texttt{i} is the head of context \texttt{c}.
\end{itemize}
\item
{\it Call-graph information:}
\begin{itemize}
\item rootCM subset (C $\times$ M) contains tuples \texttt{(c,m)} such that method \texttt{m} is an entry method in context \texttt{c}.
\item reachableCM subset (C $\times$ M) contains tuples \texttt{(c,m)} such that method \texttt{m} may be reachable in context \texttt{c}.
\item CICM subset (C $\times$ I $\times$ C $\times$ M) contains tuples \texttt{(c1,i,c2,m)} such that call site \texttt{i} in
context \texttt{c1} may call method \texttt{m2} in context \texttt{c2}.
\end{itemize}
\item
{\it Points-to information:}
\begin{itemize}
\item FC subset (F $\times$ C) contains tuples \texttt{(f,o)} such that static field (i.e. global variable) \texttt{f} may point to object \texttt{o}.
\item CVC subset (C $\times$ V $\times$ C) contains tuples \texttt{(c,v,o)} such that local variable \texttt{v} may point to
object \texttt{o} in context \texttt{c} of that variable's declaring
method. Note that both \texttt{o} and \texttt{c} are elements of domain C.
\item CFC subset (C $\times$ F $\times$ C) contains tuples \texttt{(o1,f,o2)} such that instance field \texttt{f} of object \texttt{o1} may point to object \texttt{o2}.
\end{itemize}
\end{itemize}

{\bf Under Construction: explain object-sensitive vs context-sensitive, plus how
to invoke.}

\section{Static Datarace Analysis}

To run Chord's static datarace analysis, run the following command:

\begin{framed}
\begin{verbatim}
ant -Dchord.work.dir=<WORK_DIR> -Dchord.run.analyses=datarace-java run
\end{verbatim}
\end{framed}

where \code{<WORK_DIR>} is a directory
containing a file named \code{chord.properties} that defines
properties \code{chord.main.class}, \code{chord.class.path}, and
\code{chord.src.path} specifying the main class, the application classpath, and
the Java source path, respectively, of the program to be analyzed.

Directory \code{<CHORD_MAIN_DIR>/examples/datarace_test/} provides a toy Java
program on which one can run the datarace analysis.  First run {\tt ant} in that
directory (in order to compile the program's {\tt .java} files to {\tt .class}
files) and then run the above command with \code{<WORK_DIR>} replaced by
\code{examples/datarace_test/}.  Upon successful completion, the
following files should be produced in directory
\code{examples/datarace_test/chord_output/}:

\begin{itemize}
\item
File \code{dataraces_by_fld.html}, listing all dataraces grouped by the field on
which they occur; all dataraces on the same instance field or the same static
field are listed in the same group, and so are all dataraces on array elements.
\item
File \code{dataraces_by_obj.html}, listing all dataraces grouped by the abstract
object on whose field they occur; dataraces on all static fields are listed in
the same group, and so are dataraces on different instance fields of the same
abstract object.
\end{itemize}

\section{Static Deadlock Analysis}

To run Chord's static deadlock analysis,  run the following command:

\begin{framed}
\begin{verbatim}
ant -Dchord.work.dir=<WORK_DIR> -Dchord.run.analyses=deadlock-java run
\end{verbatim}
\end{framed}

where \code{<WORK_DIR>} is a directory
containing a file named \code{chord.properties} that defines
properties \code{chord.main.class}, \code{chord.class.path}, and
\code{chord.src.path} specifying the main class, the application classpath,
and the Java source path, respectively, of the program to be analyzed.

Directory \code{<CHORD_MAIN_DIR>/examples/deadlock_test/} provides a toy Java
program on which one can run the deadlock analysis.  First run {\tt ant} in that
directory (in order to compile the program's {\tt .java} files to {\tt .class}
files) and then run the above command with \code{<WORK_DIR>} replaced by
\code{examples/deadlock_test/}.  Upon successful completion, the
file \code{deadlocks.html} should be produced in directory
\code{examples/deadlock_test/chord_output/}.



\xname{developer_guide}
\part{Guide for Developers}

\xname{arch}
\chapter{Architecture of Chord}
\label{chap:arch}

This chapter presents the high-level architecture of Chord, depicted below, and describes its key components.

\begin{center}
\includeimage{0.7}{chord_arch.png}
\end{center}

{\bf Chord Properties}

All inputs to Chord are specified by means of system properties.  Chapter
\ref{chap:properties} describes how to set properties and the meaning of each
property that is recognized by Chord.

The Java program to be analyzed is also specified via properties.  Chapter
\ref{chap:setup} describes how to setup a Java program for analysis using Chord.
Chord analyzes Java bytecode, not Java source code, and thus only requires the
program's class files.  Certain analyses, however, present their results at the
Java source code level, and thus require the program's Java source files as
well.

{\bf Java Program Representation}

Chord uses the \xlink{Joeq}{http://joeq.sourceforce.net/} Java compiler
framework to convert the Java bytecode of the input Java program, one class file
at a time, into a three-address-like intermediate program representation called
{\it quadcode} that is more suitable for analysis. Chapter
\ref{chap:program} describes the quadcode representation in detail.

{\bf Analysis Scope Construction}

A pre-requisite to analyzing a Java program using any program analysis
framework, including Chord, is to compute the {\it analysis scope}: which parts
of the program to analyze.  Chord implements several standard scope construction
algorithms from the literature that differ in aspects such as scalability,
precision, and usability for the problem at hand.  Chapter \ref{chap:scope}
describes these algorithms in detail.

{\bf Writing and Running Analyses}

Chord provides many standard analyses.  Chapter \ref{chap:predefined} describes
these analyses and how to run them.  Moreover, Chord allows users to define
their own analyses, possibly atop the provided analyses.

A distinctive aspect of Chord is that each analysis is written modularly,
independent of other analyses, along with lightweight annotations specifying the
inputs and outputs of the analysis.  Chord's runtime automatically computes
dependencies between analyses (e.g., determines which analysis produces as
output a result that is needed as input by another analysis).  Before running a
desired analysis, Chord recursively runs other analyses until the inputs to the
desired analysis have been computed; it finally runs the desired analysis to
produce the outputs of that analysis.

Chord can be invoked in one of two modes: {\it classic} or {\it modern}.  These
two modes defer in the semantics of dependencies between analyses.  In
particular, the classic mode is simpler to understand for novice users (the
dependencies are only data dependencies) but has a sequential runtime, whereas
the modern mode is harder to understand (there are both data and control
dependencies) but has a parallel runtime that is capable of running analyses
without dependencies between them in parallel.  The parallel runtime is based on
\xlink{Habanero-Java}{http://habanero.rice.edu/hj.html}, and the semantics of
the dependencies between analyses is based on the
\xlink{Habanero Concurrent Collections (CnC)}{http://habanero.rice.edu/cnc.html}
declarative parallel programming model.  Chapter \ref{chap:project} expands upon
the modular architecture of analyses in Chord.

Chord provides various {\it analysis templates}: classes containing boilerplate
code that can be extended by users to rapidly prototype different kinds of
analyses.  An example is class RHSAnalysis, named after [Reps, Horowitz, and
Sagiv 1995], which can be extended by users to write a summary-based
inter-procedural context-sensitive static analysis by merely specifying the
abstract domain and intra-procedural transfer functions.  Another example is
DynamicAnalysis, which can be extended by users to write a dynamic analysis by
merely specifying which of various provided events to instrument, and the
transfer functions for those events.  Chapters \ref{chap:writing} and
\ref{chap:running} describe how to write and run your own analyses in Chord
using the provided analysis templates.

{\bf Dynamic Analysis}

Chord uses the \xlink{Javassist}{http://www.javassist.org/} Java bytecode
manipulation framework for instrumenting bytecode and doing dynamic analysis.
Chord offers the most versatile capabilities of any existing dynamic analysis
framework for Java, particularly the ability to instrument the entire JDK
(including classes in package java.lang).  Specifically, it includes support
for:

\begin{itemize}
\item
offline as well as load-time instrumentation of Java bytecode;
\item
processing of dynamic analysis events online in the same JVM or offline in a
different JVM with an uninstrumented JDK (the latter circumvents performance and
correctness problems that can arise if a single JVM with an instrumented JDK is
used to generate and handle events); and
\item
allowing the event-generating and event-handling JVMs to run either serially
(by storing the entire trace of events to a regular file) or in parallel (by
streaming the trace of events in a piped file).
\end{itemize}

Chapter \ref{chap:dynamic} describes all aspects of dynamic analysis in Chord.

{\bf Datalog Analysis}

A common way to rapidly prototype an analysis in Chord is using a declarative
logic-programming language called Datalog.  Chord uses the BDD-based Datalog
solver \xlink{bddbddb}{http://bddbddb.sourceforge.net/} to run analyes written
in Datalog.  Chapter \ref{chap:datalog} describes all aspects of such analyses.


\xname{program}
\chapter{Java Program Representation}
\label{chap:program}

Chord uses \xlink{Joeq}{http://joeq.sourceforge.net/} to translate
Java bytecode, one class file at a time, into a three-address-like
intermediate representation of the input Java program called {\it
quadcode}.  This chapter describes all aspects of quadcode and how it
relates to bytecode.  It first explains how to pretty-print bytecode
and quadcode (Section \ref{sec:pretty-printing}) which is useful for
debugging analyses and deciphering their output.  The remaining
sections describe the quadcode representation along with the API of
Joeq and Chord for navigating it.  Briefly, the representation
consists of a set of classes that may be loaded
(Section \ref{sec:program}).  The representation of each class consists
of a set of members (Section \ref{sec:class-members}) which are the fields and methods
of the class.  The representation of a concrete
method (Section \ref{sec:methods}) consists of a control-flow graph
(CFG).  The representation of a CFG (Section \ref{sec:cfgs}) consists
of a set of registers and a set of basic blocks linked by directed edges
denoting flow of control between basic blocks.
Each basic block contains zero or more primitive statements called quads
(Section \ref{sec:quads}).  Finally, the most common way to traverse
all quads is discussed (Section \ref{sec:traversing}).

\section{Pretty-Printing}
\label{sec:pretty-printing}

Consider the following Java program contained in
file \code{examples/hello_world/src/test/HelloWorld.java} in Chord's
main directory:
\begin{framed}
\begin{verbatim}
package test;

public class HelloWorld {
    public static void main(String[] args) {
        System.out.println("Hello World!");
    }
}
\end{verbatim}
\end{framed}

First compile this program by running command \code{ant} in
directory \code{examples/hello_world/}.

To pretty-print the bytecode representation of a class, run the following command:
\begin{framed}
\begin{verbatim}
javap -classpath <CLASS_PATH> -bootclasspath <BOOT_CLASS_PATH> -private -verbose <CLASS_NAME>
\end{verbatim}
\end{framed}

where:

\begin{itemize}
\item
\verb+<CLASS_NAME>+ is the fully-qualified name of the class whose
bytecode is to be printed (\code{test.HelloWorld} in our example).
\item
\verb+<CLASS_PATH>+ is the classpath of that class (\code{examples/hello_world/} in our example).
\item
\verb+<BOOT_CLASS_PATH>+ is the boot classpath; it is optional and must be supplied if
\verb+<CLASS_NAME>+ is a class from the JDK standard library (e.g., \code{java.util.ArrayList}) that has
been modified and written to a user-defined location (e.g., it has been instrumented by Chord and
written to \code{chord_output/boot_classes/}).
\end{itemize}

Program \code{javap} comes along with the JVM.  The output of the
above command for our example is as follows:

\begin{framed}
\begin{verbatim}
Compiled from "HelloWorld.java"
public class test.HelloWorld extends java.lang.Object
  SourceFile: "HelloWorld.java"
  minor version: 0
  major version: 49
  Constant pool:
const #1 = Method   #6.#20; //  java/lang/Object."<init>":()V
const #2 = Field    #21.#22;    //  java/lang/System.out:Ljava/io/PrintStream;
const #3 = String   #23;    //  Hello World!
const #4 = Method   #24.#25;    //  java/io/PrintStream.println:(Ljava/lang/String;)V
const #5 = class    #26;    //  test/HelloWorld
const #6 = class    #27;    //  java/lang/Object
const #7 = Asciz    <init>;
const #8 = Asciz    ()V;
const #9 = Asciz    Code;
const #10 = Asciz   LineNumberTable;
const #11 = Asciz   LocalVariableTable;
const #12 = Asciz   this;
const #13 = Asciz   Ltest/HelloWorld;;
const #14 = Asciz   main;
const #15 = Asciz   ([Ljava/lang/String;)V;
const #16 = Asciz   args;
const #17 = Asciz   [Ljava/lang/String;;
const #18 = Asciz   SourceFile;
const #19 = Asciz   HelloWorld.java;
const #20 = NameAndType #7:#8;//  "<init>":()V
const #21 = class   #28;    //  java/lang/System
const #22 = NameAndType #29:#30;//  out:Ljava/io/PrintStream;
const #23 = Asciz   Hello World!;
const #24 = class   #31;    //  java/io/PrintStream
const #25 = NameAndType #32:#33;//  println:(Ljava/lang/String;)V
const #26 = Asciz   test/HelloWorld;
const #27 = Asciz   java/lang/Object;
const #28 = Asciz   java/lang/System;
const #29 = Asciz   out;
const #30 = Asciz   Ljava/io/PrintStream;;
const #31 = Asciz   java/io/PrintStream;
const #32 = Asciz   println;
const #33 = Asciz   (Ljava/lang/String;)V;

{
public test.HelloWorld();
  Code:
   Stack=1, Locals=1, Args_size=1
   0:   aload_0
   1:   invokespecial   #1; //Method java/lang/Object."<init>":()V
   4:   return
  LineNumberTable:
   line 3: 0
  LocalVariableTable:
   Start  Length  Slot  Name   Signature
   0      5      0    this       Ltest/HelloWorld;

public static void main(java.lang.String[]);
  Code:
   Stack=2, Locals=1, Args_size=1
   0:   getstatic   #2; //Field java/lang/System.out:Ljava/io/PrintStream;
   3:   ldc #3; //String Hello World!
   5:   invokevirtual   #4; //Method java/io/PrintStream.println:(Ljava/lang/String;)V
   8:   return
  LineNumberTable:
   line 5: 0
   line 6: 8
  LocalVariableTable:
   Start  Length  Slot  Name   Signature
   0      9      0    args       [Ljava/lang/String;
}
\end{verbatim}
\end{framed}

To pretty-print the quadcode representation of a class, run the following command:

\begin{framed}
\begin{verbatim}
ant -Dchord.work.dir=<WORK_DIR> -Dchord.print.classes=<CLASS_NAME> \
    -Dchord.verbose=0 -Dchord.out.file=<OUT_FILE> run
\end{verbatim}
\end{framed}

where:

\begin{itemize}
\item
\code{<WORK_DIR>} is a directory (\code{examples/hello_world/} in our example) 
that contains a file named \code{chord.properties} which defines
properties \code{chord.main.class} and \code{chord.class.path}
specifying the main class and the classpath of the input Java program.
Alternatively, these two properties can be defined directly on the
above command-line.
\item
\code{<CLASS_NAME>} is the fully-qualified name of the class whose quadcode is to be printed
({\tt test.HelloWorld} in our example).  Each occurrence of a `{\tt \$}' in the class name
must be replaced by a `{\tt \#}'.
\item
\code{<OUT_FILE>} is the file to which the quadcode must be written; if left unspecified,
the quadcode is written to the standard output.
\end{itemize}

The output of the above command for our example is as follows:

\begin{framed}
\begin{verbatim}
*** Class: test.HelloWorld
Method: main:([Ljava/lang/String;)V@test.HelloWorld
    0#1
    5#3
    5#2
    8#4
Control flow graph for main:([Ljava/lang/String;)V@test.HelloWorld:
BB0 (ENTRY) (in: <none>, out: BB2)

BB2 (in: BB0 (ENTRY), out: BB1 (EXIT))
1: GETSTATIC_A T1, .out
3: MOVE_A T2, AConst: "Hello World!"
2: INVOKEVIRTUAL_V println:(Ljava/lang/String;)V@java.io.PrintStream, (T1, T2)
4: RETURN_V

BB1 (EXIT)  (in: BB2, out: <none>)

Exception handlers: []
Register factory: Registers: 3
\end{verbatim}
\end{framed}

\section{Whole Program}
\label{sec:program}

This and the following sections describe the quadcode representation
along with the API of Joeq and Chord for navigating it.
This API is contained in
packages \javadoc{chord.program}{chord/program/package-summary.html}, \javadoc{joeq.Class}{joeq/Class/package-summary.html},
and \javadoc{joeq.Compiler.Quad}{joeq/Compiler/Quad/package-summary.html}.

The quadcode representation of the whole program is a unique global
object of
class \javadoc{chord.program.Program}{chord/program/Program.html}
which can be obtained by calling static
method \javadoc{chord.program.Program.g()}{chord/program/Program.html\#g()}.
This class provides a rich API (in the form of public instance
methods) to access various parts of the representation, most notably:

\begin{mytable}{|l|l|}
\hline
\verb+IndexSet<jq_Type> getTypes()+ & All types referenced in classes that may be loaded. \\
\hline
\verb+IndexSet<jq_Reference> getClasses()+ & All classes that may be loaded.* \\
\hline
\verb+IndexSet<jq_Method> getMethods()+ & All methods that may be called. \T \\
\hline
\end{mytable}

* Includes both classes/interfaces and array types, represented as objects
of \code{jq_Class} and \code{jq_Array}, respectively; both these are
subclasses of \code{jq_Reference}.

See Chapter \ref{chap:scope} for how Chord determines which
classes may be loaded and which methods may be called.

The quadcode representation of each type is a unique object of the
appropriate subclass
of \javadoc{joeq.Class.jq_Type}{joeq/Class/jq_Type.html} in the following
hierarchy:

\begin{verbatim}
                    jq_Type
                       |
      ------------------------------------
      |                                  |
jq_Primitive                       jq_Reference
                                         |
                          -------------------------------
                          |                             |
                      jq_Class                       jq_Array
\end{verbatim}

\section{Class Members}
\label{sec:class-members}

Each primitive type (e.g., boolean, int, etc.) is represented by a
unique \code{jq_Primtive} object.  Each class and each interface type is
represented by a unique \code{jq_Class} object.  Each array type is
represented by a unique \code{jq_Array} object.

Members (i.e., fields and methods) of the class/interface represented
by an object of
class \javadoc{joeq.Class.jq_Class}{joeq/Class/jq_Class.html} can be
accessed using the following API provided by that class.

\begin{mytable}{|l|p{3.1in}|}
\hline
\verb+String getName()+ & Fully-qualified name of the class, e.g., ``\code{java.lang.String[]}''. \\
\hline
\verb+jq_InstanceField[] getDeclaredInstanceFields()+ & All instance fields declared in the class. \\
\hline
\verb+jq_StaticField[] getDeclaredStaticFields()+ & All static fields declared in the class. \\
\hline
\verb+jq_InstanceMethod[] getDeclaredInstanceMethods()+ & All instance methods declared in the class. \\
\hline
\verb+jq_StaticMethod[] getDeclaredStaticMethods()+ & All static methods declared in the class. \T \\
\hline
\end{mytable}

Chord uses format \code{mName:mDesc@cName}, described in
class \javadoc{chord.program.MethodSign}{chord/program/MethodSign.html},
to uniquely identify each field and each method in the input Java
program, where
\code{mName} denotes the name of the field/method,
\code{mDesc} denotes the descriptor of the field/method (see below),
and \code{cName} denotes the fully-qualified name of the class
declaring the field/method.
For instance, ``\code{main:[Ljava/lang/String;@test.HelloWorld}''
uniquely identifies the main method in the example above.
We next review field descriptors and method descriptors from the Java
bytecode specification.

A field descriptor represents the type of a local variable or a
(static or instance) field.  It is a series of characters generated by
the grammar:

\begin{verbatim}
FieldDescriptor : FieldType
      FieldType : BaseType | ObjectType | ArrayType
       BaseType : B | C | D | F | I | J | S | Z
     ObjectType : L <classname> ;
      ArrayType : [ ComponentType
  ComponentType : FieldType
\end{verbatim}

The characters of \code{BaseType}, the `{\tt L}' and `{\tt ;}'
of \code{ObjectType}, and the `{\tt [}' of
\code{ArrayType} are all ASCII characters.
The {\tt <classname>} represents a fully qualified class or interface
name.  The interpretation of the field types is as shown in the below
table:

\begin{mytable}{l|l|l}
\hline
BaseType Character	& Type	& Interpretation \\
\hline
{\tt B} &byte& signed byte \\
{\tt C} &char& Unicode character \\
{\tt D} &double& double-precision floating-point value \\
{\tt F} &float& single-precision floating-point value \\
{\tt I} &int& integer \\
{\tt J} &long& long integer \\
{\tt L<classname>;}	&reference& an instance of class {\tt <classname>} \\
{\tt S}	&short& signed short \\
{\tt Z}	&boolean& true or false \\
{\tt [}	&reference& one array dimension \T \\
\hline
\end{mytable}

For example, the descriptor of type {\tt int}
 is simply {\tt I}.  The descriptor of an instance variable of
 type \code{Object} is ``\code{Ljava/lang/Object;}''.  Note that the
 internal form of the fully qualified name for class \code{Object} is
 used. The descriptor of a multidimensional double array of type
 ``\code{double[][][]}'' is ``\code{[[[D}''.

A method descriptor represents the types of the arguments and 
return result of a method:

\begin{verbatim}
   MethodDescriptor : ( ParameterDescriptor* ) ReturnDescriptor
ParameterDescriptor : FieldType
   ReturnDescriptor : FieldType | V
\end{verbatim}

A parameter descriptor represents the type of an argument of a method.
A return descriptor represents the type of the return result of a
method.  The character `{\tt V}' indicates that the method returns no
value (its return type is void).

The method descriptor is the same whether it is a static or an instance
method.  Although an instance method is passed \code{this}, a reference
to the current class instance, in addition to its intended arguments,
that fact is not reflected in the method descriptor.

For example, the method descriptor
for the method ``\code{Object foo(int i, double d, Thread t)}'' is
``\code{(IDLjava/lang/Thread;)Ljava/lang/Object;}''.  Note that
internal forms of the fully qualified names of \code{Thread} and
\code{Object} are used in the method descriptor.

\section{Methods}
\label{sec:methods}

The quadcode representation of each method is a unique object of
class \javadoc{joeq.Class.jq_Method}{joeq/Class/jq_Method.html}.  Components of the method,
most notably its control-flow graph,
can be accessed using the following API provided
by that class.

\begin{mytable}{|l|p{4in}|}
\hline
\verb+String getName()+ & Name of the method. \\
\hline
\verb+String getDesc().toString()+ & Descriptor of the method, e.g., ``\code{(Ljava/lang/String;)V}''. \\
\hline
\verb+jq_Class getDeclaringClass()+ & Declaring class of the method. \\
\hline
\verb+ControlFlowGraph getCFG()+ & Control-flow graph of the method.* \\
\hline
\verb+int getLineNumber(int bci)+ & Line number of the given bytecode offset (-1 if not found). \\
\hline
\verb+Quad getQuad(int bci)+ & First quad at the given bytecode
offset (null if not found). \\
\hline
\verb+Quad getQuad(int bci, Class kind)+ & First quad of the given
kind at the given bytecode offset (null if not found). \\
\hline
\verb+Quad getQuad(int bci, Class[] kind)+ & First quad of any
given kind at the given bytecode offset (null if not found). \\
\hline
\verb+String toString()+ & Unique identifier of the method in format \code{mName:mDesc@cName}. \T \\
\hline
\end{mytable}

* The control-flow graph must not be asked if the method is abstract (which can be determined by calling instance method \code{isAbstract()} of \code{jq_Method}).

\section{Control-Flow Graphs}
\label{sec:cfgs}

The control-flow graph (CFG) of each method consists of a set of registers,
called the register factory, and a directed graph whose nodes are basic blocks and
whose edges denote flow of control between basic blocks.

The CFG of each method is a unique object of
class \javadoc{joeq.Compiler.Quad.ControlFlowGraph}{joeq/Compiler/Quad/ControlFlowGraph.html}.
Components of the CFG can be accessed using the following API provided
by that class.

\begin{mytable}{|l|l|}
\hline
\verb+getRegisterFactory()+ & Set of all local variables. \\
\hline
\verb+EntryOrExitBasicBlock entry()+ & Unique entry basic block. \\
\hline
\verb+EntryOrExitBasicBlock exit()+ & Unique exit basic block. \\
\hline
\verb+ListIterator.BasicBlock reversePostOrderIterator()+ & Iterator over all basic blocks in reverse post-order. \\
\hline
\verb+jq_Method getMethod()+ & Containing method of the CFG. \T \\
\hline
\end{mytable}

%\subsection{Register Factory}

The register factory contains one register per argument of the method
(called {\it local variables}) and one register per temporary 
in the method body (called {\it stack variables}).
Temporaries include those declared by programmers as well as those
generated by Joeq.  The reason Joeq can generate temporaries is that the quadcode
representation, which is register-based, is constructed from Java
bytecode, which is stack-based; moreover, Joeq does the Static Single
Assignment (SSA) transformation by default, which introduces
temporaries to ensure that there is at most one static assignment to
any variable.  Registers corresponding to local variables are named {\tt R0}, {\tt R1}, ..., {\tt Rn},
while those corresponding to stack variables are named {\tt Tn+1}, {\tt Tn+2}, ..., {\tt Tm}.

For instance, the register factory of the main method in the example above 
has 3 registers: {\tt R0} denoting the {\tt args} argument of the method
and {\tt T1} and {\tt T2} denoting temporaries generated by Joeq.

Each register factory is a unique object
of class \javadoc{joeq.Compiler.Quad.RegisterFactory}{joeq/Compiler/Quad/RegisterFactory.html}.

Besides the register factory, a CFG has a directed graph whose nodes are basic blocks and whose edges
denote flow of control between basic blocks.
Each basic block contains a straight-line sequence of zero or more primitive statements called quads
(Section \ref{sec:quads}).  Each CFG is guaranteed to contain at least
two basic blocks: a unique entry basic block with no incoming edges
and a unique exit block with no outgoing edges.  The entry and exit
basic blocks do not contain any quads.

Each basic block is a unique object of
class \javadoc{joeq.Compiler.Quad.BasicBlock}{joeq/Compiler/Quad/BasicBlock.html}
(the entry and exit basic blocks are instances of a
subclass \javadoc{joeq.Compiler.Quad.EntryOrExitBasicBlock}{joeq/Compiler/Quad/EntryOrExitBsaicBlock.html}).
Components of the basic block can be accessed using the following API
provided by that class.

\begin{mytable}{|l|l|}
\hline
\verb+int size()+ & Number of quads contained in the basic block. \\
\hline
\verb+Quad getQuad(int index)+ & Quad at the given 0-based index. \\
\hline
\verb+List.BasicBlock getPredecessors()+ & List of immediate predecessor basic blocks. \\
\hline
\verb+List.BasicBlock getSuccessors()+ & List of immediate successor basic blocks. \\
\hline
\verb+jq_Method getMethod()+ & Containing method of the basic block. \T \\
\hline
\end{mytable}

\section{Quads}
\label{sec:quads}

Chord uses format \code{offset!mName:mDesc@cName}, described in
class \javadoc{chord.program.MethodElem}{chord/program/MethodElem.html},
to uniquely identify each bytecode instruction in the input Java
program, where \code{offset} is the (0-based) bytecode offset of the
instruction in its containing method, \code{mName} is the name of the
method, \code{mDesc} is the descriptor of the method, and \code{cName}
is the fully-qualified name of the class declaring the method.  For
instance, ``\code{8!main:[Ljava/lang/String;@test.HelloWorld}''
uniquely identifies the return instruction in the main method
in the example above.

The quadcode representation is register-based, as opposed to Java
bytecode that is used to construct it, which is stack-based.  As a
result, it uses {\it quads} to represent bytecode instructions.  A
quad is a primitive statement that consists of an operator and upto
four operands.  There is no one-to-one correspondence between bytecode
instructions and quads: certain bytecode instructions generate a
sequence of more than one quads while others do not generate any quad.
The API of class \code{jq_Method} provides various \code{getQuad(...)}
methods to access the quad(s)
corresponding to a bytecode instruction (see Section \ref{sec:methods}).

Each quad is a unique object of
class \javadoc{joeq.Compiler.Quad.Quad}{joeq/Compiler/Quad/Quad.html}.
Components of the quad can be accessed using the following API
provided by that class.

\begin{mytable}{|l|l|}
\hline
\verb+Operator getOperator()+ & Kind of the quad. \\
\hline
\verb+int getBCI()+ & Bytecode offset of the quad in its containing
method. \\
\hline
\verb+String toByteLocStr()+ & Unique identifier of the quad in
format \code{offset!mName:mDesc@cName}. \\
\hline
\verb+String toJavaLocStr()+ & Location of the quad in
format \code{fileName:lineNum} in Java source code. \\
\hline
\verb+String toLocStr()+ & Location of the quad in both Java bytecode and source code. \\
\hline
\verb+String toVerboseStr()+ & Verbose description of the quad (its location plus contents). \\
\hline
\verb+jq_Method getMethod()+ & Containing method of the quad. \T \\
\hline
\end{mytable}

The kind of each quad is determined by its operator which is a unique object of
the appropriate subclass
of \javadoc{joeq.Compiler.Quad.Operator}{joeq/Compiler/Quad/Operator.html}
in the following hierarchy:

\begin{verbatim}
Operator
    |
    |--- Move
    |--- Phi
    |--- Unary
    |--- Binary
    |--- New
    |--- NewArray
    |--- MultiNewArray
    |--- Getstatic
    |--- Putstatic
    |--- ALoad
    |--- AStore
    |--- Getfield
    |--- Putfield
    |--- CheckCast
    |--- InstanceOf
    |--- ALength
    |--- Return
    |
    |--- Branch
    |       |
    |       |--- IntIfCmp
    |       |--- Goto
    |       |--- Jsr
    |       |--- Ret
    |       |--- LookupSwitch
    |       |--- TableSwitch
    |
    |--- Invoke
    |       |
    |       |--- InvokeVirtual
    |       |--- InvokeStatic
    |       |--- InvokeInterface
    |
    |--- Monitor
            |
            |--- MONITORENTER
            |--- MONITOREXIT
\end{verbatim}

The number and kinds of operands of each quad depends upon the kind of
the operator.  Each of the above subclasses of \code{Operator}
provides an API to access the operands of the quad.  For instance,
the components of a \code{Getfield} quad {\tt q} of the form ``l = b.f'' can be accessed as follows:

\begin{framed}
\begin{verbatim}
Operand lo = Getfield.getDest(q);
Operand bo = Getfield.getBase(q);
if (lo instanceof RegisterOperand && bo instanceof RegisterOperand) {
    Register l = ((RegisterOperand) lo).getRegister();
    Register b = ((RegisterOperand) bo).getRegister();
    jq_Field f = Getfield.getField(q).getField();
    ...
}
\end{verbatim}
\end{framed}

\section{Traversing Quadcode}
\label{sec:traversing}

A common way to traverse all quads in the input Java program is as follows:

\texonly{\newpage}

\begin{framed}
\begin{verbatim}

import chord.program.Program;
import joeq.Compiler.Quad.QuadVisitor;
import joeq.Class.jq_Method;
import joeq.Compiler.Quad.ControlFlowGraph;
import joeq.Util.Templates.ListIterator;
import joeq.Compiler.Quad.BasicBlock;
import joeq.Compiler.Quad.Quad;

QuadVisitor qv = new QuadVisitor.EmptyVisitor() {
    public void visitMove(Quad q) { ... }
    public void visitPhi(Quad q) { ... }
    public void visitUnary(Quad q) { ... }
    ...
};
Program program = Program.g();
for (jq_Method m : program.getMethods()) {
    if (!m.isAbstract()) {
        ControlFlowGraph cfg = m.getCFG();
        ListIterator.BasicBlock it = cfg.reversePostOrderIterator();
        while (it.hasNext()) {
            BasicBlock b = it.nextBasicBlock();
            for (int i = 0; i < b.size(); i++) {
                Quad q = b.getQuad(i);
                q.accept(qv);
            }
        }
    }
}
\end{verbatim}
\end{framed}


\xname{project}
\chapter{A Chord Project: Tasks, Targets, and Dependencies}
\label{chap:project}

In order to facilitate heavy reuse and rapid prototyping, each analysis in Chord is
written modularly, independent of other analyses, along with lightweight annotations
specifying the inputs and outputs of the analysis.
In each run, upon startup, Chord organizes all analyses and their inputs and outputs
(collectively called analysis results) using a global entity
called a {\it project}.  More concretely, a project consists of a set of
analyses called {\it tasks}, a set of analysis results called {\it targets}, and
a set of data/control dependencies between tasks and targets.

The project built in a particular run is of either of the following two kinds,
depending upon whether the value of property \code{chord.classic} is true or false, respectively.
\begin{itemize}
\item
a {\it classic project}, represented as an object of
class \javadoc{chord.project.ClassicProject}{chord/project/ClassicProject.html}.
\item
a {\it modern project}, represented as an object of
class \javadoc{chord.project.ModernProject}{chord/project/ModernProject.html}.
\end{itemize}
The project representation can be obtained by calling static method \code{g()} of the
corresponding class.
A classic project is built by default.
The two kinds of projects differ
primarily in that the only kind of dependencies in a classic
project are data dependencies whereas both data and control dependencies are
allowed in a modern project.  The key advantage of a modern project is that it
can schedule independent tasks in parallel whereas a classic project always
runs tasks sequentially.  This chapter focusses on classic projects as the
runtime for modern projects is still under development.
We next explain how Chord builds a classic project
(a set of tasks, a set of targets, and a set of dependencies between them).

{\bf Tasks:}
There are two kinds of tasks corresponding to the two broad kinds of analyses in
Chord: those written imperatively in Java and those written declaratively in
Datalog.  They are summarized in the following table:

\begin{mytable}{|l||c|c|}
\hline
 Kind: & imperative (see Chapter \ref{chap:writing}) & declarative (see Chapter \ref{chap:datalog}) \\
\hline
 Location: &
	\begin{tabular}{c}
	a {\tt .class} file in the path denoted by property \\
	\code{chord.java.analysis.path} compiled \\
	from a \code{@Chord}-annotated class implementing \\
	interface \javadoc{chord.project.ITask}{chord/project/ITask.html} 
	\end{tabular} &
	\begin{tabular}{c}
	a {\tt .dlog} file in the path denoted by property \\
	\code{chord.dlog.analysis.path}
	\end{tabular} \\
\hline
 Name: & via stmt \verb+name="<NAME>"+ in {\tt @Chord} annotation & via line ``\verb+# name=<NAME>+'' in {\tt .dlog} file \\
\hline
 Form: &
	an instance of the \code{@Chord}-annotated class &
	\begin{tabular}{c}
	an instance of class \\
    \javadoc{chord.project.analyses.DlogAnalysis}{chord/project/analyses/DlogAnalysis.html}
	\end{tabular}
\T \\
\hline
\end{mytable}

Each task in Chord is of the form ``\code{\{ C1, ..., Cn \} T \{ P1, ..., Pm \}}" where:
\begin{itemize}
\item
{\tt T} is the code provided by the user to be executed when the task is executed,
\item
{\tt C1}, ..., {\tt Cn} are the names of zero or more targets specified by the user as being
consumed by the task, and
\item
{\tt P1}, ..., {\tt Pm} are the names of zero or more targets specified by the user as being
produced by the task.
\end{itemize}
The consumed targets may be produced by other tasks and, likewise, the produced
targets may be consumed by other tasks.

{\bf Targets:}
The set of targets in a project includes each target that is specified as
consumed/produced by some task in the project.  When defining tasks, the user implicitly or
explicitly provides the class (type) of each target.
Chord reports a runtime error if a target has no type or has multiple types.
Otherwise, it creates a separate instance of that class to represent that target.

{\bf Dependencies:}
Chord computes a dependency graph as a directed graph whose
nodes are all tasks and targets computed as above, and:
\begin{itemize}
\item
There is an edge from a target C to a task T if the user has specified that T consumes C.
\item
There is an edge from a task T to a target P is the user has specified that T produces P.
\end{itemize}

We next present an example project to illustrate various concepts in the rest of
this chapter:

\begin{framed}
\begin{verbatim}
{} T1 { R1 }
{} T2 { R1 }
{ R4} T3 { R2 }
{ R1, R2 } T4 { R3, R4 }
\end{verbatim}
\end{framed} 

The set of tasks in this project is \{ {\tt T1}, {\tt T2}, {\tt T3}, {\tt T4} \}
and the set of targets in the project is \{ {\tt R1}, {\tt R2}, {\tt R3}, {\tt R4} \}.
The dependency graph is as follows:

\begin{center}
\includeimage{0.3}{dependency_graph.png}
\end{center}

Class \javadoc{chord.project.ClassicProject}{chord/project/ClassicProject.html}
provides a rich API (in the form of public instance methods) for accessing tasks
and targets in the project, for running tasks, and for resetting tasks and
targets.  The most commonly used methods are as follows:

\begin{mytable}{|l|l|}
\hline
\verb+ITask getTask(String name)+ & Representation of the task named {\tt name}. \\
\hline
\verb+Object getTrgt(String name)+ & Representation of the target named {\tt name}. \\
\hline
\verb+ITask runTask(String name)+ & Execute the task named {\tt name}. \\
\hline
\verb+boolean isTaskDone(String name)+ & Whether task named {\tt name} has alread been executed. \\
\hline
\verb+boolean isTrgtDone(String name)+ & Whether target named {\tt name} has already been computed. \\
\hline
\verb+void setTaskDone(String name)+ & Force task named {\tt name} to not be executed
the next time it is demanded. \\
\hline
\verb+void setTrgtDone(String name)+ & Force target named {\tt name} to
not be computed the next time it is demanded.  \\
\hline
\verb+void resetTaskDone(String name)+ & Force task named {\tt name} to be executed
the next time it is demanded. \\
\hline
\verb+void resetTrgtDone(String name)+ & Force target named {\tt name} to
be computed the next time it is demanded.  \T \\
\hline
\end{mytable}

We next explain the above methods.

The \code{getTask(name)} and \code{getTrgt(name)} methods provides the representation
of the unique task or the unique target, respectively, with the specified name, if it exists,
and a runtime error otherwise.

A ``done'' bit, initialized to false, is kept with each task and each target in
the project.  The operation of the remaining methods above involves this bit.

The \code{runTask(name)} method runs the task with the specified name, if it exists,
and reports a runtime error otherwise.  Running a task proceeds as follows.
If the done bit of the task is true, no action is taken.  Otherwise, 
suppose the task is of the form ``\code{\{ C1, ..., Cn \} T \{ P1, ..., Pm \}}''.  Then,
the following two actions are taken in order:

\begin{enumerate}
\item
For each of the consumed targets {\tt C1}, ..., {\tt Cn} 
whose done bit is false, the unique task in the project producing that target 
is run recursively.
A runtime error is reported if no such task exists or if multiple such tasks exist.
\item
Once all consumed targets are computed, the code {\tt T} of this task itself is run.
\item
Finally, the done bit of this task as well as of each of its produced targets
{\tt P1}, ..., {\tt Pn} is set to true.
\end{enumerate}

It is the user's responsibility to ensure termination in the case in which there are
cycles in the dependency graph.  The \code{isTaskDone(name)} and \code{isTrgtDone(name)}
methods can be used in the code {\tt T} of any task to enquire
whether the done bit of the task or target with the specified name is set to true.
Moreover, methods \code{setTaskDone(name)}, \code{setTrgtDone(name)} can be used to set
the done bit of the task or target with the specified name to true, and likewise,
methods \code{resetTaskDone(name)} and \code{resetTrgtDone(name)} can be used
to set the done bit to false.

It is possible to run tasks from the command-line of Chord by specifying the value of property
\code{chord.run.analyses} as a comma-separated list of the names of tasks to be run in order
(see Chapter \ref{chap:running}).

We next illustrate the above concepts using the above example.
Suppose Chord is run with the value of property \code{chord.run.analyses} as ``{\tt T4}''.
This causes \code{runTask(T4)} to be called.  The done bit of task {\tt T4} is initialized to false.
Hence, the done bit of its first consumed target {\tt R1} is checked.
Since it is also initialized to false, the unique task producing
target {\tt R1} is demanded.  However, multiple tasks {\tt T1} and {\tt T2} 
producing target {\tt R1} are found in the project, resulting in a runtime error which
reports the ambiguity between tasks {\tt T1} and {\tt T2}.

To resolve the ambiguity (say in favor of task {\tt T1}), the user can
specify the value of property \code{chord.run.analyses} as ``{\tt T1},{\tt T4}''.  This
time, \code{runTask(T1)} is called followed by \code{runTask(T4)}.
Since the done bit of task {\tt T1} is initialized to false and it has no consumed
targets, \code{runTask(T1)} simply executes the code of task {\tt T1}, and sets
the done bit of task {\tt T1} and of its only produced target {\tt R1} to true.
Next, the call to \code{runTask(T4)} proceeds as described in the previous run
above, but this time the done bit of target {\tt R1} consumed by task {\tt T4} is
set to true.  Hence, the demand for the unique task producing target {\tt R1} (and
the ensuing ambiguity runtime error) is averted.  However, this time a different
problem occurs: the done bit of the other target {\tt R2} consumed by task {\tt T4} is
initialized to false, which results in a call to \code{runTask(T3)}
(since task {\tt T3} is the unique task that produces target {\tt R2}), which in turn
results in a call to \code{runTask(T4)}.  The result is infinite 
mutually-recursive calls to \code{runTask(T4)} and \code{runTask(T3)} unless the
code of task {\tt T3} or {\tt T4} averts it by
calling \code{setTaskDone} or \code{setTrgtDone} on some task or target
in the cycle.
This scenario resulting from a cycle in the dependence graph is rare in practice.
It typically occurs in the case of iterative refinement-based client-driven analyses:
the output of such an analysis in one iteration is fed as an input to the same analysis
in a subsequent iteration.  The code of such an analysis must explicitly
control execution as described above to avert infinite recursion.


\xname{writing}
\chapter{Writing an Analysis}
\label{chap:writing}

Chord provides several {\it analysis templates}: classes containing boilerplate
code that can be extended by users to rapidly prototype different kinds of
analyses.  These classes are organized in the following hierarchy in package
\javadoc{chord.project.analyses}{chord/project/analyses/package-summary.html}:

\begin{verbatim}
JavaAnalysis
    |
    |--- ProgramDom
    |
    |--- ProgramRel
    |
    |--- DlogAnalysis
    |
    |--- RHSAnalysis
    |        |
    |        |--- ForwardRHSAnalysis
    |        |
    |        |--- BackwardRHSAnalysis
    |
    |--- BasicDynamicAnalysis
             |
             |--- DynamicAnalysis
\end{verbatim}

The following sections describe each of these analysis templates in more detail.

\section{JavaAnalysis}
\label{sec:java}

Class \javadoc{chord.project.analyses.JavaAnalysis}{chord/project/analyses/JavaAnalysis.html}
is the most general template for writing an analysis.  An analysis can be
created using this template by extending this class as follows:

\begin{framed}
\begin{verbatim}
import chord.project.Chord;
import chord.project.ClassicProject;
import chord.project.analyses.JavaAnalysis;
import chord.program.Program;

@Chord(
    name = "<ANALYSIS_NAME>",
    consumes = { "C1", ..., "Cn" },
    produces = { "P1", ..., "Pm" },
    namesOfTypes = { "C1", ..., "Cn", "P1", ..., "Pm" },
    types = { A1.class, ..., An.class, B1.class, ..., Bm.class }
)
public class ExampleAnalysis extends JavaAnalysis {
    @Override public void run() {
        Program program = Program.g();
        ClassicProject project = ClassicProject.g();
        A1 c1 = (A1) project.getTrgt("C1");
        ...
        An cn = (An) project.getTrgt("Cn");
        B1 p1 = (B1) project.getTrgt("P1");
        ...
        Bm pm = (Bm) project.getTrgt("Pm");
        // compute produced targets p1, ..., pm from program and
        // consumed targets c1, ..., cn
        ...
    }
}
\end{verbatim}
\end{framed}

To run the analysis, class \code{ExampleAnalysis} must be compiled to a
{\tt .class} file that occurs in some element (directory or jar/zip file) of
the path specified by property \code{chord.java.analysis.path}.  This causes
the analysis to be included in a Chord project as a task that is represented
as a separate object of class \code{ExampleAnalysis}.

The {\tt @Chord} annotation, defined in class
\javadoc{chord.project.Chord}{chord/project/Chord.html}, specifies via fields
the following aspects of the analysis:
\begin{itemize}
\item
Field {\tt name} specifies the name of the analysis (\code{<ANALYSIS_NAME>}).
\item
Field {\tt consumes} specifies the names of targets that are consumed by the
analysis ({\tt C1}, ..., {\tt Cn}).
\item
Field {\tt produces} specifies the names of targets that are produced by the
analysis ({\tt P1}, ..., {\tt Pm}).
\item
Fields {\tt namesOfTypes} and {\tt types} specify the types of targets.  There
is a 1-to-1 correspondence between the arrays denoted by these two fields,
e.g., the type of the target named {\tt C1} is class {\tt A1}, and so on.  In
principle, the type of {\it any} target in the project can be specified here.
In practice, however, the types of only the targets declared as
consumed/produced by this analysis are specified.  Moreover, although the type
of each target that is consumed/produced by the above analysis is specified in
the above annotation, in practice the types of hardly any targets need to be
explicitly specified, because they can be automatically inferred by Chord from
analyses created using more specialized templates discussed below that also
consume/produce those targets.
\end{itemize}

The code of the analysis must be supplied in the {\tt run()} method.  This
method typically does the following in order:
(1) retrieves the program being analyzed and the representation of each
consumed/produced target from the project;
(2) performs some computation that uses the program and the consumed targets
as inputs; and
(3) writes the outputs of the computation to the produced targets.

The analysis templates presented in the following sections are more specialized
forms of the {\tt JavaAnalysis} template: they constrain the number and kinds of
consumed/produced targets and/or the analysis code in the {\tt run()} method.

%called JavaAnalysis (Section \ref{sec:java}), and the other for writing
%analyses declaratively in Datalog, called DlogAnalysis
%(Section \ref{sec:dlog}).  JavaAnalysis is the most general template:
%it least constrains the kind of analysis that can be written but also offers the
%least boilerplate code for writing an analysis.  Chord provides several
%additional templates for writing analyses imperatively in Java that are
%specialized forms of JavaAnalysis, namely, ProgramDom
%(Section \ref{sec:program-dom}), ProgramRel (Section \ref{sec:program-rel}),
%RHSAnalysis (Section \ref{sec:rhs}), and DynamicAnalysis
%(Section \ref{sec:dynamic}).  We next explain each of these templates.

\section{ProgramDom}
\label{sec:program-dom}

Class \javadoc{chord.project.analyses.ProgramDom}{chord/project/analyses/ProgramDom.html}
is a template for writing a {\it program domain analysis}.
A {\it program domain} represents an indexed set of values of a fixed
kind, typically from the program being analyzed, such as the set of all methods
in the program, the set of all fields in the program, etc.  Indices are 
assigned starting from 0 and in the order in which values are added to the set.
A program domain primarily serves as an input to Datalog analyses
(see Section \ref{sec:dlog}).  Thus, it is a kind of target (i.e.,
analysis result) in a Chord project.  A common way to define a program domain
is to create a program domain analysis by extending class {\tt ProgramDom} as
follows:

\begin{framed}
\begin{verbatim}
import chord.project.Chord;
import chord.project.ClassicProject;
import chord.project.analyses.ProgramDom;
import chord.program.Program;

@Chord(
    name = "<DOM_NAME>",
    consumes = { "C1", ..., "Cn" }
)
public class ExampleDom extends ProgramDom<DOM_TYPE> {
    @Override public void fill() {
        Program p = Program.g();
        ClassicProject project = ClassicProject.g();
        A1 c1 = (A1) project.getTrgt("C1");
        ...
        An cn = (An) project.getTrgt("Cn");
        // populate domain using program and consumed targets c1, ..., cn
        for (...) {
            DOM_TYPE e = ...;
            add(e); 
        }
    }
}
\end{verbatim}
\end{framed}

To run the analysis, class \code{ExampleDom} must be compiled to a {\tt .class}
file that occurs in some element (directory or jar/zip file) of the path
specified by property \code{chord.java.analysis.path}.  This causes the
analysis to be included in a Chord project as a task that is represented as a
separate object of class \code{ExampleDom}.  Moreover, that object also denotes
a target in the Chord project.  Both the task and target have the same name
\code{<DOM_NAME>}.

The \code{ProgramDom} template can be viewed as providing the following
specialized form of the general \code{JavaAnalysis} template:

\begin{itemize}
\item
It consumes any number and kinds of targets ({\tt C1}, ..., {\tt Cn}).
\item
It produces a single target, namely, the defined program domain itself
(named \code{<DOM_NAME>} and of type \code{ExampleDom}).
\item
Its \code{run()} method adds values to the defined program domain.  Typically,
it suffices to override the {\tt fill()} method (which is called by the
{\tt run()} method) and call from it the {\tt add(e)} method for each value
{\tt e} to be added to the domain in order.
\end{itemize}

It is a runtime error to explicitly specify any produced targets in the
{\tt @Chord} annotation of a class extending {\tt ProgramDom}.  If you wish to
define an analysis that produces additional targets besides a program domain,
then you can still define the program domain in a class such as
{\tt ExampleDom} that extends {\tt ProgramDom}, but you must not annotate it
with the {\tt @Chord} annotation (since this annotation causes the class to be 
egarded as defining an analysis).  Instead, define the analysis in a separate
class that extends {\tt JavaAnalysis}, as follows:

\begin{framed}
\begin{verbatim}
import chord.project.Chord;
import chord.project.ClassicProject;
import chord.project.analyses.JavaAnalysis;

@Chord(
    name = "<ANALYSIS_NAME>",
    consumes = { "C1", ..., "Cn" },
    produces = { "<DOM_NAME>", ... }
)
public class ExampleAnalysis extends JavaAnalysis {
    @Override public void run() {
        ExampleDom d = (ExampleDom) ClassicProject.g().getTrgt("<DOM_NAME>");
        d.run();  // produce domain named <DOM_NAME>
        ...
    }
}
\end{verbatim}
\end{framed}

Note that targets {\tt C1}, ..., {\tt Cn} that were declared as consumed in
the {\tt @Chord} annotation of class {\tt ExampleDom} (and any other fields
such as {\tt namesOfTypes} and {\tt types}) must now be provided in the
{\tt @Chord} annotation of class \code{ExampleAnalysis}.

\section{ProgramRel}
\label{sec:program-rel}

Class \javadoc{chord.project.analyses.ProgramRel}{chord/project/analyses/ProgramRel.html}
is a template for writing a {\it program relation analysis}.
A {\it program relation} represents a set of tuples over one or more fixed 
program domains.  A program relation primarily serves as an input or output of
Datalog analyses (see Section \ref{sec:dlog}).  Thus, it is a kind of
target (i.e., analysis result) in a Chord project.  A common way to define a
program relation is to create a program domain analysis by extending class
{\tt ProgramRel} as follows:

\begin{framed}
\begin{verbatim}
import chord.project.Chord;
import chord.project.ClassicProject;
import chord.project.analyses.ProgramDom;
import chord.project.analyses.ProgramRel;
import chord.program.Program;

@Chord(
    name = "<REL_NAME>",
    consumes = { "C1", ..., "Cn" },
    sign = "<DOM_NAMES>:<DOM_ORDER>"
)
public class ExampleRel extends ProgramRel {
    @Override public void fill() {
        Program p = Program.g();
        ProgramDom<T1> d1 = doms[0];
        ...
        ProgramDom<Tm> dm = doms[m-1];
        ClassicProject project = ClassicProject.g();
        A1 c1 = (A1) project.getTrgt("C1");
        ...
        An cn = (An) project.getTrgt("Cn");
        // populate relation using program, its domains d1, ..., dm, and
        // consumed targets c1, ..., cn
        for (...) {
            T1 o1 = ...;
            Tm om = ...;
            add(o1, ..., om); 
        }
    }
}
\end{verbatim}
\end{framed}

To run the analysis, class \code{ExampleRel} must be compiled to a {\tt .class}
file that occurs in some element (directory or jar/zip file) of the path
specified by property \code{chord.java.analysis.path}.  This causes the
analysis to be included in a Chord project as a task that is represented as a
separate object of class \code{ExampleRel}.  Moreover, that object also denotes
a target in the Chord project.  Both the task and target have the same name
\code{<REL_NAME>}.

The \code{ProgramRel} template can be viewed as providing the following
specialized form of the general \code{JavaAnalysis} template:

\begin{itemize}
\item
It consumes any number and kinds of targets ({\tt C1}, ..., {\tt Cn}).
\item
It produces a single target, namely, the defined program relation itself
(named \code{<REL_NAME>} and of type \code{ExampleRel}).
\item
Its \code{run()} method adds tuples to the defined program relation.  Typically,
it suffices to override the {\tt fill()} method (which is called by the
{\tt run()} method) and call from it the {\tt add(o1, ..., om)} method
for each tuple ({\tt o1}, ..., {\tt om}) to be added to the relation.
\end{itemize}

Unlike for program domains, the order in which tuples are added to a program
relation is irrelevant.  But the relative ordering of the program domains over
which the program relation is declared matters heavily for performance.
This is because each program relation in Chord is represented symbolically (as
oppoosed to explicitly) using a data structure called a Binary Decision Diagram
(BDD for short).  This in turn is because in pratice, a program relation (e.g.,
one representing context-sensitive points-to information)
can contain millions or billions of tuples even for a moderately-sized
input Java program; representing such a large number of tuples explicitly is
prohibitively and needlessly expensive.  The size of a BDD, on the other hand,
does not depend at all upon the number of tuples in the program relation that
the BDD represents.  Instead, it depends heavily upon the relative ordering
of the program domains over which the program relation is declared.
Hence, the {\tt @Chord} annotation on a class such as {\tt ExampleRel} that
extends \code{ProgramRel} is required to have a {\tt sign} field whose value is
the {\it sign} of the program relation.
A sign is a string of the form \code{<DOM_NAMES>:<DOM_ORDER>} where:

\begin{itemize}

\item

\code{<DOM_NAMES>} is mandatory and specifies the relation's {\it schema}: a comma-separated list of names of
the domains over which the relation is defined, with each domain name suffixed with a
non-negative integer that is typically 0 and must be unique across multiple
occurrences of the same domain name.
The order of domain names in the schema is {\it irrelevant}: users must
pick this order purely based on what order they find most convenient to remember.

For example, suppose the program relation represents the result of Class Hierarchy
Analysis (CHA), i.e., it contains each tuple of the form ($m_1$,$t$,$m_2$) such that
method $m_2$ is the resolved method of an invokevirtual or invokeinterface call
with resolved method $m_1$ on an object of class $t$.  Let {\tt M} and {\tt T}
denote the names of the program domains representing the set of all methods
and the set of all classes, respectively, in the Java program being analyzed.
Then, \code{<DOM_NAMES>} for this program relation could be any of ``{\tt M0,T0,M1}'',
``{\tt M1,T0,M0}'', ``{\tt M0,T1,M1}'', and so on.

\item

\code{<DOM_ORDER>} is optional and determines the relation's representation as a
BDD. It is a permutation of the names in \code{DOM_NAMES>} with a separator
`\_' or `x' between consecutive names.  The order of names in this
list, and the kind of separators used between them, are what determines both
the size of the BDD and the performance of operations on it (such as join,
selection, projection, etc.).

An example value of \code{<DOM_ORDER>} for the
CHA program relation above with \code{<DOM_LIST>} as ``\code{M0,T0,M1}''  is
``\code{M0xM1_T0}''; another example value is ``\code{M1_T0_M0}''.
\end{itemize}

Chapter \ref{chap:datalog} describes how BDDs are represented
(Section \ref{sec:bdd}) and how you can tune their size and the
performance of operations on them (Section \ref{sec:tuning-datalog}).

It is a runtime error to explicitly specify any produced targets in the
{\tt @Chord} annotation of a class extending {\tt ProgramRel}.  If you wish to
define an analysis that produces additional targets besides a program relation,
then you can still define the program relation in a class such as
{\tt ExampleRel} that extends {\tt ProgramRel}, but you must not annotate it
with the {\tt @Chord} annotation (since this annotation causes the class to be 
egarded as defining an analysis).  Instead, define the analysis in a separate
class that extends {\tt JavaAnalysis}, as follows:

\begin{framed}
\begin{verbatim}
import chord.project.Chord;
import chord.project.ClassicProject;
import chord.project.analyses.JavaAnalysis;

@Chord(
    name = "<ANALYSIS_NAME>",
    consumes = { "C1", ..., "Cn" },
    produces = { "<REL_NAME>", ... },
    namesOfSigns = { "<REL_NAME>", ... },
    signs = { "<DOM_ORDER>:<DOM_NAME>", ... }
)
public class ExampleAnalysis extends JavaAnalysis {
    @Override public void run() {
        ExampleRel r = (ExampleRel) ClassicProject.g().getTrgt("<REL_NAME>");
        r.run();  // produce program relation named <REL_NAME>
    }
}
\end{verbatim}
\end{framed}

Note that targets {\tt C1}, ..., {\tt Cn} that were declared as consumed in
the {\tt @Chord} annotation of class {\tt ExampleRel} (and any other fields
such as {\tt namesOfTypes} and {\tt types}) must now be provided in the
{\tt @Chord} annotation of class \code{ExampleAnalysis}.  Just like the
1-to-1 correspondence between the values of fields {\tt namesOfTypes} and
{\tt types}, there is a 1-to-1 correspondence between the values of fields
{\tt namesOfSigns} and {\tt signs}, which allow relating the name of any program
relation in the project with its sign.

\section{DlogAnalysis}
\label{sec:dlog}

A common way to rapidly prototype analyses in Chord is using a declarative
logic-programming language called Datalog.  A Datalog analysis is defined in a
{\tt .dlog} file that primarily specifies the following:
\begin{itemize}
\item
A set of {\it program domains} {\tt D1}, ..., {\tt Dk}.  A program domain is a
set of values of a fixed kind.  Each program domain in a Chord project is
a target that is represented as a separate object of class \code{ProgramDom}
(see Section \ref{sec:program-dom}).
\item
A set of {\it program relations}, including input relations {\tt I1}, ..., {\tt In}
and output relations {\tt O1}, ..., {\tt Om}.  A program relation is a set of
tuples over one or more fixed program domains {\tt D1}, ..., {\tt Dk}.  Each program
relation in a Chord project is a target that is represented as a separate object
of class \code{ProgramRel} (see Section \ref{sec:program-rel}).
\item
A set of rules (constraints) {\tt R} specifying how to compute the output
relations from the input relations.
\end{itemize}
An example such file is shown in Chapter \ref{chap:datalog} which also
explains all aspects of Datalog analyses.

To run the analysis, the {\tt .dlog} file must occur in some element (directory
or jar/zip file) of the path specified by property \code{chord.dlog.analysis.path}.
 This causes the analysis to be included in a Chord project as a task that is
represented as a separate object of class
\javadoc{chord.project.analyses.DlogAnalysis}{chord/project/analyses/DlogAnalysis.html}.
Note that, unlike class {\tt JavaAnalysis}, users must not extend class
{\tt DlogAnalysis} but must instead define the analysis in a {\tt .dlog} file.

The \code{DlogAnalysis} template can be viewed as providing the following
specialized form of the general \code{JavaAnalysis} template:
\begin{itemize}
\item
It consumes targets {\tt D1}, ..., {\tt Dk} of type {\tt ProgramDom} and
{\tt I1}, ..., {\tt In} of type {\tt ProgramRel}.
\item
It produces targets {\tt O1}, ..., {\tt Om} of type {\tt ProgramRel}.
\item
Its \code{run()} method executes Datalog solver \code{bddbddb} to compute output
relations {\tt O1}, ..., {\tt Om} from input relations {\tt I1}, ..., {\tt In}
by solving constraints {\tt R}.
\end{itemize}

\section{DynamicAnalysis}
\label{sec:dynamic}

\section{RHSAnalysis}
\label{sec:rhs}

%\begin{itemize}
%\item
%B is the domain of basic blocks.
%\item
%C is the domain of abstract calling contexts.
%\item
%E is the domain of program points that read or write a field.
%\item
%F is the domain of fields inside a class.
%\item
%H is the domain of allocation sites.
%\item
%I is the domain of invocation statements (method calls)
%\item
%L is the domain of lock acquisition points.
%\item
%M is the domain of program methods.
%\item
%P is the domain of simple statements. %SAY MORE HERE?
%\item
%R is the domain of lock release points.
%\item
%T is the domain of types (classes and primitives).
%\item
%V is the domain of reference-typed variables.
%\item
%W is the domain of loop head statements. %SAY MORE HERE?
%\item
%Z is a domain of integers corresponding to argument positions.The size of the domain is chosen to 
%accommodate the method with the greatest number of arguments.
%\end{itemize}

%\section{Core relations}

%Chord includes code to build a very large number of relations. These standard relations are very useful in writing your own analyses.  For example, relation \texttt{sub} includes all pairs of types \texttt{(t1, t2)} where \texttt{t2} is a subclass of \texttt{t1}.  Most of these standard relations are in the package \texttt{chord.rels}.

%A Java analysis should extend JavaAnalysis or ProgramRel.  It's usually easiest if an analysis only produces one relation, in which case extending ProgramRel is the best approach.  If you extend ProgramRel, all the work of creating and saving the relation is taken care of for you; all you need to do is add tuples to the relation.Typically you do this by calling \texttt{super.add(...)} with the tuple you wish to add to the relation.  For forther convenience, you can implement one of the Visitor interfaces defined in \texttt{chord.program.visitors}.  This will force you to implement some \texttt{visit} methods that iterate over all elements in some domain, such as all methods or all invocation points. 

%Sometimes, you need to create two different relations at the same time. In this case, the right approach is to extend JavaAnalysis, and explicitly create and save the relations on your own.

%Regardless of which of these approaches you take, every Java analysis must have an \texttt{@Chord} annotation.  This annotation is a set of named fields.  At a minimum, this annotation must specify the name of the analysis.  If you extend ProgramRel, the name of the analysis will be treated as the same as that of the generated relation.  For example, \texttt{@Chord(name = "F")} is a valid annotation.  The name of the Java class that generates the relation is ignored by Chord -- only the annotation matters.

%\textbf{MAYUR, IS SIGN MANDATORY?   IF NOT, WHEN CAN IT BE OMITTED?}

%There are also other annotation fields available.  You may specify the dependencies of your analysis via \texttt{consumes = \{ "xyz"\}}.

%If your analysis does not extend ProgramRel, you should also specify the relations that your analysis produces. You do this via the \texttt{produces} relation.

%Should mention the signs fields.


\xname{running}
\chapter{Running an Analysis}
\label{chap:running}

Chord provides many standard analyses.  Moreover, it allows users to define
their own analyses, possibly atop the predefined analyses. This chapter describes
how to run predefined as well as user-defined analyses.

Broadly, there are two kinds of analyses in Chord:

\begin{mytable}{|l||c|c|}
\hline
Kind:
	& imperative (see Chapter \ref{chap:writing})
	& declarative (see Chapter \ref{chap:datalog}) \\
\hline
Location:
	& \begin{tabular}{c}
	  a {\tt .class} file in the path denoted by property \\
      \code{chord.java.analysis.path} compiled \\
      from a \code{@Chord}-annotated class implementing \\
      interface \javadoc{chord.project.ITask}{chord/project/ITask.html}
      \end{tabular}
	& \begin{tabular}{c}
      a {\tt .dlog} file in the path denoted by property \\
      \code{chord.dlog.analysis.path}
      \end{tabular} \\
\hline
Name:
	& via stmt \verb+name="<NAME>"+ in {\tt @Chord} annotation
	& via line ``\verb+# name=<NAME>+'' in {\tt .dlog} file \T \\
\hline
\end{mytable}

In its most general form, the command for running an analysis is as follows:

\begin{framed}
\begin{verbatim}
ant -Dchord.work.dir=<WORK_DIR> -Dchord.run.analyses=<ANALYSIS_NAME> \
    -Dchord.dlog.analysis.path=<DLOG_ANALYSIS_PATH> \
    -Dchord.java.analysis.path=<JAVA_ANALYSIS_PATH> run
\end{verbatim}
\end{framed}

where:
\begin{itemize}
\item
\code{<WORK_DIR>} is a directory containing a file named
\code{chord.properties} that defines various properties of the program to be
analyzed that might be needed by the analysis being run, such as the program's
main class, the program's application classpath, etc.
(see Chapter \ref{chap:setup}).
\item
\code{<ANALYSIS_NAME>} is the name of the analysis to run.  More generally, it
can be a comma-separated list of names of analyses to run in order.
\item
\code{<JAVA_ANALYSIS_PATH>} is a path specifying all imperative analyses that
might be needed to run the desired analysis.
\item
\code{<DLOG_ANALYSIS_PATH>} is a path specifying all declarative analyses that
might be needed to run the desired analysis.
\end{itemize}

In order to facilitate heavy reuse and rapid prototyping, each analysis in
Chord is written modularly, independent of other analyses, along with
lightweight annotations specifying the inputs and outputs of the analysis.
Chord's runtime automatically computes producer-consumer relationships between
analyses (e.g., determines which analysis produces as output a result that is
needed as input by another analysis).  Before running the desired analysis
named \code{<ANALYSIS_NAME>}, Chord recursively runs other analyses until the
inputs to the desired analysis have been computed; it finally runs the desired
analysis to produce the outputs of that analysis.  Chapter \ref{chap:project}
explains this process in detail.  Each of these analyses must occur in the
path denoted by \code{<JAVA_ANALYSIS_PATH>} or \code{<DLOG_ANALYSIS_PATH>},
depending upon whether the analysis is written imperatively or declaratively,
respectively.

Chord provides shorthand ways for specifying analysis paths by means of the
following six properties:

\begin{mytable}{|c|l|l|l|}
\hline
Analysis Kind
	& \multicolumn{1}{c|}{Predefined}
	& \multicolumn{1}{c|}{User-defined}
	& \multicolumn{1}{c|}{All} \\
\hline
imperative
	& \code{chord.std.java.analysis.path}
	& \code{chord.ext.java.analysis.path}
	& \code{chord.java.analysis.path} \\
\hline
declarative
	& \code{chord.std.dlog.analysis.path}
	& \code{chord.ext.dlog.analysis.path}
	& \code{chord.dlog.analysis.path}
\T \\
\hline
\end{mytable}

The paths specified by the \code{chord.std.*.analysis.path} properties
conventionally include all ``standard'' analyses: analyses that are predefined
in Chord.  The default value of each of these properties is the absolute path
of file \code{chord.jar}.  Normally, users must not change the value of these
properties.

The paths specified by the \code{chord.ext.*.analysis.path} properties
conventionally include all ``external'' analyses: analyses that are written by
users.  The default value of each of these properties is the empty path.

The paths specified by the \code{chord.*.analysis.path} properties include
{\it all} analyses: both standard and external ones.  The default value of each
of these properties is simply the concatenation of the values of the
corresponding two properties above that specify the paths of standard and
external analyses.  Normally, users must not change the value of these
properties.

Running the above command can cause Chord to report a runtime error in the
following scenarios:

\begin{itemize}
\item
Either no included analysis or multiple included analyses are named
\code{<ANALYSIS_NAME>}.
\item
A result declared as consumed by the analysis named \code{<ANALYSIS_NAME>}
(or by some analysis on which the specified analysis is dependent directly or
transitively) is either not declared as produced by any included analysis or
is declared as produced by multiple included analyses.
\end{itemize}

To fix the error resulting from the ``missing analysis'' case in both the above
scenarios, simply include the missing analysis in the path specified by
properties \code{chord.*.analysis.path}.

To fix the error resulting from the ``ambiguous analysis'' case in both the
above scenarios, the names {\tt A1}, ..., {\tt An} of all analyses that were
involved in the ambiguity are provided in the error report.  Suppose {\tt Ai}
is the desired analysis from this set.  Then, one way to fix the error is to
exclude all analyses in that set except analysis {\tt Ai} from the path
specified by properties \code{chord.*.analysis.path}.  A better way is to
specify the names of {\it multiple} analyses in the value of property
\code{chord.run.analyses} (recall that this property allows a comma-separated
list of names of analyses to be run in order).  Specifically, the value of
this property must specify the name of the desired analysis {\tt Ai}
{\it before} the name of the analysis that caused the ambiguity error.

The above command is for users who have defined their own analyses and wish
to run them.  The following simpler command suffices for users who only want
to run analyses predefined in Chord (it uses the default values for
properties \code{chord.*.analysis.path}):

\begin{framed}
\begin{verbatim}
ant -Dchord.work.dir=<WORK_DIR> -Dchord.run.analyses=<ANALYSIS_NAME> run
\end{verbatim}
\end{framed}

For instance, to run a basic may-alias and call-graph analysis (called 0CFA),
run the following command:

\begin{framed}
\begin{verbatim}
ant -Dchord.work.dir=<WORK_DIR> -Dchord.run.analyses=cipa-0cfa-dlog run
\end{verbatim}
\end{framed}

This instructs Chord to run the declarative analysis named \code{cipa-0cfa-dlog},
which is defined in file \code{main/src/chord/analyses/alias/cipa_0cfa.dlog}.

The output of the above command is of the form:

\begin{framed}
{\small
\begin{verbatim}
Buildfile: build.xml

run:
     [java] Chord run initiated at: Mar 13, 2011 10:31:08 PM
     [java] ENTER: cipa-0cfa-dlog
     [java] ENTER: T
     [java] ENTER: RTA
     [java] Iteration: 0
     [java] Iteration: 1
     [java] Iteration: 2
     [java] LEAVE: RTA
     [java] SAVING dom T size: 1386
     [java] LEAVE: T
     [java] ENTER: F
     [java] SAVING dom F size: 4120
     [java] LEAVE: F
     ...
     [java] ENTER: MputStatFldInst
     [java] SAVING rel MputStatFldInst size: 739
     [java] LEAVE: MputStatFldInst
     [java] ENTER: statIM
     [java] SAVING rel statIM size: 3359
     [java] LEAVE: statIM
     [java] Starting command: 'java ... chord_analyses_alias_cipa_0cfa.dlog'
     [java] Relation VH: 541 nodes, 449.0 elements (V0,H0)
     [java] Relation FH: 137 nodes, 8.0 elements (H0,F0)
     [java] Relation HFH: 199 nodes, 35.0 elements (H0,F0,H1)
     [java] Relation IM: 590 nodes, 69.0 elements (I0,M0)
     [java] Finished command: 'java ... chord_analyses_alias_cipa_0cfa.dlog'
     [java] LEAVE: cipa-0cfa-dlog
     [java] Chord run completed at: Mar 13, 2011 10:31:36 PM
     [java] Total time: 00:00:27:671 hh:mm:ss:ms

BUILD SUCCESSFUL
Total time: 28 seconds
\end{verbatim}
}
\end{framed}

The 0CFA analysis consumes and produces multiple {\it program relations}.
The consumed program relations include \code{MputStatFldInst} and
\code{statIM}, each of which is produced by a separate imperative analysis
with the corresponding name, and the produced program relations include
\code{VH}, \code{FH}, \code{HFH}, and \code{IM}.  We next briefly discuss
these relations.

The program relations consumed by the 0CFA analysis contain basic program
facts.  For instance, \code{MputStatFldInst} is a relation containing
each tuple ($m$,$f$,$v$) such that method $m$ in the input Java program
contains a putstatic instruction of the form ``$f$ = $v$", while
\code{statIM} is a relation containing each tuple ($i$,$m$) such that $m$
is the target method of invokestatic instruction $i$.

The program relations produced by the 0CFA analysis represent points-to
information and the call graph of the input Java program as computed by
the analysis.  Specifically, relations \code{VH}, \code{FH}, and \code{HFH}
represent points-to information for local variables, static fields, and
instance fields, respectively, while relation \code{IM} represents the call
graph, namely, it contain each tuple ($i$,$m$) such that $m$ is a possible
target method of call site $i$.

Metavariables $m$, $f$, $i$, and $v$ above range over entities in so-called
{\it program domains} \code{M}, \code{F}, \code{I}, and \code{V}, respectively.
A program domain is a set of entities of a certain kind in the input Java
program.  For instance, \code{M} is the domain representing the set of all
methods in the input Java program, \code{F} is the domain representing the set
of all fields, \code{I} is the domain representing the set of all call sites
in methods in \code{M}, and \code{V} is the domain representing the set of all
local variables of reference type in methods in \code{M}.  Each of these
domains is produced by a separate Java analysis with the corresponding name.
Notice that the analyses producing these domains run upfront because these
domains are consumed by the analyses that produce relations such as
\code{MputStatFldInst} and \code{statIM}, which in turn are consumed by the
desired 0CFA analysis.

During execution, Chord dumps intermediate and final results to files in the
directory specified by property \code{chord.out.dir}, whose default value is
\code{[chord.work.dir]/chord_output/} and typically does not need to be changed
by users.  For the above example, this directory is
\code{<WORK_DIR>/chord_output/}.

The verbosity of Chord's output above is controlled by property
\code{chord.verbose}, whose default value is 1.  At verbosity level 0, the
above command produces less voluminous output of the form:

\begin{framed}
{\small
\begin{verbatim}
Buildfile: build.xml

run:
     [java] Chord run initiated at: Mar 13, 2011 10:35:01 PM
     [java] Chord run completed at: Mar 13, 2011 10:35:28 PM
     [java] Total time: 00:00:26:297 hh:mm:ss:ms

BUILD SUCCESSFUL
Total time: 26 seconds
\end{verbatim}
}
\end{framed}


\xname{dynamic}
\chapter{Dynamic Analysis}
\label{chap:dynamic}

This chapter describes all aspect of dynamic analysis in Chord.  Section
\ref{sec:writing-dynamic} describes how to write a dynamic analysis, Section
\ref{sec:running-dynamic} describes how to compile and run it, and Section
\ref{sec:instr-events} describes common dynamic analysis events supported in
Chord.

\section{Writing a Dynamic Analysis}
\label{sec:writing-dynamic}

Follow the following steps to write your own dynamic analysis.

Create a class extending \code{chord.project.analyses.DynamicAnalysis} and override
the appropriate methods in it.
The only methods that must be compulsorily overridden are method \code{getInstrScheme()},
which must return an instance of the ``instrumentation scheme" required by
your dynamic analysis (i.e., the kind and format of events to be generated during an
instrumented program's execution)
plus each \code{process<event>(<args>)} method that corresponds to event {\tt <event>}
with format {\tt <args>} enabled by the chosen instrumentation scheme.
See Section \ref{sec:instr-events} for the kinds of supported events and their formats.

A sample such class \code{MyDynamicAnalysis} is shown below:

\texonly{\newpage}

\begin{framed}
{\small
\begin{verbatim}
    import chord.project.Chord;
    import chord.project.analyses.DynamicAnalysis;
    import chord.instr.InstrScheme;

    // ***TODO***: analysis won't be recognized by Chord without this annotation
    @Chord(name = "<ANALYSIS_NAME>")    
    public class MyDynamicAnalysis extends DynamicAnalysis {
        InstrScheme scheme;
        @Override
        public InstrScheme getInstrScheme() {
            if (scheme != null)
                return scheme;
            scheme = new InstrScheme();
            // ***TODO***: Choose (<event1>, <args1>), ... (<eventN>, <argsN>)
            // depending upon the kind and format of events required by this
            // dynamic analysis to be generated for this during an instrumented
            // program's execution.
            scheme.set<event1>(<args1>);
            ...
            scheme.set<eventN>(<argsN>);
            return scheme;
        }
        @Override
        public void initAllPasses() {
            // ***TODO***: User code to be executed once and for all
            // before all instrumented program runs start.
        }
        @Override
        public void doneAllPasses() {
            // ***TODO***: User code to be executed once and for all
            // after all instrumented program runs finish.
        }
        @Override
        public void initPass() {
            // ***TODO***: User code to be executed once before each instrumented program run starts.
        }
        @Override
        public void donePass() {
            // ***TODO***: User code to be executed once after each instrumented program run finishes.
        }
        @Override
        public void process<event1>(<args1>) {
            // ***TODO***: User code for handling events of kind <event1> with format <args1>.
        }
        ...
        @Override
        public void process<eventN>(<argsN>) {
            // ***TODO***: User code for handling events of kind <eventN> with format <argsN>.
        }
    }
\end{verbatim}
}
\end{framed}

\section{Compiling and Running a Dynamic Analysis}
\label{sec:running-dynamic}

Compile the analysis by placing the directory containing class \code{MyDynamicAnalysis} created
above in the path defined by property \code{chord.ext.java.analysis.path}.

Provide the IDs of program runs to be generated (say 1, 2, ..., M) and the command-line arguments to be
used for the program in each of those runs (say \code{<args1>}, ..., \code{<argsM>}) via properties
\code{chord.run.ids=1,2,...,N} and \code{chord.args.1=<args1>}, ..., \code{chord.args.M=<argsM>}.
By default, \code{chord.run.ids=0} and \code{chord.args.0=""}, that is, the program will be run only
once (using run ID 0) with no command-line arguments.

To run the analysis, set property \code{chord.run.analyses} to
\code{<ANALYSIS_NAME>} (recall that \code{<ANALYSIS_NAME>} is the name provided
in the \code{@Chord} annotation for class \code{MyDynamicAnalysis} created
above).

{\bf Note:} The IBM J9 JVM on Linux is highly recommended if you intend to use Chord for
dynamic program analysis, as it allows you to instrument the entire JDK; using any other
platform will likely require excluding large parts of the JDK from being instrumented.
Additionally, if you intend to use online (load-time) bytecode instrumentation in
your dynamic program analysis, then you will need JDK 6 or higher, since this functionality
requires the \code{java.lang.instrument} API with class retransformation support (the latter
support is available only in JDK 6 and higher).

You can change the default values of various properties for configuring your dynamic analysis;
see Section \ref{sec:scope-props} and Section \ref{sec:instr-props} in Chapter \ref{chap:properties}.
For instance:
\begin{itemize}
\item
You can set property \code{chord.scope.kind} to {\tt dynamic} so that the program scope
is computed dynamically (i.e., by running the program) instead of statically.
\item
You can exclude certain classes (e.g., JDK classes) from being instrumented by setting
properties \code{chord.std.scope.exclude}, \code{chord.ext.scope.exclude}, and
\code{chord.scope.exclude}.
\item
You can choose between online (i.e. load-time) and offline bytecode instrumentation by
setting property \code{chord.instr.kind} to {\tt online} or {\tt offline}.
\item
You can require the event-generating and event-handling JVMs to be one and the same (by setting
property \code{chord.trace.kind} to {\tt none}),
or to be separate (by setting property {\tt chord.trace.kind} to {\tt full} or {\tt pipe}, depending upon
whether you want the two JVMs to exchange events by a regular file or a POSIX pipe, respectively).
Using a single JVM can cause correctness/performance issues if event-handling Java code itself is instrumented
(e.g., say the event-handling code uses class \code{java.util.ArrayList} which is not excluded from program scope).
Using separate JVMs prevents such issues since the event-handling JVM runs uninstrumented bytecode (only
the event-generating JVM runs instrumented bytecode).
If a regular file is used to exchange events between the two JVMs, then the JVMs run serially:
the event-generating JVM first runs to completion, dumps the entire dynamic trace to the regular file,
and then the event-handling JVM processes the dynamic trace.
If a POSIX pipe is used to exchange events between the two JVMs, then the JVMs run in lockstep.
Obviously, a pipe is more efficient for very long traces, but it 
not portable (e.g., it does not currently work on Windows/Cygwin), and the traces cannot be reused
across Chord runs (see the following item).
\item
You can reuse dynamic traces from a previous Chord run by setting property
\code{chord.reuse.traces} to {\tt true}.  In this case, you must also set property
{\tt chord.trace.kind} to {\tt full}.
\item
You can set property \code{chord.dynamic.haltonerr} to {\tt false} to prevent Chord from terminating
even if the program on which dynamic analysis is being performed crashes.
\end{itemize}

Chord offers much more flexibility in crafting dynamic analyses.
You can define your own instrumentor (by subclassing \code{chord.instr.CoreInstrumentor}
instead of using the default \code{chord.instr.Instrumentor}) and your own event handler (by subclassing \code{chord.runtime.CoreEventHandler}
instead of using the default \code{chord.runtime.EventHandler}).
You can ask the dynamic analysis to use your custom instrumentor and/or your custom event handler by overriding
methods \code{getInstrumentor()} and \code{getEventHandler()}, respectively, defined in \code{chord.project.analyses.CoreDynamicAnalysis}.
Finally, you can define your own dynamic analysis template by subclassing \code{chord.project.analyses.CoreDynamicAnalysis}
instead of subclassing the default \code{chord.project.analyses.DynamicAnalysis}.

\section{Common Dynamic Analysis Events}
\label{sec:instr-events}

Chord provides support for instrumenting common dynamic analysis events.
The below table describes these events.

\begin{mytable}{|l|p{4.3in}|} 
\hline
{\bf Event Kind} & {\bf Description} \\
\hline
EnterMainMethod(\bt) & After thread \bt\ enters method \bm\ (in domain M).
\\
\hline
EnterMethod(\bm, \bt) & After thread \bt\ enters method \bm\ (in domain M).
\\
\hline
LeaveMethod(\bm, \bt) & Before thread \bt\ leaves method \bm\ (in domain M).
\\
\hline
EnterLoop(\bb, \bt) & Before thread \bt\ begins loop \bb\ (in domain B).
\\
\hline
LoopIteration(\bb, \bt) & Before thread \bt\ starts a new iteration of loop \bb\ (in domain B).
\\
\hline
LeaveLoop(\bb, \bt) & After thread \bt\ finishes loop \bb\ (in domain B).
\\
\hline
BasicBlock(\bb, \bt) & Before thread \bt\ enters basic block \bb\ (in domain B).
\\
\hline
Quad(\bp, \bt) & Before thread \bt\ executes quad at program point \bp\ (in domain P).
\\
\hline
BefMethodCall(\bi, \bt, \bo) & 
\begin{tabular}{p{4.3in}}
Before thread \bt\ executes the method invocation statement at program point \bi\ (in domain I) with this argument as object \bo. \\
{\bf Note:} Not generated before constructor calls; use BefNew event.
\end{tabular}
\\
\hline
AftMethodCall(\bi, \bt, \bo) &
\begin{tabular}{p{4.3in}}
After thread \bt\ executes the method invocation statement at program point \bi\ (in domain I) with this argument as object \bo. \\
{\bf Note:} Not generated after constructor calls; use AftNew event.
\end{tabular}
\\
\hline
BefNew(\bh, \bt, \bo) & Before thread \bt\ executes a \code{new} bytecode instruction and allocates fresh object \bo\ at program point \bh\ (in domain H).
\\
\hline
AftNew(\bh, \bt, \bo) & After thread \bt\ executes a \code{new} bytecode instruction and allocates fresh object \bo\ at program point \bh\ (in domain H).
\\
\hline
NewArray(\bh, \bt, \bo) & After thread \bt\ executes a \code{newarray} bytecode instruction and allocates fresh object \bo\ at program point \bh\ (in domain H).
\\
\hline
GetstaticPrimitive(\be, \bt, \bg) & After thread \bt\ reads primitive-typed static field \bg\ (in domain F) at program point \be\ (in domain E).
\\
\hline
GetstaticReference(\be, \bt, \bg, \bo) & After thread \bt\ reads object \bo\ from reference-typed static field \bg\ (in domain F) at program point \be\ (in domain E).
\\
\hline
PutstaticPrimitive(\be, \bt, \bg) & After thread \bt\ writes primitive-typed static field \bg\ (in domain F) at program point \be\ (in domain E).
\\
\hline
PutstaticReference(\be, \bt, \bg, \bo) & After thread \bt\ writes object \bo\ to reference-typed static field \bg\ (in domain F) at program point \be\ (in domain E).
\\
\hline
GetfieldPrimitive(\be, \bt, \bb, \bg) & After thread \bt\ reads primitive-typed instance field \bg\ (in domain F) of object \bb\ at program point \be\ (in domain E).
\\
\hline
GetfieldReference(\be, \bt, \bb, \bg, \bo) & After thread \bt\ reads object \bo\ from reference-typed instance field \bg\ (in domain F) of object \bb\ at program point \be\ (in domain E).
\\
\hline
PutfieldPrimitive(\be, \bt, \bb, \bg) & After thread \bt\ writes primitive-typed instance field \bg\ (in domain F) of object \bb\ at program point \be\ (in domain E).
\\
\hline
PutfieldReference(\be, \bt, \bb, \bg, \bo) & After thread \bt\ writes object \bo\ to reference-typed instance field \bg\ (in domain F) of object \bb\ at program point \be\ (in domain E).
\\
\hline
AloadPrimitive(\be, \bt, \bb, \bi) & After thread \bt\ reads the primitive-typed element at index \bi\ of array object \bb\ at program point \be\ (in domain E).
\\
\hline
AloadReference(\be, \bt, \bb, \bi, \bo) & After thread \bt\ reads object \bo\ from the reference-typed element at index \bi\ of array object \bb\ at program point \be\ (in domain E).
\\
\hline
AstorePrimitive(\be, \bt, \bb, \bi) & After thread \bt\ writes the primitive-typed element at index \bi\ of array object \bb\ at program point \be\ (in domain E).
\\
\hline
AstoreReference(\be, \bt, \bb, \bi, \bo) & After thread \bt\ writes object \bo\ to the reference-typed element at index \bi\ of array object \bb\ at program point \be\ (in domain E).
\\
\hline
%ReturnPrimitive(\bp, \bt) & Not yet supported.
%\\
%ReturnReference(\bp, \bt) & Not yet supported.
%\\
%ExplicitThrow(\bp, \bt, \bo) & Not yet supported.
%\\
%ImplicitThrow(\bp, \bt, \bo) & Not yet supported.
%\\
ThreadStart(\bi, \bt, \bo) & Before thread \bt\ calls the \code{start()} method of \code{java.lang.Thread} at program point \bi\ (in domain I) and spawns a thread \bo.
\\
\hline
ThreadJoin(\bi, \bt, \bo) & Before thread \bt\ calls the \code{join()} method of \code{java.lang.Thread} at program point \bi\ (in domain I) to join with thread \bo.
\\
\hline
AcquireLock(\bl, \bt, \bo) & After thread \bt\ executes a statement of the form ``monitorenter \bo'' or enters a method synchronized on \bo\ at program point \bl\ (in domain L).
\\
\hline
ReleaseLock(\br, \bt, \bo) & Before thread \bt\ executes a statement of the form ``monitorexit \bo'' or leaves a method synchronized on \bo\ at program point \br\ (in domain R).
\\
\hline
Wait(\bi, \bt, \bo) & Before thread \bt\ calls the \code{wait()} method of \code{java.lang.Object} on object \bo\ at program point \bi\ (in domain I).
\\
\hline
Notify(\bi, \bt, \bo) & Before thread \bt\ calls the \code{notify()} or \code{notifyAll()} method of \code{java.lang.Object} on object \bo\ at program point \bi\ (in domain I).
\T \\
\hline
\end{mytable}


\xname{datalog}
\chapter{Datalog Analysis}
\label{chap:datalog}

A common way to rapidly prototype an analysis in Chord is using a declarative
logic-programming language called Datalog.  This chapter describes all aspects
of Datalog analyses in Chord.  Section \ref{sec:writing-datalog} explains how to
write and run a Datalog analysis.  Section \ref{sec:tuning-datalog} explains how
to tune its performance.  Finally, Section \ref{sec:bdd} explains the
representation of BDDs (Binary Decision Diagrams) which are a data structure
central to executing Datalog analyses.

\section{Writing a Datalog Analysis}
\label{sec:writing-datalog}

A Datalog analysis declares a bunch of input/output program relations, each over
one or more program domains, and provides a bunch of rules (constraints)
specifying how to compute the output relations from the input relations.  It can
be defined in any file with suffix {\tt .dlog} in any directory in the path
specified by property \code{chord.dlog.analysis.path}.  An example Datalog
analysis is shown below:

\texonly{\newpage}

\begin{framed}
{\small
\begin{verbatim}
    # name=datarace-dlog

    # Program domains
    .include "E.dom"
    .include "F.dom"
    .include "T.dom"

    # BDD variable order
    .bddvarorder E0xE1_T0_T1_F0

    # Input/intermediate/output program relations
    field(e:E0,f:F0) input
    write(e:E0) input
    reach(t:T0,e:E0) input
    alias(e1:E0,e2:E1) input
    escape(e:E0) input
    unguarded(t1:T0,e1:E0,t2:T1,e2:E1) input
    hasWrite(e1:E0,e2:E1)
    candidate(e1:E0,e2:E1) 
    datarace(t1:T0,e1:E0, t2:T1,e2:E1) output

    # Analysis constraints
    hasWrite(e1,e2) :- write(e1).
    hasWrite(e1,e2) :- write(e2).
    candidate(e1,e2) :- field(e1,f), field(e2,f), hasWrite(e1,e2), e1 <= e2.
    datarace(t1,e1,t2,e2) :- candidate(e1,e2), reach(t1,e1), reach(t2,e2), \
        alias(e1,e2), escape(e1), escape(e2), unguarded(t1,e1,t2,e2).
\end{verbatim}
}
\end{framed}

Any line that begins with a ``{\tt \#}" is regarded a comment, except a
line of the form ``\code{# name=<ANALYSIS_NAME>}'', which specifies the name
\code{<ANALYSIS_NAME>} of the Datalog analysis.
Each Datalog analysis is expected to have exactly one such line.
The above Datalog analysis is named {\tt datarace-dlog}.

The ``\code{.include "<DOM_NAME>.dom"}'' lines specify each domain named
\code{<DOM_NAME>} that is needed by the Datalog analysis, i.e., each domain over which
any relation that is input/output by the Datalog analysis is defined. 
The declaration of each such relation specifies the relation's schema:
all the domains over which the relation is defined.
If the same domain appears multiple times in a relation's schema then
contiguous integers starting from 0 must be used to distinguish them; for instance,
in the above example, {\tt candidate} is a binary relation, both of whose
attributes have domain {\tt E}, and they are distinguished as {\tt E0} and {\tt E1}.

Each relation is represented symbolically (as opposed to explicitly)
using a graph-based data structure called a Binary Decision Diagram (BDD for short).
Each domain containing N elements is assigned log2(N) BDD variables.
The size of a BDD and the efficiency of operations on it depends heavily
on the order of these BDD variables.
The ``\code{.bddvarorder <BDD_VAR_ORDER>}'' line in the Datalog analysis enables the Datalog
analysis writer to specify this order.
It must list all domains along with their numerical suffixes, separated
by `{\tt \_}' or `{\tt x}'.
Using a `{\tt \_}' between two domains, such as {\tt T0\_T1}, means that the BDD variables assigned
to domain {\tt T0} precede those assigned to domain {\tt T1} in the BDD variable
order for this Datalog analysis.
Using a `{\tt x}' between two domains, such as {\tt E0xE1}, means that the
BDD variables assigned to domains {\tt E0} and {\tt E1}
will be interleaved in the BDD variable order for this Datalog analysis.
See Section \ref{sec:bdd} for more details on BDD representations.

Each Datalog analysis rule is a Horn clause of the form
``{\tt R(t) :- R1(t1), ..., Rn(tn)}"
meaning that if relations {\tt R1}, ..., {\tt Rn} contain tuples {\tt t1}, ..., {\tt tn}
respectively, then relation {\tt R} contains tuple {\tt t}.
A backslash may be used at the end of a line to break long rules for readability.
The Datalog analysis solver bddbddb used in Chord does not apply any
sophisticated optimizations to simplify the rules; besides the BDD variable order,
the manner in which these rules are expressed heavily affects the performance of
the solver.  For instance, an important manual optimization involves breaking down
long rules into multiple shorter rules communicating via intermediate relations.
See Section \ref{sec:tuning-datalog} for hints on tuning the performance
of Datalog analyses.

Running a Datalog analysis is no different from running any other kind of analysis
in Chord.  See Chapter \ref{chap:running} for how to run an analysis.

\section{Tuning Performance}
\label{sec:tuning-datalog}

There are several tricks analysis writers can try to improve the
performance of bddbddb, the Datalog solver used by Chord, often
by several orders of magnitude.
Try these tricks by running the following command:

\begin{framed}
\begin{verbatim}
ant -Ddlog.file=<FILE> -Dwork.dir=<DIR> solve
\end{verbatim}
\end{framed}

where {\tt <FILE>} is the file defining the Datalog analysis
to be tuned, and {\tt <DIR>} is the directory containing the
program domains ({\tt *.dom} files) and program relations ({\tt *.bdd} files)
consumed by the analysis (this is by default the
\code{chord_output/bddbddb/} directory generated
by a previous run of Chord.

\begin{enumerate}

\item

Set properties \verb+noisy=yes+, \verb+tracesolve=yes+, and \verb+fulltracesolve=yes+
on the above command line and observe which rule gets ``stuck" (i.e., takes several seconds to solve).
\verb+fulltracesolve+ is seldom useful, but \verb+noisy+ and \verb+tracesolve+ are
often very useful.  Once you identify the rule that is getting stuck, it
will also tell you which relations and which domains used in that rule,
and which operation on them, is taking a long time to solve.  Then try
to fix the problem with that rule by doing either or both of the following:
\begin{itemize}
\item
Break down the rule into multiple rules by creating intermediate relations (the more
relations you have on the RHS of a rule the slower it generally takes to solve
that rule).
\item
Change the relative order of the domains of those
relations in the BDD variable order
(note that you can use either `\_' or `x' between a pair of domains).
\end{itemize}

\item

Once you have ensured that none of the rules is getting ``stuck",
you will notice that some rules are applied too many times, and so
although each application of the rule itself isn't taking too much
time, the cumulative time for the rule is too much.  After finishing
solving a Datalog analysis, bddbddb prints how long each rule took to
solve (both in terms of the number of times it was applied and the
cumulative time it took).  It sorts the rules in the order of the
cumulative time.  You need to focus on the rules that took the most
time to solve (they will be at the bottom of the list).  Assuming you
removed the problem of rules getting ``stuck", the rules will roughly
be in the order of the number of times they were applied.  Here is an
example:

\begin{framed}
\begin{verbatim}
OUT> Rule VH(u:V0,h:H0) :- VV(u:V0,v:V1), VH(v:V1,h:H0), VHfilter
(u:V0,h:H0).
OUT>    Updates: 2871
OUT>    Time: 6798 ms
OUT>    Longest Iteration: 0 (0 ms)
OUT> Rule IM(i:I0,m:M0) :- reachableI(i:I0), specIMV(i:I0,m:M0,v:V0), VH(v:V0,_:H0).
OUT>    Updates: 5031
OUT>    Time: 6972 ms
OUT>    Longest Iteration: 0 (0 ms)
\end{verbatim}
\end{framed}

Notice that the second rule was applied 5031 times whereas the first
was applied 2871 times.  More importantly, the second rule took 6972
milliseconds in all, compared to 6798 for the first rule.  Hence, you
should focus on the second rule first, and try to speed it up.  This
means that you should focus only on relations IM, reachableI, specIMV,
and VH, and the domains I0, M0, V0, and H0.  Any changes you make that
do not affect these relations and domains are unlikely to make your
solving faster.  In general, look at the last few rules, not just the
last one, and try to identify the ``sub-analysis" of the Datalog analysis
that seems problematic, and then focus on speeding up just that sub-
analysis.

\item

You can add the \verb+.split+ keyword at the end of certain rules as a
hint to bddbddb to decompose those rules into simpler ones that can be
solved faster.  You can also set property \verb+split_all_rules=yes+ as shorthand
for splitting all rules without adding the \verb+.split+ keyword to any of
them, though I seldom find splitting all rules helpful.

\item

You can try to decompose a single Datalog analysis file into two separate Datalog analysis
files.  Of course, you cannot separate mutually-recursive rules into two
different analyses, but if you unnecessarily club
together rules that could have gone into different analyses, then they
can put conflicting demands on bddbddb (e.g., on the BDD variable order).
So if rule 2 uses the result of rule 1 and rule 1 does not use the result of
rule 2, then put rule 1 and rule 2 in separate Datalog analyses.

\item

Observe the sizes of the BDDs representing the relations that are
input and output.  bddbddb prints both the number of tuples in each
relation and the number of nodes in the BDD.  Try changing the
BDD variable order for the domains of the relation, and observe how the
number of nodes in the BDD for that relation change.  You will
notice that some orders perform remarkably better than others.  Then
note down these orders as invariants that you will not violate as
you tweak other things.

\item

The relative order of values *within* domains (e.g.,
in domains named \verb+M+, \verb+H+, \verb+C+, etc. in Chord) affects the
performance of bddbddb, but
I've never tried changing this and studying its effect.  It might be
worth trying.  For instance, John Whaley's PLDI'04 paper describes a
specific way in which he numbers contexts (in domain \verb+C+) and that it was
fundamental to the speedup of his ``infinity"-CFA points-to analysis.

\item

Finally, it is worth emphasizing that BDDs are not magic.
If your algorithm itself is fundamentally hard to scale, then BDDs are
unlikely to help you a whole lot.  Secondly, many things are awkward to
encode as integers (e.g., the abstract contexts in domain \verb+C+ 
in Chord) or as Datalog rules.
For instance, I've noticed that summary-based context-sensitive program
analyses are hard to express in Datalog.  The may-happen-in-parallel
analysis provided in Chord shows a relatively simple kind of summary-based
analysis that uses the Reps-Horwitz-Sagiv tabulation algorithm.  But this
is as far as I could get---more complicated summary-based algorithms are
best written in Java itself instead of Datalog.
\end{enumerate}

\section{BDD Representation}
\label{sec:bdd}

Each domain containing N elements is assigned log2(N) BDD variables in the underlying BDD factory with contiguous IDs.
For instance,
domain {\tt F0} containing [128..256) elements will be assigned 8 variables with IDs (say) 63,64,65,66,67,68,69,70 and
domain {\tt Z0} containing [8..16) elements will be assigned 4 variables with IDs (say) 105,106,107,108.

If two domains are uninterleaved in the declared domain order in a Datalog program (i.e., `{\tt \_}' is used instead of `{\tt x}' between them),
then the BDD variables assigned to those domains are ordered in reverse order in the underlying BDD factory.
For instance, the BDD variable order corresponding to the declared domain order {\tt F0\_Z0} is (in level2var format)
``70,69,68,67,66,65,64,63,108,107,106,105".

If two domains are interleaved in the declared domain order in a Datalog program (i.e., `{\tt x}' is used instead of `{\tt \_}' between them),
then the BDD variables assigned to those domains are still ordered in reverse order of IDs in the underlying BDD factory,
but they are also interleaved.
For instance, the BDD variable order corresponding to the declared domain order {\tt F0xZ0} is (in level2var format)
``70,108,69,107,68,106,67,105,66,65,64,63".

Each BDD variable is at a unique ``level" which is its 0-based position in the BDD variable order in the underlying BDD factory.

We will next illustrate the format of a BDD stored on disk (in a .bdd file) using the following example:

\begin{framed}
\begin{verbatim}
# V0:16 H0:12
# 14 15 16 17 18 19 20 21 22 23 24 25 26 27 28 29
# 82 83 84 85 86 87 88 89 90 91 92 93
28489 134
39 36 33 30 27 24 21 18 15 12 9 6 3 0 81 79 77 75 73 71 69 67 65 63 61 59 57 55  \
   53 51 82 80 78 76 74 72 70 68 66 64 62 60 58 56 54 52 37 34 31 28 25 22 19 16 \
   13 10 7 4 1 117 116 115 114 113 112 111 110 109 108 107 106 50 49 48 47 46 45 \
   44 43 42 41 40 105 104 103 102 101 100 99 98 97 96 95 94 93 92 91 90 89 88 87 \
   86 85 84 83 133 132 131 130 129 128 127 126 125 124 123 122 121 120 119 118   \
   38 35 32 29 26 23 20 17 14 11 8 5 2
287 83 0 1
349123 84 287 0
349138 85 0 349123
...
\end{verbatim}
\end{framed}

The first comment line indicates the domains in the relation (in the above case, {\tt V0} and {\tt H0},
represented using 16 and 12 unique BDD variables, respectively).

If there are $N$ domains, there are $N$ following comment lines, each specifying the
BDD variables assigned to the corresponding domain.

The following line has two numbers: the number of nodes in the represented BDD (28489 in this case) and the number of variables
in the BDD factory from which the BDD was dumped to disk.  Note that the number of variables (134 in this case) is not
necessarily the number of variables in the represented BDD (16+12=28 in this case) though it is guaranteed to be
greater than or equal to it.

The following line specifies the BDD variable order in var2level format.  In this case, the specified order subsumes
{\tt V0\_H0} (notice that levels ``81 79 77 75 73 71 69 67 65 63 61 59 57 55 53 51", which are at positions ``14 15 ... 28 29" 
in the specified order are lower than levels ``105 104 103 102 101 100 99 98 97 96 95 94" which are at positions
``82 83 .. 92 93"). 

Each of the following lines specifies a unique node in the represented BDD; it has format ``X V L H" where:
\begin{itemize}
\item X is the ID of the BDD node
\item V is the ID of the bdd variable labeling that node (unless it is 0 or 1, in which case it represents a leaf node)
\item L is the ID of the BDD node's low child
\item H is the ID of the BDD node's high child
\end{itemize}

The order of these lines specifying BDD nodes is such that the lines specifying the BDD nodes in the L and H entries
of each BDD node are guaranteed to occur before the line specifying that BDD node (for instance, the L entry 287 on
the second line and the R entry 349123 on the third line are IDs of BDD nodes specified on the first and second lines,
respectively).

Note on Terminology: 
The {\it support} of a BDD {\tt b} is another BDD {\tt r} = {\tt b.support()}
that represents all the variables used in {\tt b}.
The support BDD {\tt r} is a linear tree each of whose nodes contains a separate variable,
the low branch is 0, and high branch is the node representing the next variable.
To list all the variables used in a BDD {\tt b} use {\tt r.scanSet()}.
Needless to say, {\tt scanSet()} simply walks along the high branches
starting at the root of BDD {\tt r}.



\end{document}
