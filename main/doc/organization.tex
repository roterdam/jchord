\xname{organization}
\chapter{Organization of this Guide}
\label{chap:organization}

Chord is a program analysis platform that is both {\it stand-alone}, in that it
provides many standard analyses for users to run, as well as {\it extensible},
in that it allows users to write their own analyses, possibly atop the provided
analyses.

Hence, this user guide consists of two parts: a
\xlink{guide for end-users}{enduser\_guide.html} and a
\xlink{guide for developers}{developer\_guide.html}.

End-users are those users who only wish to run predefined analyses in Chord
whereas developers are those users who additionally wish to write their own
analyses in Chord.

Unlike end-users, developers need to understand Chord's source code and API, and
code written by them is executed as part of a Chord run.

Hence, the part of the guide for end-users concerns how to run Chord, and the
part of the guide for developers concerns how to extend Chord.

