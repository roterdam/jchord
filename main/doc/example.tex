\xname{example}
\chapter{An Example Run}
\label{chap:example}

This chapter describes how to setup a Java program and run predefined analyses
in Chord on it.

Suppose the Java program to be analyzed has the following directory structure:

\begin{framed}
\begin{verbatim}
    example/
        src/
            foo/
                Main.java
                ...
        classes/
            foo/
                Main.class
                ...
        lib/
            src/
                taz/
                    ...
            jar/
                taz.jar
        chord.properties
\end{verbatim}
\end{framed}

The above structure is typical: the program's Java source
files are under {\tt src/}, its class files are under {\tt classes/},
the source and jar files of the libraries used by the program are
under \code{lib/src/} and \code{lib/jar/}, respectively.

To run Chord on the above program, run the following command in
Chord's main directory:

\begin{framed}
\begin{verbatim}
ant -Dchord.work.dir=<...>/example run
\end{verbatim}
\end{framed}

where \code{<...>} denotes the absolute or relative path of the
directory containing the above \code{example/} directory.
Property \code{chord.work.dir} specifies the working directory during Chord's execution.
Chord searches for a file named \code{chord.properties} under that directory, and
loads upfront all properties defined in that file, if it exists.  Alternatively, these properties
may be defined on the command-line, using the \code{-D<key>=<value>} format.
A sample \code{chord.properties} file for the above program is as follows:

\begin{framed}
\begin{verbatim}
    chord.main.class=foo.Main
    chord.class.path=classes:lib/jar/taz.jar
    chord.src.path=src:lib/src
    chord.run.ids=0,1
    chord.args.0="-thread 1 -n 10"
    chord.args.1="-thread 2 -n 50"
\end{verbatim}
\end{framed}

Each relative file/directory name in the value of any property
defined in this file (e.g., the \code{lib/src} directory name in the value of
property \code{chord.src.path} above) is treated relative to Chord's working directory
(which is specified by property \code{chord.work.dir}).

Chapter \ref{chap:properties} presents an exhaustive listing of
properties recognized by Chord.  Here, we only describe those defined
in the above sample \code{chord.properties} file:

\begin{itemize}
\item
\code{chord.main.class} specifies the fully-qualified name of the main
class of the program.
\item
\code{chord.class.path} specifies the application classpath
of the program (the standard library classpath is implicitly
included).
\item
\code{chord.src.path} specifies the Java source path of the program.
All program analyses in Chord operate on Java bytecode.  The only use
of this property is to HTMLize the Java source files of the program so
that the results of program analyses can be reported at the Java
source code level.
\item
\code{chord.run.ids} specifies a list of IDs to identify runs of the
program.  It is used by dynamic program analyses to determine how many
times the program must be run.  An additional use of this property is
to allow specifying the command-line arguments to use in the run
having ID {\tt <id>} via property \code{chord.args.<id>}, as
illustrated by properties \code{chord.args.0} and \code{chord.args.1}
in the above example.
\end{itemize}

In the above case, Chord does not do much beyond loading the above
\code{chord.properties} file.  For Chord to do something interesting,
additional properties must be set that specify the task(s)
Chord must perform.  All tasks are summarized in Section \ref{sec:func-props}.
The most common task is to run one or more analyses on an input Java program.
Chord implements many standard analyses, including various may-alias and
call-graph analyses, thread-escape analysis, various concurrency analyses,
etc. Many of these analyses are described in Chapter \ref{chap:predefined-analyses}.
Moreover, users can extend these analyses or define their own ones, as
explained in Chapter \ref{chap:analyses}.
Broadly, there are two kinds of analyses in Chord:

\begin{itemize}
\item
Those expressed imperatively in Java. They are in \code{*.java} files under
directory \code{main/src/chord/analyses/}, and are of the form:

\begin{framed}
\begin{verbatim}
    @Chord(
        name=<JAVA ANALYSIS NAME>
        ...
    )
    public class XYZ extends ... {
        ...
    }
\end{verbatim}
\end{framed}

\item

Those expressed declaratively in Datalog.  They are in \code{*.dlog} files under
directory \code{main/src/chord/analyses/}, and are of the form:

\begin{framed}
\begin{verbatim}
    # name=<DLOG ANALYSIS NAME>

    # Program domains
    .include ...
    .include ...

    # BDD variable order
    .bddvarorder ...
\end{verbatim}
\end{framed}
\end{itemize}

To run one or more such analyses, run the following command in Chord's main directory:

\begin{framed}
\begin{verbatim}
ant -Dchord.work.dir=<PROGRAM DIR> -Dchord.run.analyses=<ANALYSIS NAME> run
\end{verbatim}
\end{framed}

where {\tt <PROGRAM DIR>} denotes a directory containing a Java program as described above,
and {\tt <ANALYSIS NAME>} denotes the name of an analysis to run,
such as {\tt <JAVA ANALYSIS NAME>} and {\tt <DLOG ANALYSIS NAME>} in the Java and
Datalog analyses depicted above.

For instance, to run a basic may-alias and call-graph analysis (called 0CFA),
run the following command in Chord's main directory:

\begin{framed}
\begin{verbatim}
ant -Dchord.work.dir=<PROGRAM DIR> -Dchord.run.analyses=cipa-0cfa-dlog run
\end{verbatim}
\end{framed}

This instructs Chord to run the Datalog analysis named \code{cipa-0cfa-dlog},
which happens to be defined in file \code{main/src/chord/analyses/alias/cipa_0cfa.dlog}.

The output of the above command is of the form:

\begin{framed}
{\small
\begin{verbatim}
    Buildfile: build.xml

    run:
         [java] Chord run initiated at: Mar 13, 2011 10:31:08 PM
         [java] ENTER: cipa-0cfa-dlog
         [java] ENTER: T
         [java] ENTER: RTA
         [java] Iteration: 0
         [java] Iteration: 1
         [java] Iteration: 2
         [java] LEAVE: RTA
         [java] SAVING dom T size: 1386
         [java] LEAVE: T
         [java] ENTER: F
         [java] SAVING dom F size: 4120
         [java] LEAVE: F
         ...
         [java] ENTER: MputStatFldInst
         [java] SAVING rel MputStatFldInst size: 739
         [java] LEAVE: MputStatFldInst
         [java] ENTER: statIM
         [java] SAVING rel statIM size: 3359
         [java] LEAVE: statIM
         [java] Starting command: 'java ... chord_analyses_alias_cipa_0cfa.dlog'
         [java] Relation VH: 541 nodes, 449.0 elements (V0,H0)
         [java] Relation FH: 137 nodes, 8.0 elements (H0,F0)
         [java] Relation HFH: 199 nodes, 35.0 elements (H0,F0,H1)
         [java] Relation IM: 590 nodes, 69.0 elements (I0,M0)
         [java] Finished command: 'java ... chord_analyses_alias_cipa_0cfa.dlog'
         [java] LEAVE: cipa-0cfa-dlog
         [java] Chord run completed at: Mar 13, 2011 10:31:36 PM
         [java] Total time: 00:00:27:671 hh:mm:ss:ms

    BUILD SUCCESSFUL
    Total time: 28 seconds
\end{verbatim}
}
\end{framed}

Notice that although Chord is only asked to run the 0CFA analysis on the command line, it automatically
runs many other analyses before finally running the desired analysis and terminating.
Each analysis in Chord is written modularly, independent of other analyses, along with lightweight
annotations specifying the inputs and outputs of the analysis.  Chord's runtime automatically computes
producer-consumer relationships between analyses (e.g., determines which analysis produces as output a
result that is needed as input by another analysis).  Before running a desired analysis
(such as 0CFA in the above example), Chord recursively runs other analyses until the
inputs to the desired analysis have been computed; it finally runs the desired analysis to produce
the outputs of that analysis.

In the above example, the 0CFA analysis consumes {\it program relations} 
such as \code{MputStatFldInst} and \code{statIM}, each of which is produced by a separate Java analysis with the
corresponding name, and produces program relations \code{VH}, \code{FH}, \code{HFH}, and \code{IM}.

The program relations consumed by the 0CFA analysis contain basic program facts.  For instance,
\code{MputStatFldInst} is a relation containing each tuple ($m$,$f$,$v$) such that method $m$ in the input Java
program contains a putstatic instruction of the form ``$f$ = $v$", while \code{statIM} is a relation
containing each tuple ($i$,$m$) such that $m$ is the target method of invokestatic instruction $i$.

The program relations produced by the 0CFA analysis represent points-to information and the
call graph of the input Java program as computed by the analysis.
Specifically, relations \code{VH}, \code{FH}, and \code{HFH} represent points-to information 
for local variables, static fields, and instance fields, respectively,
while relation \code{IM} represents the call graph, namely, it contain each tuple ($i$,$m$) such that
$m$ is a possible target method of call site $i$.

Metavariables $m$, $f$, $i$, and $v$ above range over entities in so-called {\it program domains} \code{M},
\code{F}, \code{I}, and \code{V}, respectively.
A program domain is a set of entities of a certain kind in the input Java program.  For instance,
\code{M} is the domain representing
the set of all methods in the input Java program, \code{F} is the domain representing the set of all fields,
\code{I} is the domain representing
the set of all call sites in methods in \code{M}, and \code{V} is the domain representing the set of all local variables of reference type in methods in \code{M}.
Each of these domains is produced by a separate Java analysis with the corresponding name.
Notice that the analyses producing these domains run upfront because these domains
are consumed by the analyses that produce relations such as \code{MputStatFldInst} and
\code{statIM}, which in turn are consumed by the desired 0CFA analysis.

During execution, Chord dumps intermediate and final results to files in the
directory specified by property \code{chord.out.dir}, whose default value is \code{[chord.work.dir]/chord_output/}
and typically does not need to be changed by users.
For the above example, this directory is {\tt <PROGRAM DIR>}\code{/chord_output/}.

The verbosity of Chord's output above is controlled by property \code{chord.verbose}, whose
default value is 1.  At verbosity level 0, the above command produces less voluminous output
of the form:

\begin{framed}
{\small
\begin{verbatim}
    Buildfile: build.xml
    
    run:
         [java] Chord run initiated at: Mar 13, 2011 10:35:01 PM
         [java] Chord run completed at: Mar 13, 2011 10:35:28 PM
         [java] Total time: 00:00:26:297 hh:mm:ss:ms

    BUILD SUCCESSFUL
    Total time: 26 seconds
\end{verbatim}
}
\end{framed}
