\chapter{Getting Started}
\label{chap:getting-started}

This chapter describes how to download, install, and run Chord.
Section \ref{sec:downloading-binaries} describes how to obtain pre-built binaries of Chord.
Section \ref{sec:downloading-sources} describes how to obtain the source code of Chord
and Section \ref{sec:downloading-sources} explains how to build it.
Finally, Section \ref{sec:running-chord} describes how to run Chord.

\section{Downloading Binaries}
\label{sec:downloading-binaries}

To obtain Chord's pre-built binaries, download file \xlink{chord-bin-2.0.tar.gz}{http://jchord.googlecode.com/files/chord-bin-2.0.tar.gz}.
It contains the following files:

\begin{enumerate}
\item
\code{chord.jar}, which contains the class files of Chord and of libraries used by Chord.
\item
\code{libbuddy.so}, \code{buddy.dll}, and \code{libbuddy.dylib}: you can keep one of these files
depending upon whether you intend to run Chord on Linux, Windows/Cygwin, or MacOS, respectively.
These files are needed only if you want to use the BuDDy BDD library for executing program
analyses written in Datalog.
\item
\code{libchord_instr_agent.so}: this file is needed only if you want to use the JVMTI-based bytecode
instrumentation agent for executing dynamic program analyses in Chord.
\end{enumerate}

Novice users can ignore items (2) and (3) until they become more familiar with Chord.
The binaries mentioned in items (2) and (3) might not be compatible with your machine, in which case you
can either forgo using them (with hardly any noticeable difference in functionality),
or you can download the sources (see Section \ref{sec:downloading-sources}) and build them yourself
(see Section \ref{sec:compiling-sources}).

\section{Downloading Source Code}
\label{sec:downloading-sources}

To obtain Chord's source code, download the following files:

\begin{itemize}
\item
Mandatory: file \xlink{chord-src-2.0.tar.gz}{http://jchord.googlecode.com/files/chord-src-2.0.tar.gz}, which contains Chord's source code and the jars of libraries used by Chord.
\item
Optional: file \xlink{chord-libsrc-2.0.tar.gz}{http://jchord.googlecode.com/files/chord-libsrc-2.0.tar.gz}, which contains the source
code of libraries used by Chord (e.g., joeq, javassist, bddbddb, etc.)
\end{itemize}

Alternatively, you can obtain the latest development snapshot from the SVN repository:

\begin{verbatim}
    prompt> svn checkout http://jchord.googlecode.com/svn/trunk/ chord
\end{verbatim}

Instead of checking out the entire \code{trunk/}, which contains several sub-directories, you can check out
specific sub-directories:

\begin{itemize}
\item
\code{main/} contains Chord's source code and the jars of libraries used by Chord.
\item
\code{libsrc/} contains the source code of libraries used by Chord (e.g., joeq, javassist, bddbddb, etc.).
\item
\code{test/} contains Chord's regression tests.
\item
many more; these might eventually move into \code{main/}.
\end{itemize}

Files \code{chord-2.0-src.tar.gz} and \code{chord-2.0-libsrc.tar.gz} mentioned above are essentially stable releases of
the \code{main/} and \code{libsrc/} directories, respectively. 

\section{Compiling the Source Code}
\label{sec:compiling-sources}

Compiling Chord's source code requires the following software:

\begin{itemize}
\item
A JVM with JDK 5 or higher, e.g. \xlink{IBM J9}{http://www.ibm.com/developerworks/java/jdk/} or
\xlink{Oracle HotSpot}{http://www.oracle.com/technetwork/java/javase/downloads/index.html}.
\item
\xlink{Apache Ant}{http://ant.apache.org/}, a Java build tool.
\end{itemize}

Directory {\tt main/} contains a file named {\tt build.xml} which is
interpreted by Apache Ant.  To see the various targets available, run
the following command in that directory:

\begin{verbatim}
    prompt> ant
\end{verbatim}

To compile Chord, run the following command in the same directory:

\begin{verbatim}
    prompt> ant compile
\end{verbatim}

This will compile Chord's Java sources from \code{main/src/} to class files
in \code{main/classes/}, as well as build a jar file \code{main/chord.jar} containing
these class files as well as the class files in the jars of libraries that are used by Chord
and are provided under \code{main/lib/} (e.g., \code{joeq.jar}, \code{javassist.jar},
\code{bddbddb.jar}, etc.).
Additionally:

\begin{itemize}
\item

If system property \code{chord.use.buddy} is set to \code{true},
then the C source code of the \xlink{BuDDy}{http://buddy.sourceforge.net/} BDD library
from \code{main/bdd/} will be compiled to a shared library in \code{main/}
(\code{libbuddy.so} on Linux, \code{buddy.dll} on Windows, and
\code{libbuddy.dylib} on MacOS); this library is used for executing
program analyses written in Datalog
using \xlink{bddbddb}{http://bddbddb.sourceforge.net/} (a BDD-based Datalog solver).

\item

If system property \code{chord.use.jvmti} is set to \code{true},
then the C++ source code of the JVMTI-based bytecode instrumentation agent from
\code{main/agent/} will be compiled to a shared library in \code{main/}
(\code{libchord_instr_agent.so} on all architectures); this agent is
used for computing program scope dynamically and for executing
dynamic program analyses.
\end{itemize}

Properties \code{chord.use.buddy} and \code{chord.use.jvmti} are defined in file \code{main/chord.properties}.  The
default value of both these properties is \code{false}.
If you set either of them to \code{true}, then you will also need a utility like GNU Make (to run the
\code{Makefile}'s under \code{main/bdd/} and \code{main/agent/}) and a C++ compiler.

\section{Running Chord}
\label{sec:running-chord}

Running Chord requires a JVM with JDK 5 or higher.
There are two equivalent ways to run Chord.
One way is to run the following command in the \code{main/} directory:

\begin{verbatim}
    prompt> ant -D<key1>=<value1> ... -D<keyN>=<valueN> run
\end{verbatim}

The above requires Apache Ant (a Java build tool) to be installed on your machine.  The alternative, which
does not require Apache Ant, is to run the following command:

\begin{verbatim}
    prompt> java -cp <...>/chord.jar -D<key1>=<value1> ... -D<keyN>=<valueN> chord.project.Boot
\end{verbatim}

where \code{<...>} denotes the absolute or relative path of the directory containing file \code{chord.jar};
that directory is also expected to contain any other binaries in Chord's installation
(e.g., \code{libbuddy.so} and \code{libchord_instr_agent.so}).

Each ``\code{-D<key>=<value>}" argument above sets the system property named \code{<key>} to
the value denoted by \code{<value>}.  The only way to specify inputs to Chord is via system properties; there is no
command-line argument processing.
Chapter\ref{chap:properties} describes all system properties recognized by Chord.

