\section{Tuning a Datalog Analysis}
\label{sec:tuning-datalog-analysis}

There are several tricks analysis writers can try to improve the
performance of \bddbddb, the Datalog solver used by \Chord, often
by several orders of magnitude:

\begin{enumerate}
\item
Set system properties \verb+noisy=yes+, \verb+tracesolve=yes+, and \verb+fulltracesolve=yes+
while running \bddbddb\ and observe which rule gets ``stuck" (i.e., takes several seconds to solve).
\verb+fulltracesolve+ is seldom useful, but \verb+noisy+ and \verb+tracesolve+ is
often very useful.  Once you identify the rule that is getting stuck, it
will also tell you which relations and which domains used in that rule
and which operation on them is taking a long time to solve.  Then try
to fix the slowdown problem with that rule by either simplifying the
relations involved (e.g., breaking their set of attributes into two if
possible) or changing the relative order of the domains of those
relations in the BDD variable order (doing the latter alone
frequently causes the problem to go away).

\item
Once you have ensured that none of the rules is getting ``stuck",
you will notice that some rules are applied too many times, and so
although each application of the rule itself isn't taking too much
time, the cumulative time for the rule is too much.  After finishing
solving a Datalog analysis, \bddbddb\ prints how long each rule took to
solve (both in terms of the number of times it was applied and the
cumulative time it took).  It sorts the rules in the order of the
cumulative time.  You need to focus on the rules that took the most
time to solve (they will be at the bottom of the list).  Assuming you
removed the problem of rules getting ``stuck", the rules will roughly
be in the order of the number of times they were applied.  Here is an
example:

\begin{verbatim}
OUT> Rule VH(u:V0,h:H0) :- VV(u:V0,v:V1), VH(v:V1,h:H0), VHfilter
(u:V0,h:H0).
OUT>    Updates: 2871
OUT>    Time: 6798 ms
OUT>    Longest Iteration: 0 (0 ms)
OUT> Rule IM(i:I0,m:M0) :- reachableI(i:I0), specIMV(i:I0,m:M0,v:V0), VH(v:V0,_:H0).
OUT>    Updates: 5031
OUT>    Time: 6972 ms
OUT>    Longest Iteration: 0 (0 ms)
\end{verbatim}

Notice that the second rule was applied 5031 times whereas the first
was applied 2871 times.  More importantly, the second rule took 6972
milliseconds in all, compared to 6798 for the first rule.  Hence, you
should focus on the second rule first, and try to speed it up.  This
means that you should focus only on relations IM, reachableI, specIMV,
and VH, and the domains I0, M0, V0, and H0.  Any changes you make that
do not affect these relations and domains are unlikely to make your
solving faster.  In general, look at the last few rules, not just the
last one, and try to identify the "sub-program" of the Datalog program
that seems problematic, and then focus on speeding up just that sub-
program.

\item
Tweak the BDD variable ordering in the Datalog file (the line that
starts with \verb+bddvarorder+).  This is one of the most
effective ways of making \bddbddb\ run faster.  As explained above, use
\verb+noisy+ and \verb+tracesolve+ to find out which pairs of domains in the
bddvarorder are responsible for the slowdown, and try to change their
relative ordering (note that you can use either `\_' or `x' between a pair
of domains, and the latter is commutative).

\item
You can add the \verb+.split+ keyword at the end of certain rules as a
hint to \bddbddb\ to break up those rules into simpler ones that can be
solved faster.  You can also set property \verb+split_all_rules=yes+ as shorthand
for splitting all rules without adding the \verb+.split+ keyword to any of
them, though I seldom find splitting all rules helpful.

\item
You can yourself break down rules by creating intermediate relations (the more
relations you have on the RHS of a rule the slower it takes to solve
that rule).  This is another very effective way to make \bddbddb\ run
faster.

\item
Try breaking down a single Datalog program into two programs.  Of
course, you cannot separate mutually-recursive rules into two
different programs, but if you unnecessarily club
together rules that could have gone into different programs, then they
can put conflicting demands on \bddbddb\ (e.g., on \verb+bddvarorder+).
So if rule 2 uses the result of rule 1 and rule 1 does not use the result of
rule 2, then put rule 1 and rule 2 in two different Datalog programs.

\item
Observe the sizes of the BDDs representing the relations that are
input and output.  \bddbddb\ tells both the number of tuples in each
relation and the number of "nodes" in the BDD.  Try changing the
bddvarorder for the domains of the relation, and observe how the
number of "nodes" in the bdd for that relation change.  You will
notice that some orderings do remarkably better than others.  Then
note down these orderings as invariants that you will not violate as
you tweak other things.

\item
The relative ordering of values *within* domains (e.g.
in domains named \verb+M+, \verb+H+, \verb+C+, etc. in Chord) affects the
solving time of \bddbddb, but
I've never tried changing this and studying its effect.  It might be
worth trying.  For instance, John Whaley's PLDI'04 paper describes a
specific way in which he numbers contexts (in domain \verb+C+) and that it was
fundamental to the speedup of his infinity-CFA points-to analysis.

\item
Finally, it is worth emphasizing that BDDs are not magic.
If your algorithm itself is fundamentally hard to scale, then BDDs are
unlikely to help you a whole lot.  Secondly, many things are awkward to
encode as integers (e.g., the abstract contexts in the domain \verb+C+ 
in Chord) or as Datalog rules.
For instance, I've noticed that summary-based context-sensitive program
analyses are hard to express in Datalog.  The may-happen-in-parallel
analysis provided in Chord shows a relatively simple kind of summary-based
analysis that uses the Reps-Horwitz-Sagiv tabulation algorithm.  But this
is as far as I could get---more complicated summary-based algorithms are
best written in Java itself instead of Datalog.
\end{enumerate}
