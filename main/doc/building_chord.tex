\section{Building Chord}
\label{sec:building-chord}

Compiling Chord's source installation requires the following software:

\begin{itemize}
\item
A JVM with JDK 5 or higher, e.g. \xlink{IBM J9}{http://www.ibm.com/developerworks/java/jdk/} or
\xlink{Oracle HotSpot}{http://www.oracle.com/technetwork/java/javase/downloads/index.html}.
\item
\xlink{Apache Ant}{http://ant.apache.org/}, a Java build tool.
\end{itemize}

Directory {\tt main/} contains a file named {\tt build.xml} which is
interpreted by Apache Ant.  To see the various targets available, run
the following command in that directory:

\begin{verbatim}
    prompt> ant
\end{verbatim}

To compile Chord, run the following command in the same directory:

\begin{verbatim}
    prompt> ant compile
\end{verbatim}

This will compile Chord's Java sources from \code{main/src/} to class files
in \code{main/classes/}, as well as build a jar file \code{main/chord.jar} containing
these class files as well as the class files in the jars of libraries that are used by Chord
and are provided under \code{main/lib/} (e.g., \code{joeq.jar}, \code{javassist.jar},
\code{bddbddb.jar}, etc.).
Additionally:
\begin{itemize}
\item
If system property \code{chord.use.buddy} is set to \code{true} (default is \code{false}),
then the C source code of the \xlink{BuDDy}{http://buddy.sourceforge.net/} BDD library
from \code{main/bdd/} will be compiled to a shared library in \code{main/}
(\code{libbuddy.so} on Linux, \code{buddy.dll} on Windows, and
\code{libbuddy.dylib} on MacOS); this library is used for executing
program analyses written in Datalog
using \xlink{bddbddb}{http://bddbddb.sourceforge.net/} (a BDD-based Datalog solver).
\item
If system property \code{chord.use.jvmti} is set to \code{true} (default is \code{false}),
then the C++ source code of the JVMTI-based bytecode instrumentation agent from
\code{main/agent/} will be compiled to a shared library in \code{main/}
(\code{libchord_instr_agent.so} on all architectures); this agent is
used for computing program scope dynamically and for executing
dynamic program analyses.
\end{itemize}
The above system properties can be defined in file \code{main/chord.properties}
or on the command-line of ant (using the \code{-D<key>=<value>} format).  If either of these
properties is set to \code{true}, then you will also need a utility like GNU Make (to run the
\code{Makefile}'s under \code{main/bdd/} and \code{main/agent/}) and a C++ compiler.

