\xname{relation}
\chapter{Program Relation}
\label{chap:relation}

Finally, Section \ref{sec:bdd} explains the
representation of BDDs (Binary Decision Diagrams) which are a data structure
central to executing Datalog analyses.

\section{Constructing a Relation}

1. Using ProgramRel
2. Produced by analysis
3. Manual

\section{Operations on a Relation}

\section{BDD Representation Internals}
\label{sec:bdd}

Each domain containing N elements is assigned log2(N) BDD variables in the
underlying BDD factory with contiguous IDs.
For instance,
domain {\tt F0} containing [128..256) elements will be assigned 8 variables with
IDs (say) 63,64,65,66,67,68,69,70 and
domain {\tt Z0} containing [8..16) elements will be assigned 4 variables with
IDs (say) 105,106,107,108.

If two domains are uninterleaved in the declared domain order in a Datalog
program (i.e., `{\tt \_}' is used instead of `{\tt x}' between them),
then the BDD variables assigned to those domains are ordered in reverse order in
the underlying BDD factory.
For instance, the BDD variable order corresponding to the declared domain order
{\tt F0\_Z0} is (in level2var format)
``70,69,68,67,66,65,64,63,108,107,106,105".

If two domains are interleaved in the declared domain order in a Datalog program
(i.e., `{\tt x}' is used instead of `{\tt \_}' between them),
then the BDD variables assigned to those domains are still ordered in reverse
order of IDs in the underlying BDD factory,
but they are also interleaved.
For instance, the BDD variable order corresponding to the declared domain order
{\tt F0xZ0} is (in level2var format)
``70,108,69,107,68,106,67,105,66,65,64,63".

Each BDD variable is at a unique ``level" which is its 0-based position in the
BDD variable order in the underlying BDD factory.

We will next illustrate the format of a BDD stored on disk (in a .bdd file)
using the following example:

\begin{framed}
\begin{verbatim}
# V0:16 H0:12
# 14 15 16 17 18 19 20 21 22 23 24 25 26 27 28 29
# 82 83 84 85 86 87 88 89 90 91 92 93
28489 134
39 36 33 30 27 24 21 18 15 12 9 6 3 0 81 79 77 75 73 71 69 67 65 63 61 59 57 55
\
   53 51 82 80 78 76 74 72 70 68 66 64 62 60 58 56 54 52 37 34 31 28 25 22 19 16
\
   13 10 7 4 1 117 116 115 114 113 112 111 110 109 108 107 106 50 49 48 47 46 45
\
   44 43 42 41 40 105 104 103 102 101 100 99 98 97 96 95 94 93 92 91 90 89 88 87
\
   86 85 84 83 133 132 131 130 129 128 127 126 125 124 123 122 121 120 119 118
\
   38 35 32 29 26 23 20 17 14 11 8 5 2
287 83 0 1
349123 84 287 0
349138 85 0 349123
...
\end{verbatim}
\end{framed}


The first comment line indicates the domains in the relation (in the above case,
{\tt V0} and {\tt H0},
represented using 16 and 12 unique BDD variables, respectively).

If there are $N$ domains, there are $N$ following comment lines, each specifying
the
BDD variables assigned to the corresponding domain.

The following line has two numbers: the number of nodes in the represented BDD
(28489 in this case) and the number of variables
in the BDD factory from which the BDD was dumped to disk.  Note that the number
of variables (134 in this case) is not
necessarily the number of variables in the represented BDD (16+12=28 in this
case) though it is guaranteed to be
greater than or equal to it.

The following line specifies the BDD variable order in var2level format.  In
this case, the specified order subsumes
{\tt V0\_H0} (notice that levels ``81 79 77 75 73 71 69 67 65 63 61 59 57 55 53
51", which are at positions ``14 15 ... 28 29"
in the specified order are lower than levels ``105 104 103 102 101 100 99 98 97
96 95 94" which are at positions
``82 83 .. 92 93").

Each of the following lines specifies a unique node in the represented BDD; it
has format ``X V L H" where:
\begin{itemize}
\item X is the ID of the BDD node
\item V is the ID of the bdd variable labeling that node
\item L is the ID of the BDD node's low child or 0 or 1
\item H is the ID of the BDD node's high child or 0 or 1
\end{itemize}

The order of these lines specifying BDD nodes is such that the lines specifying
the BDD nodes in the L and H entries
of each BDD node are guaranteed to occur before the line specifying that BDD
node (for instance, the L entry 287 on
the second line and the R entry 349123 on the third line are IDs of BDD nodes
specified on the first and second lines,
respectively).

Note on Terminology:
The {\it support} of a BDD {\tt b} is another BDD {\tt r} = {\tt b.support()}
that represents all the variables used in {\tt b}.
The support BDD {\tt r} is a linear tree each of whose nodes contains a separate
variable,
the low branch is 0, and high branch is the node representing the next variable.
To list all the variables used in a BDD {\tt b} use {\tt r.scanSet()}.
Needless to say, {\tt scanSet()} simply walks along the high branches
starting at the root of BDD {\tt r}.

