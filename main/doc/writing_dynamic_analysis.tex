\section{Writing a Dynamic Program Analysis}
\label{sec:writing-dynamic-analysis}

{\bf Step 1: Setting up the Instrumentation Scheme}

Determine the instrumentation scheme required by your dynamic analysis, that is,
the kind and format of events to be generated during an instrumented program's
execution. See class \verb+chord.instr.InstrScheme+ for the kinds of supported
events and their formats.

{\bf Step 2: Implementing the Analysis}

Create a class extending \verb+chord.project.DynamicAnalysis+ and overriding
the appropriate methods in it.

The only method that must be compulsorily overridden is method \verb+getInstrScheme()+,
which must return an instance of the instrumentation scheme chosen in Step 1 above,
plus each \verb+process<event>(<args>)+ method that corresponds to event \verb+<event>+
with format \verb+<args>+ enabled by the chosen instrumentation scheme.
All other methods are no-ops if not overridden.

A sample such class called \verb+MyDynamicAnalysis+ is shown below:

\begin{verbatim}
    import chord.project.DynamicAnalysis;
    import chord.instr.InstrScheme;

    // ***TODO***: analysis won't be recognized by Chord without this annotation
    @Chord(name=<name-of-analysis>)    
    
    public class MyDynamicAnalysis extends DynamicAnalysis {
        InstrScheme scheme;

        public InstrScheme getInstrScheme() {
            if (scheme != null)
                return scheme;
            scheme = new InstrScheme();
            // ***TODO***: Choose (<event1>, <args1>), ... (<eventN>, <argsN>)
            // depending upon the kind and format of events required by this
            // dynamic analysis to be generated for this during an instrumented
            // program's execution.
            scheme.set<event1>(<args1>);
            ...
            scheme.set<eventN>(<argsN>);
            return scheme;
        }

        public void initAllPasses() {
            // ***TODO***: User code to be executed once and for all
            // before all instrumented program runs start.
        }

        public void doneAllPasses() {
            // ***TODO***: User code to be executed once and for all
            // after all instrumented program runs finish.
        }

        public void initPass() {
            // ***TODO***: User code to be executed once before each
            // instrumented program run starts.
        }

        public void donePass() {
            // ***TODO***: User code to be executed once after each
            // instrumented program run finishes.
        }

        // User-defined event handlers for this dynamic analysis.
        // No-ops if not overridden.
        public void process<event1>(<args1>) {
            // ***TODO***: User code for handling events of kind <event1> with format <args1>.
        }
        ...
        public void process<eventN>(<argsN>) {
            // ***TODO***: User code for handling events of kind <eventN> with format <argsN>.
        }
    }
\end{verbatim}

{\bf Step 3: Choosing or Implementing the Runtime Event Handler}

Determine the runtime event handler for your dynamic analysis. There are two kinds of runtime event
handlers, offline and online; see package \verb+chord.runtime+ for more details.

For most dynamic analyses, an offline event hander suffices, in which case you don't need to do
anything in this step (except that you may want to change the default values of system properties
\verb+chord.trace.pipe+ and \verb+chord.trace.block.size+ for performance; see
Section \ref{sec:chord-sysprops} for the meaning of these properties). Otherwise, you need to
create a class extending class \verb+chord.runtime.Runtime+ and override the appropriate methods
in it, and provide the fully-qualified name of the created class as the value of system property
\verb+chord.runtime.class+ (whose default value is the class \verb+chord.runtime.BufferedRuntime+).

{\bf Step 4: Compiling the Analysis}

Compile the analysis by placing the directory containing class \verb+MyDynamicAnalysis+ created
above in the path defined by system property \verb+chord.java.analysis.path+.

{\bf Step 5: Specifying Program Inputs}

Provide the IDs of program runs to be generated (say 1, 2, ..., M) and the command-line arguments to be
used for the program in each of those runs (say <args1>, ..., <argsM>) via system properties
\verb+chord.run.ids=1,2,...,N+ and \verb+chord.args.1=<args1>+, ..., \verb+chord.args.M=<argsM>+.

By default, \verb+chord.run.ids=0+ and \verb+chord.args.0=""+, that is, the program will be run only
once (using run ID 0) with no command-line arguments.

{\bf Step 6: Running the Analysis}

To run the analysis, set system property \verb+chord.run.analyses=<name-of-analysis>+ where
\verb+<name-of-analysis>+ is the name that was provided in the \verb+@Chord+ annotation for class
\verb+MyDynamicAnalysis+ created above.
