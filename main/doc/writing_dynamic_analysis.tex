\section{Writing a Dynamic Analysis}
\label{sec:writing-dynamic-analysis}

Follow the following steps to write your own dynamic analysis.

\subsection{Implementing the Analysis}

Create a class extending \code{chord.project.analyses.DynamicAnalysis} and override
the appropriate methods in it.
The only methods that must be compulsorily overridden are method \code{getInstrScheme()},
which must return an instance of the ``instrumentation scheme" required by
your dynamic analysis (i.e., the kind and format of events to be generated during an
instrumented program's execution)
plus each \code{process<event>(<args>)} method that corresponds to event {\tt <event>}
with format {\tt <args>} enabled by the chosen instrumentation scheme.
See Section \ref{sec:instr-events} for the kinds of supported events and their formats.

A sample such class \code{MyDynamicAnalysis} is shown below:

\begin{verbatim}
    import chord.project.analyses.DynamicAnalysis;
    import chord.instr.InstrScheme;

    // ***TODO***: analysis won't be recognized by Chord without this annotation
    @Chord(name=<name-of-analysis>)    
    public class MyDynamicAnalysis extends DynamicAnalysis {
        InstrScheme scheme;

        @Override
        public InstrScheme getInstrScheme() {
            if (scheme != null)
                return scheme;
            scheme = new InstrScheme();
            // ***TODO***: Choose (<event1>, <args1>), ... (<eventN>, <argsN>)
            // depending upon the kind and format of events required by this
            // dynamic analysis to be generated for this during an instrumented
            // program's execution.
            scheme.set<event1>(<args1>);
            ...
            scheme.set<eventN>(<argsN>);
            return scheme;
        }

        @Override
        public void initAllPasses() {
            // ***TODO***: User code to be executed once and for all
            // before all instrumented program runs start.
        }

        @Override
        public void doneAllPasses() {
            // ***TODO***: User code to be executed once and for all
            // after all instrumented program runs finish.
        }

        @Override
        public void initPass() {
            // ***TODO***: User code to be executed once before each
            // instrumented program run starts.
        }

        @Override
        public void donePass() {
            // ***TODO***: User code to be executed once after each
            // instrumented program run finishes.
        }

        // User-defined event handlers for this dynamic analysis.
        // No-ops if not overridden.
        @Override
        public void process<event1>(<args1>) {
            // ***TODO***: User code for handling events of kind
            // <event1> with format <args1>.
        }
        ...
        @Override
        public void process<eventN>(<argsN>) {
            // ***TODO***: User code for handling events of kind
            // <eventN> with format <argsN>.
        }
    }
\end{verbatim}

\subsection{Configuring the Analysis}

You can change the default values of various properties for configuring your dynamic analysis;
see Sections \ref{sec:scope-props} and \ref{sec:instr-props}.
For instance:
\begin{itemize}
\item
You can set property \code{chord.scope.kind} to {\tt dynamic} so that the program scope
is computed dynamically (i.e., by running the program) instead of statically.
\item
You can exclude certain classes (e.g., JDK classes) from being instrumented by setting
properties \code{chord.std.scope.exclude}, \code{chord.ext.scope.exclude}, and
\code{chord.scope.exclude}.
\item
You can choose between online (i.e. load-time) and offline bytecode instrumentation by
setting property \code{chord.instr.kind} to {\tt online} or {\tt offline}.
\item
You can require the event-generating and event-handling JVMs to be one and the same (by setting
property \code{chord.trace.kind} to {\tt none}),
or to be separate (by setting property {\tt chord.trace.kind} to {\tt full} or {\tt pipe}, depending upon
whether you want the two JVMs to exchange events by a regular file or a POSIX pipe, respectively).
Using a single JVM can cause correctness/performance issues if event-handling Java code itself is instrumented
(e.g., say the event-handling code uses class \code{java.util.ArrayList} which is not excluded from program scope).
Using separate JVMs prevents such issues since the event-handling JVM runs uninstrumented bytecode (only
the event-generating JVM runs instrumented bytecode).
If a regular file is used to exchange events between the two JVMs, then the JVMs run serially:
the event-generating JVM first runs to completion, dumps the entire dynamic trace to the regular file,
and then the event-handling JVM processes the dynamic trace.
If a POSIX pipe is used to exchange events between the two JVMs, then the JVMs run in lockstep.
Obviously, a pipe is more efficient for very long traces, but it 
not portable (e.g., it does not currently work on Windows/Cygwin), and the traces cannot be reused
across Chord runs (see the following item).
\item
You can reuse dynamic traces from a previous Chord run by setting property
\code{chord.reuse.traces} to {\tt true}.  In this case, you must also set property
{\tt chord.trace.kind} to {\tt full}.
\item
You can set property \code{chord.dynamic.haltonerr} to {\tt false} to prevent Chord from terminating
even if the program on which dynamic analysis is being performed crashes.
\end{itemize}

Chord offers much more flexibility in crafting dynamic analyses.
You can define your own instrumentor (by subclassing \code{chord.instr.CoreInstrumentor}
instead of using the default \code{chord.instr.Instrumentor}) and your own event handler (by subclassing \code{chord.runtime.CoreEventHandler}
instead of using the default \code{chord.runtime.EventHandler}).
You can ask the dynamic analysis to use your custom instrumentor and/or your custom event handler by overriding
methods \code{getInstrumentor()} and \code{getEventHandler()}, respectively, defined in \code{chord.project.analyses.CoreDynamicAnalysis}.
Finally, you can define your own dynamic analysis template by subclassing \code{chord.project.analyses.CoreDynamicAnalysis}
instead of subclassing the default \code{chord.project.analyses.DynamicAnalysis}.

\subsection{Compiling and Running the Analysis}

Compile the analysis by placing the directory containing class \code{MyDynamicAnalysis} created
above in the path defined by property \code{chord.java.analysis.path}.

Provide the IDs of program runs to be generated (say 1, 2, ..., M) and the command-line arguments to be
used for the program in each of those runs (say \code{<args1>}, ..., \code{<argsM>}) via properties
\code{chord.run.ids=1,2,...,N} and \code{chord.args.1=<args1>}, ..., \code{chord.args.M=<argsM>}.
By default, \code{chord.run.ids=0} and \code{chord.args.0=""}, that is, the program will be run only
once (using run ID 0) with no command-line arguments.

To run the analysis, set property \code{chord.run.analyses} to \code{<name-of-analysis>}
(recall that \code{<name-of-analysis>} is the name provided in the \code{@Chord} annotation for class
\code{MyDynamicAnalysis} created above).

