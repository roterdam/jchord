\section{Datarace Analysis}
\label{sec:datarace}

Run the datarace analysis provided in \Chord\ by running the following command in \Chord's {\tt main/} directory:

\begin{quote}
\begin{verbatim}
prompt> ant -Dchord.work.dir=<...> -Dchord.run.analyses=datarace-java run
\end{verbatim}
\end{quote}

\noindent where directory {\tt <...>} contains a file named \code{chord.properties} which
defines properties \code{chord.main.class}, \code{chord.class.path}, and \code{chord.src.path}.
See Section \ref{sec:chord-sysprops} for the meaning of these properties.

Directory \code{main/examples/datarace_test/} provides a toy Java program on which you can run the datarace
analysis.
First run {\tt ant} in that directory (in order to compile the program's {\tt .java} files to
{\tt .class} files) and then run the above command in Chord's {\tt main/} directory
with {\tt <...>} replaced by \code{examples/datarace_test/}.
Upon successful completion, the following files should be produced in directory
\code{main/examples/datarace_test/chord_output/}:

\begin{itemize}
\item
File \code{dataraces_by_fld.html}, listing all dataraces grouped by the field on which they occur; all
dataraces on the same instance field or the same static field are listed in the same group, and so are
all dataraces on array elements.
\item
File \code{dataraces_by_obj.html}, listing all dataraces grouped by the abstract object on whose field they occur;
dataraces on all static fields are listed in the same group, and so are dataraces on different
instance fields of the same abstract object.
\end{itemize}


