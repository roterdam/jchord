\section{Download and Installation}
\label{sec:download-and-installation}

Ensure that the following software is installed on your machine:

\begin{itemize}
\item
JDK 5 or higher, e.g. from
\htmladdnormallink{IBM}{http://www.ibm.com/developerworks/java/jdk/} or
\htmladdnormallink{Sun}{http://java.sun.com/javase/downloads/index.jsp}
\item
\htmladdnormallink{Apache Ant}{http://ant.apache.org/}, a Java-based build tool
\item
a C++ compiler, e.g. \htmladdnormallink{GCC}{http://gcc.gnu.org/}
\item
a Make utility, e.g. \htmladdnormallink{GNU Make}{http://www.gnu.org/software/make/}
\item
\htmladdnormallink{Cygwin}{http://www.cygwin.com/}, if it is a Windows machine
\end{itemize}

Download the latest Chord source code from the SVN repository:

\begin{verbatim}
    prompt> svn checkout http://jchord.googlecode.com/svn/trunk/ chord
\end{verbatim}
For many users, it may be sufficient to only checkout directory {\tt
  trunk/main/} instead of the entire {\tt trunk/}.  Other directories
of interest are {\tt trunk/test/}, which contains regression tests,
and {\tt trunk/libsrc/}, which contains the source code of software
used by Chord (Joeq, Javassist, bddbddb, JavaBDD, and jwutil).  All
other directories under {\tt trunk/} are projects built atop Chord
that might be of interest to some users.

Directory {\tt main/} contains a {\tt build.xml} file which is
interpreted by Apache Ant.  To see the various targets available, run
the following command in that directory:

\begin{verbatim}
    prompt> ant
\end{verbatim}

\noindent To compile \Chord, run the following command in the same
directory:

\begin{verbatim}
    prompt> ant compile
\end{verbatim}

\noindent This will compile the following:
\begin{itemize}
\item
the Java source code of \Chord\ from \code{main/src/} to Java bytecode
in \code{main/classes/}.
\item
the C source code of BDD library
\htmladdnormallink{BuDDy}{http://buddy.sourceforge.net/} from
\code{main/bdd/} to a shared library in \code{main/lib/}
(\code{libbuddy.so} on Linux, \code{buddy.dll} on Windows, and
\code{libbuddy.dylib} on MacOS); this library is needed for executing
program analyses written in Datalog using
\htmladdnormallink{bddbddb}{http://bddbddb.sourceforge.net/} (a
BDD-based Datalog solver).
\item
the C++ source code of the Chord instrumentation agent from
\code{main/agent/} to a shared library in \code{main/lib/}
(\code{libchord_instr_agent.so} on all architectures); this agent is
needed for computing program scope dynamically and for executing
dynamic program analyses.
\end{itemize}

