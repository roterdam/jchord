\section{Download and Installation}
\label{sec:download-and-installation}

Ensure that the following software is installed on your machine:

\begin{itemize}
\item
JDK 5 or higher, e.g. from \htmladdnormallink{IBM}{http://www.ibm.com/developerworks/java/jdk/} or
\htmladdnormallink{Sun}{http://java.sun.com/javase/downloads/index.jsp}
\item
\htmladdnormallink{Apache Ant}{http://ant.apache.org/}, a Java-based build tool
\item
a C++ compiler, e.g. \htmladdnormallink{GCC}{http://gcc.gnu.org/}
\item
a Make utility, e.g. \htmladdnormallink{GNU Make}{http://www.gnu.org/software/make/}
\item
\htmladdnormallink{Cygwin}{http://www.cygwin.com/}, if it is a Windows machine
\end{itemize}

Download a \Chord\ source release from \url{http://code.google.com/p/jchord/downloads/list} or
the latest source from the SVN repository at \url{http://code.google.com/p/jchord/source/checkout}.

Directory \code{main/} contains a \code{build.xml} file which is interpreted by Apache Ant.
To see the various targets available, run the following command in that directory:

\begin{quote}
\begin{verbatim}
prompt> ant
\end{verbatim}
\end{quote}

\noindent To compile \Chord, run the following command in the same directory:

\begin{quote}
\begin{verbatim}
prompt> ant compile
\end{verbatim}
\end{quote}

\noindent This will compile the following:
\begin{itemize}
\item
the Java source code of \Chord\ from \code{main/src/java/} to Java bytecode in \code{main/classes/}
\item
the C source code of BDD library \htmladdnormallink{BuDDy}{http://buddy.sourceforge.net/} from \code{main/src/bdd/}
to a shared library in \code{main/lib/} (\code{libbuddy.so} on Linux, \code{buddy.dll} on Windows, and
\code{libbuddy.dylib} on MacOS);
this library is needed for executing program analyses written in Datalog using
\htmladdnormallink{bddbddb}{http://bddbddb.sourceforge.net/} (a BDD-based Datalog solver).
\item
the C++ source code of the Chord instrumentation agent from \code{main/src/agent/} to a shared
library in \code{main/lib/} (\code{libchord_instr_agent.so} on all architectures);
this agent is needed for building program scope dynamically
and for executing dynamic program analyses.
\end{itemize}

