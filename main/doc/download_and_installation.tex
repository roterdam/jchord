\section{Download and Installation}
\label{sec:download-and-installation}

You can obtain pre-built binaries as well as the source code of Chord.
The following sections explain both options in detail.

\subsection {Binary Installation}
\label{sec:binary-install}

To obtain Chord's pre-built binaries, download file \xlink{chord-bin-2.0.tar.gz}{http://jchord.googlecode.com/files/chord-bin-2.0.tar.gz}.
It contains the following files:

\begin{enumerate}
\item
\code{chord.jar}, which contains the class files of Chord and of libraries used by Chord.
\item
\code{libbuddy.so}, \code{buddy.dll}, and \code{libbuddy.dylib}: you can keep one of these files
depending upon whether you intend to run Chord on Linux, Windows/Cygwin, or MacOS, respectively.
These files are needed only if you want to use the BuDDy BDD library for executing program
analyses written in Datalog.
\item
\code{libchord_instr_agent.so}: this file is needed only if you want to use the JVMTI-based bytecode
instrumentation agent for executing dynamic program analyses in Chord.
\end{enumerate}

\noindent Novice users can ignore items (2) and (3) above until they become more familiar with Chord.
The provided binaries in those items might not be compatible with your machine, in which case you
will either have to forgo using them (with hardly any noticeable difference in Chord's functionality)
or you will have to build them yourself (see Section \ref{sec:source-install}).

\subsection{Source Installation}
\label{sec:source-install}

To obtain Chord's source code, download the following files:

\begin{enumerate}
\item
Mandatory: file \xlink{chord-src-2.0.tar.gz}{http://jchord.googlecode.com/files/chord-src-2.0.tar.gz}, which contains Chord's source code and the jars of libraries used by Chord.
\item
Optional: file \xlink{chord-libsrc-2.0.tar.gz}{http://jchord.googlecode.com/files/chord-libsrc-2.0.tar.gz}, which contains the source
code of libraries used by Chord (e.g., joeq, javassist, bddbddb, etc.)
\end{enumerate}

\noindent Alternatively, you can obtain the latest development snapshot from the SVN repository:

\begin{verbatim}
    prompt> svn checkout http://jchord.googlecode.com/svn/trunk/ chord
\end{verbatim}

\noindent Instead of checking out the entire \code{trunk/}, which contains several sub-directories, you can check out
specific sub-directories:
\begin{itemize}
\item \code{main/} contains Chord's source code and the jars of libraries used by Chord.
\item \code{libsrc/} contains the source code of libraries used by Chord (e.g., joeq, javassist, bddbddb, etc.).
\item \code{test/} contains Chord's regression tests.
\item many more; these might eventually move into \code{main/}.
\end{itemize}

\noindent Files \code{chord-2.0-src.tar.gz} and \code{chord-2.0-libsrc.tar.gz} mentioned above are essentially stable releases of
the \code{main/} and \code{libsrc/} directories, respectively. 

