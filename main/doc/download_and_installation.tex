\section{Download and installation}
\label{sec:download-and-installation}

Ensure that the following software is installed on your machine:

\begin{itemize}
\item
A JVM with JDK 5 or higher, e.g.
\xlink{IBM J9}{http://www.ibm.com/developerworks/java/jdk/} or
\xlink{Oracle HotSpot}{http://www.oracle.com/technetwork/java/javase/downloads/index.html}
\item
\xlink{Apache Ant}{http://ant.apache.org/}, a Java-based build tool
\end{itemize}

{\bf Note:} If you intend to use Chord for dynamic analysis, then
IBM J9 is highly recommended, as it allows you to instrument
the entire JDK; using any other JVM will likely require excluding
large parts of the JDK from being instrumented.

Obtain the latest stable release of Chord from \url{http://jchord.googlecode.com/checkout/}
or the latest development snapshot from the SVN repository: 

\begin{verbatim}
    prompt> svn checkout http://jchord.googlecode.com/svn/trunk/ chord
\end{verbatim}
For many users, it may be sufficient to only checkout directory {\tt
  trunk/main/} instead of the entire {\tt trunk/}.  Other directories
of interest are {\tt trunk/test/}, which contains regression tests,
and {\tt trunk/libsrc/}, which contains the source code of software
used by Chord (Joeq, Javassist, bddbddb, JavaBDD, and jwutil).  All
other directories under {\tt trunk/} are projects built atop Chord
that might be of interest to some users.

Directory {\tt main/} contains a {\tt build.xml} file which is
interpreted by Apache Ant.  To see the various targets available, run
the following command in that directory:

\begin{verbatim}
    prompt> ant help
\end{verbatim}

\noindent To compile Chord, simply run the following command in the same
directory:

\begin{verbatim}
    prompt> ant 
\end{verbatim}

\noindent This will compile
the Java source code of Chord from \code{main/src/} to Java bytecode
in \code{main/classes/}.
Additionally:
\begin{itemize}
\item
If system property \code{chord.use.buddy} is set to true (default is false),
then the C source code of BDD library
\xlink{BuDDy}{http://buddy.sourceforge.net/} from
\code{main/bdd/} will be compiled to a shared library in \code{main/lib/}
(\code{libbuddy.so} on Linux, \code{buddy.dll} on Windows, and
\code{libbuddy.dylib} on MacOS); this library is used for executing
program analyses written in Datalog using
\xlink{bddbddb}{http://bddbddb.sourceforge.net/} (a
BDD-based Datalog solver).
\item
If system property \code{chord.use.jvmti} is set to true (default is false),
then the C++ source code of the Chord instrumentation agent from
\code{main/agent/} will be compiled to a shared library in \code{main/lib/}
(\code{libchord_instr_agent.so} on all architectures); this agent is
used for computing program scope dynamically and for executing
dynamic program analyses.
\end{itemize}
The above system properties can be set either in file \code{main/chord.properties}
(recommended) or on the command line as follows:

\begin{verbatim}
    prompt> ant -Dchord.use.buddy=true -Dchord.use.jvmti=true
\end{verbatim}

