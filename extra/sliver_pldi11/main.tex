%-----------------------------------------------------------------------------
%
%               Template for sigplanconf LaTeX Class
%
% Name:         sigplanconf-template.tex
%
% Purpose:      A template for sigplanconf.cls, which is a LaTeX 2e class
%               file for SIGPLAN conference proceedings.
%
% Author:       Paul C. Anagnostopoulos
%               Windfall Software
%               978 371-2316
%               paul@windfall.com
%
% Created:      15 February 2005
%
%-----------------------------------------------------------------------------


\documentclass[nocopyrightspace,9pt]{sigplanconf}

% The following \documentclass options may be useful:
%
% 10pt          To set in 10-point type instead of 9-point.
% 11pt          To set in 11-point type instead of 9-point.
% authoryear    To obtain author/year citation style instead of numeric.

\usepackage{amsmath,amssymb,amsthm,graphicx,subfig,multirow,rotating,paralist,color,verbatim}
\usepackage[utf8]{inputenc}

\begin{document}

\conferenceinfo{PLDI'11,} {June 4--8, 2011, San Jose, California, USA.}
\CopyrightYear{2011}
\copyrightdata{978-1-4503-0663-8/11/06}

%\titlebanner{banner above paper title}        % These are ignored unless
%\preprintfooter{short description of paper}   % 'preprint' option specified.

\title{Scaling Abstraction Refinement via Pruning}

\authorinfo{Percy Liang}
           {UC  Berkeley}
           {pliang@cs.berkeley.edu}
\authorinfo{Mayur Naik}
           {Intel Labs Berkeley}
           {mayur.naik@intel.com}

\input std-macros
\input macros

\maketitle

\category{D.2.4}{Software Engineering}{Software/Program Verification}

\terms
Algorithms, Experimentation, Theory, Verification

\keywords
heap abstractions, static analysis, concurrency, iterative refinement, pruning, slicing

\input abstract
\input introduction
\input preliminaries
\input theory
\input klimited
\input abstractions
\input experiments
\input related
\input conclusion

\bibliographystyle{abbrvnat}
\bibliography{main}

\input appendix

\end{document}
