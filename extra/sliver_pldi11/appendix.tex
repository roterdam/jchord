\appendix
\Section{proofs}{Proofs}

Instead of directly proving \refprop{soundness}, we state a more general
theorem which will be useful later:

\begin{theorem}[Soundness]
\label{thm:soundness}
Let $\alpha$ and $\beta$ be two abstractions with $\beta \preceq \alpha$ ($\beta$ is coarser),
and let $X$ be any set of input tuples.
For any derivation $\ba \in \bD(\alpha(X))$,
define $\bb = (b_1, \dots, b_{|\ba|})$ where each $b_i$ is the unique element in $\beta(a_i)$.
Then $\bb \in \bD(\beta(X))$.
\end{theorem}
\begin{proof}[Proof of \refthm{soundness}]
Define $A=\alpha(X)$ and $B=\beta(X)$.
Consider $\ba \in \bD(A)$ and let $\bb$ be as defined in the theorem.
For each position $i$,
we have two cases.
First, if $a_i \in A$, then $b_i \in (\beta \circ \alpha)(X) = \beta(X) = B$.
Otherwise, let $z \in \sZ$ be the rule and $J$ be the indices of the tuples used to derive $a_i$.
The same rule $z$ and the corresponding tuples $\{ b_j : j \in J \}$ can also be used to derive $b_i$.
Therefore, $\bb \in \bD(B)$.
\end{proof}

\begin{proof}[Proof of \refprop{soundness} (abstraction is sound)]
Apply \refthm{soundness} with $\beta = \alpha$ and $\alpha$ as the identity function (no abstraction).
\end{proof}

%%%%%%%%%%%%%%%%%%%%%%%%%%%%%%

Before we prove \refthm{master}, we state a useful lemma.
\begin{lemma}[Pruning is idempotent]
\label{lem:Pidem}
For any set of tuples (concrete or abstract) $X$, $\bP(X) = \bP(\bP(X))$.
\end{lemma}
\begin{proof}[Proof]
Since $\bP(X) \subset X$ by definition and $\bP$ is monotonic,
we have $\bP(\bP(X)) \subset \bP(X)$.
For the other direction, let $x \in \bP(X)$.
Then $x$ is part of some derivation ($x \in \bx \in \bD(X)$).
All the input tuples of $\bx$ (those in $\bx\cap X$) are also in $\bP(X)$,
so $\bx \in \bD(\bP(X))$.  Therefore $x \in \bP(\bP(X))$.
\end{proof}

\begin{proof}[Proof of \refthm{master} (pruning is sound and complete)]
We define variables for the intermediate quantities in \refeqn{master}:
$A = \alpha(X)$ and $B = \beta(X)$, $\tilde B = \bP(B)$, $\tilde A = \alpha(\tilde B)$,
and $A' = A \cap \tilde A$.
We want to show that pruning is sound ($\bP(A) \subset \bP(A')$)
and complete ($\bP(A) \supset \bP(A')$).
Completeness follows directly because $A \supset A'$ and $\bP$ is monotonic (increasing
the number of input tuples can only increase the number of drived tuples).

Now we show soundness.
Let $a \in \bP(A)$.  By definition of $\bP$ (\refeqn{Pdef}), there is a derivation $\ba \in \bD(A)$ containing $a$.
For each $a_i \in \ba$, let $b_i$ be the unique element in $\beta(a_i)$ (a singleton set because $\beta \preceq \alpha$),
and let $\bb$ be the corresponding sequence constructed from the $b_i$s.
Since $\beta \preceq \alpha$, we have $\bb \in \bD(B)$ by \refthm{soundness},
and so each input tuple in $\bb$ is also in $\bP(B) = \tilde B$;
in particular, $b = \tilde B$ for $\beta(a) = \{ b \}$.
Since $\beta \preceq \alpha$, $a \in \alpha(b)$, and so $a \in \tilde A$.
We have thus shown that $\bP(A) \subset \tilde A$.
Finishing up, $\bP(A') = \bP(A \cap \tilde A) \supset \bP(A \cap \bP(A)) = \bP(\bP(A)) = \bP(A)$,
where the last equality follows from idempotence (\reflem{Pidem}).
\end{proof}

%Intuition: derive everything we could before
%\begin{theorem}[Relevance]
%\label{thm:relevance}
%Let $\alpha_1 \preceq \alpha_2$ be two abstractions ($\alpha_2$ is finer).
%Let $X$ be any set of tuples.
%Let $A_1 = \alpha_1(X)$ and $A_2 = \alpha_2(X)$ being the abstract tuples.
%Let $\tilde A_1 = \bP(A_1)$ be the relevant abstract values with respect to $\alpha_1$.
%Let $\tilde A_2 = \alpha_2(\tilde A_1)$ be the corresponding abstract values with respect to $\alpha_2$.
%Then for any sequence $(b_1, \dots, b_n)$ with $b_n = \xo$,
%\[ (b_1, \dots, b_n) \in \bD(A_2) \implies (b_1, \dots, b_n) \in \bD(\tilde A_2). \]
%Compactly, $\bD(\alpha_2(X)) \subset \bD(\alpha_2(\bP(\alpha_1(X))))$.
%\end{theorem}
%
%\begin{proof}[Proof of \refthm{relevance} (relevance)]
%Consider any derivation $(b_1, \dots, b_n) \in \bD(A_2)$ with $b_n = \xo$.
%For each $i = 1, \dots, n$, let $a_i = \alpha_1(b_i)$.
% By the soundness theorem, $(a_1, \dots, a_n) \in \bD(A_1)$.
%Because $(a_1, \dots, a_n)$ is a derivation of $a_n = \xo$,
% all these tuples are used, so
% each $a_i \in \bR(A_1) = \tilde A_1$.
%We now argue that $(b_1, \dots, b_n) \in \bD(\tilde A_2)$.
% For each $i = 1, \dots, n$:
%  If $b_i \in A_2$, then $a_i \in A_1$.  Since $a_i$ is part of a derivation of $\xo$ under $\alpha_1$,
%   we also have $a_i \in \tilde A_1$.
%   Note that $a_i \supset b_i$ because $\alpha_1 \preceq \alpha_2$.
%   Therefore, $b_i \subset a_i \subset \cup_{a \in \tilde A_1} a$.
%   Applying $\alpha_2$ to both sides yields $b_i \in \tilde A_2$.
%  If $b_i \not\in A_2$, it must have been derived due to some rule applied to $(J,i) \in P(b_1, \dots, b_n)$.
%   The same rule still applies.
%\end{proof}

We now show that the Prune-Refine algorithm is correct, which follows from a
straightforward application of \refthm{master}.

\begin{proof}[Proof of \refthm{algorithm} (correctness of the Prune-Refine algorithm)]
First, we argue that pre-pruning is correct.
For each iteration $t$, we invoke \refthm{master} with $\alpha = \alpha_t,
\beta = \beta_t$ and $X$ be such that $\alpha(X) = A_t$.
The result is that $\bP(A_t) = \bP(A_t')$, so without loss of generality,
we will assume $A_t = A_t'$ for the rest of the proof.

Now fix an iteration $t$.
We will show that $\bP(\alpha_t(X)) = \bP(A_t)$ by induction,
where the inductive hypothesis is $\bP(\alpha_t(X)) = \bP(\alpha_t(\tilde A_s))$.
For the base case ($s = -1$), we define $\tilde A_{-1} = \{X\}$, we get a tautology.
For the inductive case, apply the theorem with 
applied to $\beta = \alpha_s, \alpha = \alpha_t$ and $X$ such that $\alpha_s(X) = \alpha_s(\tilde A_{s-1})$,
we get that 
\[ \bP(\alpha_t(\tilde A_{s-1})) = \bP(\alpha_t(\bP(\alpha_s(\tilde A_{s-1})))) = \bP(\alpha_t(\tilde A_s)). \]
When $s = t-1$, we have $\bP(\alpha_t(X)) = \bP(\alpha_t(\tilde A_{t-1})) = \bP(A_t)$,
completing the claim.
Finally, if the algorithm returns {\em proven}, we have
\[ \emptyset = \bP(A_t) = \bF(\alpha_t(X)) \supset \bF(X). \]
%If the algorithm returns {\em not proven},
%then $A_t \subset \{ \{ x \} : x \in X \}$,
%so $\xo \in \bF(A_t)$ (not proven) implies $\xo \in \bF(X)$ (not proven) by monotonicity of $\bF$.
\end{proof}
