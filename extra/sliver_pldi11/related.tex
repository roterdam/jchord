\Section{related}{Related Work}

% Larger space of abstractions
There is a wealth of literature which attempts to scale static analyses without
sacrificing precision.  One general theme is to work with a flexible family of
abstractions, which in principle allows us to conform to the needs of the
client.  Milanova et al.~\cite{kobj,MilanovaRountevRyder2005} consider
abstractions where each local variable can be independently treated
context-sensitively or context-insensitively, and different $k$ values can be
chosen for different allocation sites.  Lhot\'{a}k and
Hendren~\cite{LhotakHendren2006,LhotakHendren2008} present Paddle, a
parametrized framework for BDD-based, $k$-limited alias analysis.
\cite{smaragdakis11context} scale up $k$-object-sensitivity, increasing $k$ by one
using types rather than allocation sites.  However, in all of this work, which parts of the
abstraction should be more refined is largely left up to the user.

% Client-driven
Client-driven approaches use feedback from a client query to determine what
parts of an abstraction to refine.
Plevyak and Chien~\cite{PlevyakChien1994} use a refinement-based algorithm for
type inference, where context-sensitivity is driven by detecting type conflicts.
Guyer and Lin~\cite{GuyerLin2003} present a pointer analysis for C
which detects loss of precision (e.g., at merge points) and introduce context-sensitivity.
Our method for determining relevant input tuples is similar in spirit but more general.

Section~3.1 of Liang et al.~\cite{liang11minimal} (not the focus of that work)
also computes the set of relevant tuples by running a transformed Datalog program.  However, what
is done with this information is quite different.  \cite{liang11minimal} merely
stops refining the irrelevant sites whereas we actually prune all irrelevant
tuples, thereby exploiting this information fully.  As we saw in
\refsec{experiments}, this difference had major ramifications.

%Similar to iterative refinement client-driven approaches is lazy abstraction
%\cite{henzinger02lazy,mcmillan06lazy} from model checking, which 

% Demand-driven
Demand-driven analyses \cite{HeintzeTardieu2001,ZhengRugina2008} do not refine
the abstraction but rather try to compute an analysis on an existing
abstraction more efficiently.
Sridharan et al.~\cite{SridharanBodik2006} presents an algorithm
which casts pointer analysis as a CFL-reachability problem and relaxes by
introducing additional edges.

Our Prune-Refine algorithm has a flavor of the client-driven approach, in that we do
refine our abstraction, but also of the demand-driven approach, in that we do
not perform a full computation (in particular, ignoring tuples which were
pruned).  There are three important differences between the present work
and the work described earlier:
First, while most of that work is specific to pointer analysis, our Prune-Refine algorithm
is generally applicable to any Datalog program.
Second, while past work is based on selected iterative refinement,
Prune-Refinement is based on generalizing the idea of program slicing.
%Finally, we were able to prove rigorous theoretical guarantees that our algorithm is as precise
%as a more expensive algorithm.

%In contrast to client-driven analyses, which compute an exhaustive solution of
%varying precision, demand-driven approaches compute a partial solution of fixed
%precision.  Heintze and Tardieu's demand-driven alias analysis for
%C~\cite{HeintzeTardieu2001} performs a provably optimal computation to
%determine the points-to sets of variables queried by the client. Their analysis
%is applied to call-graph construction in the presence of function pointers. A
%more recent alias analysis developed by Zheng and Rugina~\cite{ZhengRugina2008}
%uses a demand-driven algorithm that is capable of answering alias queries
%without constructing points-to sets.

% Model checking
%Lazy abstraction \cite{henzinger02lazy} is an abstraction refinement technique
%for model checking by which one starts with a small set of predicates and tries
%to prove a query.  If the proof fails, an abstract counterexample (derivation
%in our terminology) is generated, and predicates are refined on this path.
%One difference is that they refine one predicate at a time.
%
%This work has been extended to interpolants \cite{mcmillan06lazy}, which
%provide a richer (and as they show, more effective) class of abstraction
%predicates.


%Our family of abstractions is still $\sA = \{ \klimabs_k : k = 0, 1 \}$.
%Let $K : \H \to \{ 0, 1, 2, \dots \}$ be a function mapping each site to a $k$ value.
%In \cite{liang11minimal}, we considered the abstraction family:
%\begin{align}
%\klimabs_K(c) = \klimabs_{K(c[1])}(c).
%\end{align}

% Dynamic - lower bound
Other forms of pruning have been implemented in various settings.
\cite{vipindeep05pruning} use dynamic analysis to prune down set of paths and
then focus a static analysis on these paths.
\cite{spoon04ddp} use pruning for type inference in functional languages,
where pruning is simply a heuristic which shortcuts a search algorithm.
As a result, pruning can hurt precision.
One advantage of our pruning approach is that it comes with strong soundness
and completeness guarantees.

%This is another fruitful but orthogonal way to prune.  The intuition is that if
%a query was observed to be true dynamically, there is no point trying to prove
%it statically.
%For example, we can use the dynamic analysis framework of
%\cite{liang10abstraction} to obtain lower bounds, predicates which are
%certainly true.  If we observe dynamically observe that a variable $v$ points
%to an object with abstraction $a$, the static analysis need not even attempt to
%prove that $v$ does not point to $a$.
