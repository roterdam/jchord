\Section{abstractionExamples}{Abstractions}

%Recall from \refsec{preliminaries} that we separate the Datalog program that
%computes the query from the abstraction.  In \refsec{klimited}, we described
%the Datalog programs that computed the queries for our three clients based on
%$\infty$-object-sensitivity.  We now describe the abstractions that we use to
%further abstract this analysis.

We have already defined the $k$-limited abstraction, which corresponds to
$k$-object-sensitivity.  We now present two orthogonal variants of this basic abstraction: one that
additionally limits the repetition of allocation sites (\refsec{barelyAbstraction})
and one that further abstracts allocation sites using type information
(\refsec{classAbstraction}).

%%%%%%%%%%%%%%%%%%%%%%%%%%%%%%
\Subsection{barelyAbstraction}{Barely-repeating $k$-limited abstraction}

When we applied the $k$-limited abstraction in practice,
we noticed empirically that a major reason why it did not scale was
the seemingly unnecessary combinatorial explosion associated with chains formed
by cycling endlessly through the same allocation sites.
For $k$-CFA, this repetition corresponds to recursion.
For $k$-object-sensitivity, this corresponds to recursive allocation, as
illustrated in \reffig{recursion}.\footnote{
Incidentally, the example in the figure also gives an interesting example where
$k$-object-sensitivity for any finite $k$ (no matter how large) is less precise
than $\infty$-object-sensitivity.}
We therefore wish to define an abstraction that not only truncates chains at length $k$ but also truncates
a chain when it starts repeating.

\begin{figure}
\begin{center}
\begin{minipage}{2in}
\begin{verbatim}
class A {
  f() {
0:  v = new A
    if (*) return v
    else return v.f()
  }
}
\end{verbatim}
\end{minipage}
\begin{minipage}{1in}
\begin{verbatim}
1:  x1 = new A
2:  x2 = new A
    y1 = x1.f()
    y2 = x2.f()
\end{verbatim}
\end{minipage}
\end{center}
\caption{\label{fig:recursion} An example illustrating the repetition of allocation sites.
The points-to set of {\tt y1} using $\infty$-object-sensitivity
is $\{\vec{01},\vec{001},\vec{0001},\dots\}$ (any positive number of zeros followed by a 1),
and the points-to set of {\tt y2} is $\{\vec{02},\vec{002},\vec{0002},\dots\}$.
While these two sets are disjoint, if we use a $k$-limited abstraction for any finite $k$,
we would conclude erroneously that both variables might point to $\bzero_k*$, where $\bzero_k$ is a chain of $k$ zeros.
Incidentally, this demonstrates an intrinsic limitation of the $k$-object-sensitivity abstraction.
Using the barely-repeating $k$-limited abstraction, we can increase $k$ while avoiding chains longer than $\vec{00}$
since $\vec{00}$ is barely-repeating.
This results in computational savings, and in this case, in no loss in precision.
}
\end{figure}

For a sequence $c$, we say $c$ is {\em non-repeating} if all its elements are distinct.
We say $c$ is {\em barely-repeating} if (i) $c$ excluding the last element ($c[1..|c|-1]$)
is {\em non-repeating} and (ii) the last element of $c$ is repeated earlier in $c$.
%We say $c$ is {\em barely-repeating} if $c$ excluding the last element ($c[1..|c|-1]$) is {\em non-repeating}.
%Note that a barely-repeating includes the non-repeating case.
Let $\distinct(c)$ be the length of the longest prefix of $c$ that is barely-repeating, if it exists, and $\infty$
otherwise:
\begin{align}
\label{eqn:barelyLen}
\distinct(c) \eqdef
\begin{cases} 
\max_{m' : c[1..m'] \text{ is barely-repeating}} m' & \text{if $m'$ exists}, \\
\infty & \text{otherwise}.
\end{cases}
\end{align}
%\begin{align}
%\label{eqn:barelyLen}
%\distinct(c) \eqdef \max_{m' : c[1..m'] \text{ is barely-repeating}} m'
%\end{align}
For example, $\distinct(\vec{10010}) = 3$ because $\vec{100}$ is barely-repeating, but $\vec{1001}$ is not.

We now define the {\em barely-repeating $k$-limited abstraction} $\klimdabs_k$ as follows:
\begin{align}
\label{eqn:barelyAbstraction}
\klimdabs_k(c) & \eqdef \klimabs_{\min\{k,\distinct(c)\}}(c),
\end{align}
\reffig{barelyExample} shows an example of $\klimdabs_k$.
We show that $\klimdabs_k$ is a valid abstraction:
\begin{proposition}
\label{prop:barelyAbstraction}
The function $\klimdabs_k$ defined in \refeqn{barelyAbstraction} is a valid abstraction
(\refdef{abstraction}).
\end{proposition}
\begin{proof}
We consider two cases: (i) for $\{c\} \in \range(\klimdabs_k)$,
we have $\klimdabs_k(c) = \{c\}$; and (ii) for any $c* \in \range(\klimdabs_k)$,
either $|c| = k$ or $c$ is barely-repeating; in either case,
it is easy to see that any extension $c' \in c*$ will have $\klimdabs_k(c') = c*$.
\end{proof}
Remark: one might wonder why we defined the abstraction using the barely-repeating criterion
as opposed to the simpler non-repeating criterion.
It turns out that using the latter in \refeqn{barelyLen} would not result in a valid abstraction.
If $\klimdabs_k$ were defined using the non-repeating criterion,
then $\klimdabs_3(\vec{00}) = \vec{0}*$.
But for $\vec{01} \in \vec{0}*$, we have $\klimdabs_3(\vec{01}) = \{\vec{01}\} \neq \vec{0}*$.

\begin{figure}
\[
\begin{array}{lllllll}
\{\vec{0}\}    &            &              &            & \{\vec{1}\} \\
\vec{00}*      &            & \{\vec{01}\} &            & \{\vec{10}\} &            & \vec{11}* \\
               &            & \vec{010}*   & \vec{011}* & \vec{100}*   & \vec{101}* &
\end{array}
\]
\caption{\label{fig:barelyExample} For the barely-repeating $k$-limited abstraction
for $\dom = \{0,1\}$ and $k = 3$, we show the equivalence classes under $\klimdabs_k$.
Compare this with the classes for the $k$-limited abstraction (\reffig{repeatingExample}).
Note that, for example,
$\vec{000}*$ and $\vec{001}*$ are collapsed into $\vec{00}*$ since $\vec{000}$ and $\vec{001}$ are not barely-repeating,
but $\vec{00}$ is.
}
\end{figure}

%%%%%%%%%%%%%%%%%%%%%%%%%%%%%%
\Subsection{classAbstraction}{Type-based abstraction}

We now introduce an abstraction that we will use in the pre-pruning step of the Prune-Refine algorithm.
We start by defining an equivalence relation over allocation sites $\dom$,
represented by a function $\hclass : \dom \mapsto \sP(\dom)$
mapping each allocation site $h \in \dom$ to its its equivalence class.
Given such an $\hclass$, we extend it to sequences by elementwise application:
\begin{align}
\hclass(c) = \hclass(c[1]) \times \cdots \times \hclass(c[|c|]), \quad c \in \dom^*.
\end{align}

To construct $\hclass$, we consider using two sources of type information associated
with an allocation site, motivated by \cite{smaragdakis11context}:
\begin{align}
\typeIs(h)  &= \text{declaring type of allocation site $h$} \label{eqn:typeIs} \\
\typeHas(h) &= \text{type of class containing allocation site $h$} \label{eqn:typeHas}
\end{align}
Using these two functions, we can construct three equivalence relations,
$\hclassIs$, $\hclassHas$, and $\hclassIsHas$ as follows:
\begin{align}
\hclass_f(h) = \{ h' : f(h) = f(h') \}, \label{eqn:hclass}
\end{align}
with $f \in \{ \typeIs, \typeHas, \typeIs \times \typeHas \}$.

% Complementary
Now we have three options for $\hclass$ which are complementary to the
$k$-limited abstraction: $\hclass$ abstracts a chain by abstracting each site
uniformly; $\pi_k$ keeps all sites up to a length $k$ prefix precisely but
abstracts away the rest completely.  This complementarity is desirable
for pre-pruning.

Since $\hclass$ is not coarser than $\klimabs_k$, we cannot
use it directly in the Prune-Refine algorithm.
We must compose $\hclass$ with $\klimabs_k$ or $\klimdabs_k$ to yield another
abstraction which will be coarser than $\klimabs_k$ or $\klimdabs_k$,
respectively.  But in what order should we compose?
We must be careful because the composition of two abstractions is not necessarily an abstraction.
The following proposition shows which compositions are valid:
\begin{proposition}
The functions
(i) $\klimabs_k \circ \hclass$ and (ii) $\hclass \circ \klimabs_k$
are valid abstractions (see \refdef{abstraction}) and equivalent;
(iii) $\klimdabs_k \circ \hclass$ is also valid,
but (iv) $\hclass \circ \klimdabs_k$ is not.
\end{proposition}
\begin{proof}
For each of these four composed functions, each set $s$ in the range of the function
must be either of the form $s = w_1 \times \cdots \times w_m$ for $m < k$ (case 1)
or $s = w_1 \times \cdots \times w_m \times \dom^*$ for some $m \le k$ (case 2),
where $w_i \in \range(\hclass)$ for each $i = 1, \dots, m$.

For (i) and (ii), it is straightforward to check
$(\klimabs_k \circ \hclass)(c) = (\hclass \circ \klimabs_k)(c) = s$ for each $c \in s$.
Intuitively, the truncation ($\klimabs_k$) and coarsening ($\tau$) operate
independently and can be interchanged.

For (iii) and (iv), the two dimensions do not act independently;
the amount of truncation depends on the amount of coarsening:
the coarser $\hclass$ is, the more truncation one might need to limit repetitions.
Showing that $\klimdabs_k \circ \hclass$ is valid
proceeds in a similar manner to \refprop{barelyAbstraction}.
If $s$ falls under case 1, note that no $c \in s$ is repeating
because the $w_i$'s must be disjoint; therefore $\klimdabs_k(\tau(c)) = s$.
If $s$ falls under case 2, note that for any $c[1..m] \in s$ must be barely-repeating but any longer prefix is not,
and therefore, $\klimdabs_k(\tau(c)) = s$.

To show that (iv) is not an abstraction,
consider the following counterexample:
let $\dom = \{ 0,1,2 \}$,
and define $\hclass(v) = \dom$ for all $v \in \dom$
(there is one equivalence class).
Consider two elements $\vec{01}$ and $\vec{00}$ under the composed abstraction:
For $\vec{01}$, we have $\klimdabs_3(\vec{01}) = \{ \vec{01 }\}$,
so $\hclass(\klimdabs_3(\vec{01})) = \dom^2$;
for $\vec{00}$, we have $\klimdabs_3(\vec{00}) = \vec{00}*$,
so $\hclass(\klimdabs_3(\vec{00})) = \dom^2 \times \dom^*$.
But $\dom^2 \subsetneq \dom^2 \times \dom^*$ (notably,
the two sets are neither equal nor disjoint),
so $\hclass \circ \klimdabs_3$ does not define an equivalence relation.
\end{proof}

\noindent 
In light of this result, we will use the valid abstractions $\cklimabsk$ and $\cklimdabsk$,
which work by first applying $\hclass$ and then applying $\klimabs_k$ or $\klimdabs_k$.

%%%%%%%%%%%%%%%%%%%%%%%%%%%%%%
%\Subsection{computation}{Computation}
%
%In order to use the PR algorithm (\reffig{pseudocode}),
%we need to implement the various steps efficiently.
%
%We now present techniques for computing 
%
%\paragraph{Pre-pruning step}
%
%Computing $\cklimdabsk$
%
%\paragraph{Pruning step}
%
%However, need to compute $\ext$.
%
%\paragraph{Refinement step}
%
%We use $\alpha_t = \klimdabs_{t+1}$
%and $\beta_t = \klimdabs_{t+1}$
%and we set $\hclass(h)$ to be the dynamic type of allocation site $h$.
%
%Computing $\klimdabs_{k+1}$
%
%Suppose we have we have a current $A_k \subset \range(\klimdabs_k)$.
%For $c \in \dom^*$.
%$\alpha_{k+1}(\{c\}) = $
%Given $c$, we consider
